% This is part of the book TeX for the Impatient.
% Copyright (C) 2003 Paul W. Abrahams, Kathryn A. Hargreaves, Karl Berry.
% See file fdl.tex for copying conditions.

\input macros
\chapter{Usando este livro}

\chapterdef{uselivro}

Este livro \'e um guia e manual do tipo ``fa\c ca voc\^e mesmo'' para o 
\TeX.  Aqui nesta se\c c\~ao informamos como usar o livro para o 
m\'aximo de vantagem.  Recomendamos que voc\^e primeiro leia ou fa\c ca 
uma varredura, na sequ\^encia,~\chapternum{uselivro} 
at\'e~\chapternum{examples}, as quais informam o que voc\^e precisa 
saber para come\c car a usar o \TeX.  Se voc\^e j\'a tem experi\^encia 
usando \TeX, ainda ser\'a \'util saber que tipos de informa\c c\~ao 
est\~ao nessas se\c c\~oes do livro.  As 
se\c c\~oes~\chapternum{concepts}--~\chapternum{tips}, as quais ocupam a 
maioria do resto do livro, s\~ao desenhadas para serem acessadas 
aleatoriamente.  No entanto, se voc\^e \'e o tipo de pessoa que gosta de 
ler manuais de refer\^encia, descobrir\'a que \'e poss\'ivel prosseguir 
sequencialmente se estiver disposto a fazer muitos desvios no in\'icio.  
Em ~\chapterref{usingtex}, ``Usando \TeX'', explica-se como produzir um 
documento do \TeX\ a partir de um arquivo de entrada do \TeX{}.  Tamb\'em 
descrevemos as conven\c c\~oes para a prepara\c c\~ao desse arquivo de entrada, 
explana-se um pouco acerca de como \TeX{} funciona, e informamos sobre 
recursos adicionais que est\~ao dispon\'iveis.  A leitura dessa se\c c\~ao 
ajudar\'a a entender os exemplos na pr\'oxima se\c c\~ao.  ~\chapterref{examples}, 
``Exemplos'', cont\'em uma sequ\^encia de exemplos que ilustram o uso de 
\TeX.  Cada exemplo consiste de uma p\'agina de sa\'ida junto com a entrada 
que usamos para cri\'a-la.  Esses exemplos orientar\~ao e ajudar\~ao a 
localizar o material mais detalhado que voc\^e precisar\'a conforme 
prosseguir.  Ao ver quais comandos s\~ao usados na entrada, voc\^e saber\'a 
onde procurar informa\c c\~oes mais detalhadas sobre como obter os efeitos 
mostrados na sa\'ida.  Os exemplos tamb\'em podem servir como modelos para 
documentos simples, embora devamos adverti-lo de que, como tentamos 
armazenar v\'arios comandos \TeX\ em um pequeno n\'umero de p\'aginas, os 
exemplos n\~ao s\~ao necessariamente ilustra\c c\~oes de um design de documento 
bom ou completo.  Ao ler a explica\c c\~ao de um comando, voc\^e pode encontrar 
alguns termos t\'ecnicos n\~ao familiares.  Em~\chapterref{concepts}, 
``Conceitos'', n\'os definimos e explicamos esses termos.  N\'os tamb\'em 
discutimos outros t\'opicos que n\~ao s\~ao abordados em outras partes no 
livro.  A contracapa do livro cont\'em uma lista de todos os conceitos e 
as p\'aginas em que s\~ao descritos.  Sugerimos que voc\^e fa\c ca uma c\'opia 
dessa lista e mantenha-a por perto para que voc\^e possa identificar e 
procurar imediatamente um conceito desconhecido.

Os comandos do \TeX\ s\~ao seu vocabul\'ario principal, e a maior parte deste 
livro \'e dedicada a explic\'a-los.  Nas se\c c\~oes~\chapternum{paras} 
at\'e~\chapternum{general} n\'os descrevemos os comandos.  Voc\^e 
encontrar\'a informa\c c\~ao geral acerca das descri\c c\~oes dos 
comandos em \xrefpg{cmddesc}.  As descri\c c\~oes dos comandos s\~ao 
organizadas funcionalmente, mais ou menos como um dicion\'ario de 
sin\^onimos.  Portanto, se voc\^e sabe o que deseja fazer, mas n\~ao 
sabe qual comando faz isso para voc\^e, pode usar o \'indice para 
gui\'a-lo ao grupo de comandos certo.  Os comandos que consideramos 
particularmente \'uteis e f\'aceis de entender s\~ao indicados com uma 
m\~ao apontada~(\hand).

\chapterref{capsule}, ``Resumo da c\'apsula de comandos'', \'e um 
\'indice especializado que complementa as descri\c c\~oes mais completas 
nas se\c c\~oes~\chapternum{paras}-\chapternum{general}.  O \'indice 
lista os comandos do \TeX\ em ordem alfab\'etica, com uma breve 
explica\c c\~ao de cada comando e uma refer\^encia \`a p\'agina onde \'e 
descrito mais completamente.  O resumo da c\'apsula o ajudar\'a quando 
voc\^e quiser apenas um lembrete r\'apido do que um comando faz.

\TeX\ \'e um programa complexo que ocasionalmente trabalha sua vontade 
de maneira misteriosa.  Em \chapterref{tips}, ``Dicas e T\'ecnicas'', 
n\'os fornecemos conselhos sobre como resolver uma variedade de 
problemas espec\'ificos que voc\^e pode encontrar de tempos em tempos.  
E se voc\^e est\'a perplexo com as mensagens de erro do \TeX, voc\^e 
encontrar\'a socorro em \chapterref{errors}, ``Entendendo as mensagens 
de erro''.  As guias cinzas na lateral do livro ajudar\~ao a localizar 
rapidamente partes do livro.  Elas dividem o livro nas seguintes partes 
principais:
\olist
\li explica\c c\~oes gerais e exemplos
\li conceitos
\li descri\c c\~oes de comandos (cinco guias mais curtas)
\li conselhos, mensagens de erro e macros do |eplain.tex|
\li resumo da c\'apsula de comandos
\li \'indice
\endolist

Em muitos lugares, fornecemos refer\^encias de p\'aginas para 
^{\texbook} (veja \xrefpg{resources} para uma cita\c c\~ao).  Essas 
refer\^encias se aplicam \`a d\'ecima s\'etima edi\c c\~ao do \texbook.  
Para outras edi\c c\~oes, algumas refer\^encias podem estar desativadas 
para uma p\'agina ou duas.


\section Syntactic conventions

In any book about preparing input for a computer,
it's necessary to indicate clearly the literal characters that should be typed
and to distinguish those characters from the explanatory text.
We use the Computer Modern typewriter font for {\tt literal input
like this}, and also for the names of \TeX{} commands.
When there's any possibility of confusion, we enclose \TeX\
input in single quotation marks, `{\tt like this}'.
However, we occasionally use parentheses when we're indicating single
characters such as (|`|) (you can see why).

For the sake of your eyes we usually just put spaces 
where you should put spaces. In some places where
we need to emphasize the space, however,
we use a `\visiblespace' character
{\catcode `\ =\other\pix^^| |}%
to indicate it.
Naturally enough, this character is called a \emph{visible space}.
\pix^^{spaces//visible}


\section Descriptions of the commands

\xrdef{cmddesc}
Sections~\chapternum{paras}--\chapternum{general} contain
a description of what nearly every \TeX\ command does.  ^^{commands}
Both the primitive commands ^^{primitive//command}
and those of ^{\plainTeX} are covered.
The primitive commands are those built into the \TeX\ computer program, while
the \plainTeX{} commands are defined in a standard file of 
auxiliary definitions (see \xref\plainTeX). 
The only commands we've omitted are those that are used purely locally
in the definition of \plainTeX\ (\knuth{Appendix~B}).
The commands are organized as follows:
\ulist\compact
\li ``Commands for composing paragraphs'', \chapterref{paras},
deal with characters, words, lines, and entire paragraphs.
\li ``Commands for composing pages'', \chapterref{pages},
deal with pages, their components, and the output routine.
\li ``Commands for horizontal and vertical modes'', \chapterref{hvmodes},
have corresponding or identical
forms for both horizontal modes (paragraphs and hboxes) and vertical
modes (pages and vboxes).
These commands provide boxes, spaces, rules, leaders,
and alignments.
\li ``Commands for composing math formulas'', \chapterref{math},
provide capabilities for constructing math formulas.
\li ``Commands for general operations'', \chapterref{general},
provide 
\TeX's programming features and
everything else that doesn't fit into any of the other sections.
\endulist
You should think of these categories as being suggestive rather than
rigorous, because the commands don't really fit neatly into these
(or any other) categories.
 
Within each section, the descriptions of the commands are organized
by function.  When several commands are closely related, they are described as
a group; otherwise, each command has its own explanation.
The description of each command
includes one or more examples and the output
produced by each example when examples are appropriate (for
some commands they aren't).
When you are looking at a subsection containing functionally related
commands, be sure to check the end of a subsection for a ``see also'' 
item that refers you to related commands that are described elsewhere.
 
Some commands are closely related to certain concepts.
For instance, the |\halign| and |\valign| commands are related to
``alignment'', the |\def|
command is related to ``macro'',
and the |\hbox| and |\vbox| commands are related to ``box''.
In these cases we've usually given a bare-bones des\-crip\-tion of the
commands themselves and explained  the underlying ideas
in the concept. 

The examples associated with the commands have been typeset with
^|\parindent|, the paragraph indentation, set to zero so that
paragraphs are normally unindented.
This convention makes the examples easier to read.
In those examples where the paragraph indentation is essential,
we've set it explicitly to a nonzero value.

The pointing hand in front of a command or a group of commands indicates
that we judged this command or group of commands to be particularly useful
and easy to understand.

Many commands expect ^{arguments} of one kind or another
(\xref{arg1}).  The arguments of
a command give \TeX\ additional information that it needs in order to
carry out the command.  Each argument is indicated by an italicized
term in angle brackets that indicates what kind of argument it~is:
 
\display{%
\halign{\<#>\quad&#\hfil\cr
argument&a single token or some text enclosed in braces\cr
charcode&a character code, i.e., an integer between $0$ and $255$\cr
dimen&a dimension, i.e., a length\cr
glue&glue (with optional stretch and shrink)\cr
number&an optionally signed integer (whole number)\cr
register&a register number between $0$ and $255$\cr
}}
^^{\<dimen>}
^^{\<argument>}
^^{\<charcode>}
^^{\<glue>}
^^{\<number>}
^^{\<register>}

\noindent
All of these terms are explained in more detail in \chapterref{concepts}.
In addition, we sometimes use terms such as \<token list> that are either
self-explanatory or explained in the description of the command.
Some commands have special formats that require either braces or 
particular words.
These are set in the same bold font that we use
for the command headings.

Some commands are parameters (\xref{introparms}) or table entries. 
^^{parameters//as commands}
This is indicated in the command's listing.
You can either use a parameter as an argument or assign a value to it.
The same holds for table entries.
We use the term ``parameter'' to refer to entities such as |\pageno|
that are actually registers but behave just like parameters.
^^{registers//parameters as}


\endchapter
\byebye
