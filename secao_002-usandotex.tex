% Isto é parte do livro TeX for the Impatient.
% Copyright (C) 2003 Paul W. Abrahams, Kathryn A. Hargreaves, Karl Berry.
% Veja o arquivo fdl.tex para condições de reprodução.

\input macros.tex
\chapter{Usando \TeX}

\chapterdef{usandotex}

% Evite reclamações de caixa insuficiente acerca do parágrafo vazio que 
% precede o cabeçalho da primeira seção.
% 
\def\par{{\parfillskip = 0pt plus 1fil\endgraf}\let\par=\endgraf}
\vglue-\abovesectionskip % nós já pulamos o suficiente
\vskip0pt % Faz \combineskips funcionar.

\section Transformando entrada em tinta

\subsection Programas e arquivos de que voc\^e precisa

Para produzir um documento \TeX, voc\^e precisar\'a executar o programa 
\TeX\ e v\'arios programas relacionados.  Voc\^e tamb\'em precisar\'a de 
arquivos de suporte para \TeX\ e possivelmente para esses outros 
programas.  Neste livro, podemos falar sobre o \TeX, mas n\~ao podemos 
falar sobre os outros programas e os arquivos de suporte, exceto em 
termos muito gerais, porque eles dependem do seu ambiente \TeX\ local.  
As pessoas que lhe fornecem \TeX\ devem poder fornecer a voc\^e o que 
chamamos de \emph{informa\c c\~oes locais}. \pix^^{informa\c c\~ao local}
As informa\c c\~oes locais informam como iniciar \TeX, como usar os 
programas relacionados e como obter acesso aos arquivos de suporte.

A entrada para \TeX\ consiste em um arquivo de texto comum que voc\^e 
pode preparar com um ^{editor de texto}.  Um arquivo de entrada do \TeX, 
ao contr\'ario de um arquivo de entrada para um processador de texto 
t\'ipico, normalmente n\~ao cont\'em quaisquer ^{caracteres de controle} 
invis\'iveis.  Tudo o que o \TeX\ v\^e \'e vis\'ivel para voc\^e 
tamb\'em se voc\^e olhar para uma listagem do arquivo.

Seu arquivo de entrada pode acabar sendo pouco mais que um esqueleto que 
chama outros arquivos de entrada.  Os usu\'arios do \TeX\ costumam 
organizar grandes documentos, como livros, dessa maneira.  Voc\^e pode 
usar o comando ^|\input| (\xref\input) para incorporar um arquivo de 
entrada em outro.  Em particular, voc\^e pode usar |\input| para 
incorporar arquivos contendo \emph{defini\c c\~oes de macro}---%
^^{macros//em arquivos auxiliares}
defini\c c\~oes auxiliares que melhoram as capacidades do \TeX.  Se 
alguns arquivos de macro estiverem dispon\'iveis em sua instala\c c\~ao 
do \TeX\, a informa\c c\~ao local acerca do \TeX\ dever\'a lhe dizer 
como acessar os arquivos de macro e o que eles podem fazer por voc\^e.  
A forma padr\~ao de \TeX, a descrita neste livro, incorpora uma 
cole\c c\~ao de macros e outras defini\c c\~oes conhecidas como 
^{\plainTeX} (\xref{\plainTeX}).

Quando \TeX\ processa seu documento, ele produz um arquivo chamado 
^{\dvifile}.  A abreviatura ``|dvi|'' significa ``device independent'' 
(independente de dispositivo).  A abrevia\c c\~ao foi escolhida porque 
as informa\c c\~oes no \dvifile\ s\~ao independentes do dispositivo 
usado para imprimir ou exibir o documento.

Para imprimir seu documento ou visualiz\'a-lo com um 
\emph{visualizador}, ^^{visualizador}, voc\^e precisa processar o 
^{\dvifile} com um programa \emph{controlador de dispositivo\/} 
^^{controladores de dispositivo}(Um visualizador \'e um programa que 
permite que voc\^e veja em uma tela alguma aproxima\c c\~ao de como 
ser\'a a sa\'ida da tipografia).  Diferentes dispositivos de sa\'ida 
geralmente exigem controladores de dispositivos diferentes.  Depois de 
executar o controlador de dispositivo, voc\^e tamb\'em pode precisar 
transferir a sa\'ida do controlador de dispositivo para a impressora ou 
outro dispositivo de sa\'ida.
^^{impressoras} ^^{dispositivos de sa\'ida}  A informa\c c\~ao local 
acerca do \TeX\ devem informar como obter o controlador correto de 
dispositivo e us\'a-lo.

Como o \TeX\ n\~ao possui conhecimento embutido de fontes espec\'ificas, 
ele usa \emph{arquivos de fonte} ^^{arquivos de fonte} para obter 
informa\c c\~ao acerca das fontes usadas em seu documento.  Os arquivos 
de fonte tamb\'em devem fazer parte do seu ambiente local do \TeX.  Cada 
fonte normalmente requer dois arquivos: um contendo as dimens\~oes dos 
caracteres na fonte (o \emph{arquivo de métricas}) 
^^{arquivo de métricas}e um contendo as formas dos caracteres 
(o \emph{arquivo de forma}).^^{arquivo de forma}  Vers\~oes ampliadas de 
uma fonte compartilham o arquivo de m\'etricas, mas possuem arquivos de 
formas diferentes. ^^{magnifica\c c\~ao} Os arquivos de m\'etricas as 
vezes s\~ao chamados de ^{\tfmfile}s, e as diferentes variedades de 
arquivos de formas as vezes s\~ao chamadas de ^{\pkfile}s, ^{\pxlfile}s, 
e ^{\gffile}s.  Esses nomes correspondem aos nomes dos arquivos que o 
\TeX\ e seus programas associados usam.  Por exemplo, |cmr10.tfm| \'e o 
arquivo de m\'etricas para a fonte |cmr10| (Computer Modern Roman de 10 
pontos).

O pr\'oprio \TeX\ usa apenas o arquivo de m\'etricas, j\'a que ele n\~ao 
se importa com a apar\^encia dos caracteres, mas apenas com quanto 
espa\c co eles ocupam.  O controlador de dispositivo normalmente usa o 
arquivo de forma, j\'a que \'e respons\'avel por criar a imagem impressa 
de cada caractere de tipografia.  Alguns controladores de dispositivo 
tamb\'em precisam usar o arquivo de m\'etricas.  Alguns controladores de 
dispositivo podem utilizar fontes residentes em uma impressora e n\~ao 
precisam de arquivos de formas para essas fontes.
\secondprinting{\vfill\eject}


\subsection Executando o {\TeX}

\bix^^{executando \TeX}
Voc\^e pode executar o \TeX\ em um arquivo de entrada |screed.tex| 
^^{arquivos de entrada}
digitando algo como `|run tex|' ou apenas `|tex|' (verifique sua 
informa\c c\~ao local).  \TeX\ responder\'a algo como:
% 23/04/1990 é o 426° aniversário de Shakespeare, e o 26° de Karl.
\csdisplay
This is TeX, Version 3.0 (preloaded format=plain 90.4.23)
**
|
O ``formato pr\'e carregado'' aqui se refere a uma forma pr\'e 
compilada das macros do ^{\plainTeX} que vem com o \TeX.  Voc\^e pode 
agora digitar `|screed|' para que o \TeX\ processe seu arquivo.  Quando 
estiver pronto, voc\^e ver\'a algo como:
\csdisplay
(screed.tex [1] [2] [3] )
Output written on screed.dvi (3 pages, 400 bytes).
Transcript written on screed.log.
|
exibido no seu terminal ou impresso no registro da sua execu\c c\~a se 
n\~ao estiver trabalhando em um terminal.  A maior parte dessa sa\'ida 
\'e auto explicativa.  Os n\'umeros entre par\ênteses s\~ao n\'umeros de 
p\'aginas que o \TeX\ mostra quando envia cada p\'agina do seu documento 
ao \dvifile.  \TeX\ normalmente assumir\'a uma extens\~ao `|.tex|' para 
um nome de arquivo de entrada se o nome do arquivo de entrada que voc\^e 
deu n\~ao tiver uma extens\~ao.  Para algumas formas do \TeX\ voc\^e 
pode ser capaz de invocar \TeX\ diretamente para um arquivo de entrada 
digitando:
\csdisplay
tex screed
|
ou algo como isso.

Em vez de fornecer sua entrada \TeX\ a partir de um arquivo, voc\^e pode 
digit\'a-la diretamente no seu terminal.  Para fazer isso, digite 
`^|\relax|' em vez de `|screed|' no aviso `|**|'.  \TeX\ ir\'a agora 
avisar-lhe com um `|*|' para cada linha de entrada e interpretar cada 
linha de entrada conforme a vir.  Para finalizar a entrada, digite um 
comando como `|\bye|' que diz ao \TeX\ que voc\^e est\'a pronto.  A 
entrada direta as vezes \'e uma maneira pr\'atica de experimentar o 
\TeX.

Quando seu arquivo de entrada cont\'em outros arquivos de entrada 
incorporados, as informa\c c\~oes exibidas indicam quando o \TeX\ inicia 
e termina o processamento de cada arquivo incorporado.
^^{arquivos de entrada//incorporado}
\xrdef{arquivosentrada}
O \TeX\ exibe um par\^entese esquerdo e o nome do arquivo quando ele 
come\c ca a trabalhar em um arquivo, e exibe o par\^entese direito 
correspondente quando o trabalho est\'a feito com o arquivo.  Se voc\^e 
receber quaisquer ^{mensagens de erro} na sa\'ida exibida, voc\^e 
poder\'a combin\'a-las com um arquivo procurando pelo par\^entese 
esquerdo n\~ao fechado mais recente.

Para uma explica\c c\~ao mais completa de como executar o \TeX, veja 
\knuth{Chapter~6} e sua ^{informa\c c\~ao local}.
\eix^^{executando o \TeX}


\section Preparando um arquivo de entrada

Nesta se\c c\~ao explicamos algumas das conven\c c\~oes que voc\^e deve 
seguir ao preparar a entrada para \TeX\null.  Algumas das 
informa\c c\~oes dadas aqui tamb\'em aparecem nos exemplos em 
\chapterref{exemplos} deste livro.
^^{entrada, preparando}

\subsection Comandos e sequ\^encias de controle

\bix^^{comandos}
\bix^^{sequ\^encias de controle}
A entrada para \TeX\ consiste de uma sequ\^encia de comandos que 
informam ao \TeX\ como tipografar seu documento.  A maioria dos 
caracteres age como comandos de um tipo particularmente simples: 
``tipografe-me''.  A letra `|a|', por exemplo, \'e um comando para 
tipografar um 'a'.  Mas h\'a outro tipo de comando---uma 
\emph{sequ\^encia de controle}---que d\'a ao \TeX\ uma instru\c c\~ao 
mais elaborada.  Uma sequ\^encia de controle normalmente come\c ca com 
uma barra invertida (|\|), embora voc\^e possa alterar essa 
conven\c c\~ao se precisar.
\xrdef{@backslash}
Por exemplo, a entrada:

\csdisplay
Ela cravou um punhal (\dag) no cora\c c\~ao do vil\~ao.
|
cont\'em a sequ\^encia de controle |\dag|; essa sequ\~encia produz a 
sa\'ida de tipografia:
\display{%
Ela cravou um punhal (\dag) no cora\c c\~ao do vil\~ao.
}
\noindent Tudo neste exemplo, exceto o |\dag| e os espa\c cos funcionam 
como um comando ``tipografe-me''.  Vamos explicar mais sobre espa\c cos 
em \xrefpg{spaces}.

Existem dois tipos de sequ\^encias de controle: 
\emph{palavras de controle} 
^^{palavras de controle}
e \emph{s\'imbolos de controle}:
^^{s\'imbolos de controle}
\ulist\compact
\li Uma palavra de controle consiste de uma barra invertida seguida por 
uma ou mais letras, por exemplo, `|\dag|'.  O primeiro caractere que não 
for uma letra marca o fim da palavra de controle.
\li Um s\'imbolo de controle consiste em uma barra invertida seguida por 
um caractere que n\~ao \'e uma letra, por exemplo, `|\$|'.  O caractere 
pode ser um espa\c co ou at\'e mesmo o fim de uma linha (que \'e um 
caractere perfeitamente leg\'itimo).
\endulist
\noindent
Uma palavra de controle (mas n\~ao um s\'imbolo de controle) absorve 
qualquer espa\c co ou fim de linha que o segue.  
^^{sequ\^encias de controle//absorvendo espa\c cos}
Se voc\^e n\~ao quiser perder um espa\c co ap\'os uma palavra de 
controle, siga a sequ\^encia de controle com um ^{espa\c co de controle} 
(|\!visiblespace|) ou com `|{}|'.  Assim ou:
\csdisplay
As maravilhas do \TeX\!visiblespace!.nunca cessar\~ao!!
|
ou:\hfil\ 
\csdisplay
As maravilhas do \TeX{} nunca cessar\~ao!!
|
produz:
\display{%
As maravilhas do \TeX{} nunca cessar\~ao!
}
\noindent em vez de:
\display{%
As maravilhas do \TeX nunca cessar\~ao!
}
\noindent
que \'e o que voc\^e ganharia se deixasse de fora o `|\|\visiblespace' 
ou o `|{}|'.

N\~ao execute uma palavra de controle junto com o texto que a segue---o 
\TeX\ n\~ao vai saber onde a palavra de controle termina.  Por exemplo, 
a sequ\^encia de controle |\c| coloca um acento cedilha no caractere que 
o seguir.  A palavra francesa {\it gar\c con\/} deve ser escrita como 
`|gar\c!visiblespace!.con|', e n\~ao `|gar\ccon|'; se voc\^e escrever a 
\'ultima, o \TeX\ reclamar\'a acerca de uma sequ\^encia de controle 
indefinida |\ccon|.

Um s\'imbolo de controle, por outro lado, n\~ao absorve nada que o siga.  
Assim, voc\^e deve digitar `\$13.56' como `|\$13.56|', n\~ao 
`|\$!vs13.56|'; a \'ultima forma produzir\'a `\hbox{\$ 13.56}'.  No 
entanto, esses comandos de acentua\c c\~ao que s\~ao nomeados pelos 
s\'imbolos de controle s\~ao definidos de tal forma que produzem o 
efeito de absorver um espa\c co seguinte.  Assim, voc\^e pode digitar a 
palavra francesa {\it d\'eshabiller\/} ou como `|d\'eshabiller|' ou como 
`|d\'!visiblespace!.eshabiller|'.

Cada sequ\^encia de controle tamb\'em \'e um comando, mas n\~ao o 
contr\'ario.  
^^{comandos//versus sequ\^encias de controle}
^^{sequ\^encias de controle//versus comandos}
Por exemplo, a letra `|N|' \'e um comando, mas n\~ao \'e uma sequ\^encia 
de controle.  Neste livro, usualmente usamos ``comando'' em vez de 
``sequ\^encia de controle'' quando qualquer termo serviria.  N\'os 
usamos a ``sequ\^encia de controle'' quando queremos enfatizar aspectos 
da sintaxe do \TeX\ que n\~ao se aplicam a comandos em geral.

\eix^^{sequ\^encias de controle}
\eix^^{comandos}


\subsection Argumentos

\xrdef{arg1}
Alguns comandos precisam ser seguidos por um ou mais \emph{argumentos} 
^^{arguments} que ajudem a determinar o que o comando faz.  Por exemplo, 
o comando |\vskip|, que diz ao \TeX\ para pular para baixo (ou para 
cima) a p\'agina, espera um argumento especificando quanto espa\c co 
pular.  Para pular para baixo duas polegadas, voc\^e deve digitar 
`|\vskip 2in|', onde |2in| \'e o argumento de |\vskip|.

Comandos diferentes esperam diferentes tipos de argumentos.  Muitos 
comandos esperam dimens\~oes, tal como o |2in| no exemplo acima.  Alguns 
comandos, particularmente aqueles definidos por macros, esperam 
argumentos que sejam um caractere ou algum texto entre chaves.  No 
entanto, outros exigem que seus argumentos sejam colocados entre chaves, 
isto \'e, eles n\~ao aceitam argumentos de caracteres \'unicos.  A 
descri\c c\~ao de cada comando neste livro informa quais tipos de 
argumentos, se houver algum, o comando espera.  Em alguns casos, as 
chaves obrigat\'orias definem um grupo (consulte \xref{grupo de chave}).

\secondprinting{\vfill\eject}


\subsection Par\^ametros

\xrdef{introparms}
Alguns comandos s\~ao par\^ametros (\xref{par\^ametro}).
^^{par\^ametros//como comandos}
Voc\^e pode usar um par\^ametro de duas maneiras:
\olist
\li Voc\^e pode usar o valor de um par\^ametro como um argumento para 
outro comando.  Por exemplo, o comando \hbox{|\vskip\parskip|} provoca 
um salto vertical segundo o valor do par\^ametro de cola |\parskip| 
(pular par\'agrafo).
\li Voc\^e pode alterar o valor do par\^ametro atribuindo algo a ele.  
Por exemplo, a atribui\c c\~ao \hbox{|\hbadness=200|} faz com que o 
valor do par\^ametro num\'erico de |\hbadness| seja $200$.
\endolist
\noindent
N\'os tamb\'em usamos o termo ``par\^ametro'' para nos referirmos a 
entidades como |\pageno| que, na verdade, s\~ao registros, mas se 
comportam como par\^ametros.
^^{registros//como par\^ametros}

Alguns comandos s\~ao nomes de tabelas.  Esses comandos s\~ao usados 
como par\^ametros, exceto pelo fato de exigirem um argumento adicional 
que define uma entrada espec\'ifica na tabela.  Por exemplo, |\catcode| 
nomeia uma tabela de c\'odigos de categoria 
(\xref{c\'odigo de categoria}).  Assim, o comando \hbox{|\catcode`~=13|} 
define o c\'odigo de categoria do caractere `|~|' para $13$.


\subsection Espa\c cos

\xrdef{espacos}
\bix^^{espacos}
Voc\^e pode usar livremente espa\c cos extras em sua entrada.  Sob quase 
todas as circunst\^ancias \TeX\ trata v\'arios espa\c cos seguidos como 
sendo equivalentes a um \'unico espa\c co.   Por exemplo, n\~ao importa 
se voc\^e coloca um espa\c co ou dois espa\c cos ap\'os um ^{ponto} na 
sua entrada.  Seja l\'a o que voc\^e fa\c ca, o \TeX\ realiza suas 
manobras de final de frase e deixa a quantidade de espa\c co apropriada 
(na maioria dos casos) ap\'os o ponto.  O \TeX\ tamb\'em trata o final 
de uma linha de entrada como o equivalente a espa\c co.  Assim, voc\^e 
pode finalizar suas linhas de entrada conforme for conveniente---\TeX\ 
transforma linhas de entrada em par\'agrafos da mesma maneira, n\~ao 
importa onde as quebras de linha est\~ao em sua entrada.

Uma linha de branco em sua entrada marca do final de um par\'agrafo.  
^^{paragrafos//finalizando}
V\'arias linhas em branco s\~ao equivalentes a uma \'unica.

\TeX\ ignora espa\c cos de entrada dentro de f\'ormulas matem\'aticas 
(veja abaixo).  Assim, voc\^e pode incluir ou omitir espa\c cos em 
qualquer lugar dentro de uma f\'ormula matem\'atica---\TeX\ n\~ao se 
importa.  Mesmo dentro de uma f\'ormula matem\'atica, no entanto, voc\^e 
n\~ao deve executar uma palavra de controle junto com a seguinte letra.

Se voc\^e est\'a definindo suas pr\'oprias macros, voc\^e precisa ser 
particularmente cuidadoso sobre onde voc\^e coloca fins de linha em suas 
defini\c c\~oes.  \'E muito f\'acil definir uma macro que produz um 
^{espa\c co indesejado} al\'em do que mais deveria produzir.  N\'os 
discutimos esse problema em outro lugar, j\'a que \'e um pouco 
t\'ecnico; veja \xrefpg{espacoindesejado}.

Um espa\c co ou seu equivalente entre duas palavras na sua entrada n\~ao 
se transforma simplesmente em um caractere de espa\c co na sua sa\'ida.  
Alguns desses espa\c cos de entrada se transformam em extremidades de 
linhas na sa\'ida, uma vez que as linhas de entrada geralmente n\~ao 
correspondem \`as linhas de sa\'ida.  Os outros se transformam em 
espa\c cos de largura vari\'avel chamados ``cola'' (\xref{cola}), os 
quais tem um tamanho natural (o tamanho que ``quer ter''), mas pode 
esticar ou encolher.  Quando \TeX\ est\'a tipografando um par\'agrafo 
que deve ter uma margem direita (o caso usual), ele ajusta as larguras 
da cola em cada linha para fazer com que as linhas terminem na margem.  
(A \'ultima linha de um par\'agrafo \'e uma exce\c c\~ao, pois 
normalmente n\~ao \'e necess\'ario terminar na margem direita).

Voc\^e pode evitar que um espa\c co de entrada se transforme em um fim 
de linha usando um ^{til} (^|~|).  Por exemplo, voc\^e n\~ao desejaria 
que \TeX\ colocasse uma quebra de linha entre o `Fig.' e `8' de 
`Fig.~8'.  Digitando `|Fig.~8|' voc\^e pode evitar tal quebra de linha.
\eix^^{espacos}
\needspace{2in}


\subsection Coment\'arios

\xrdef{comentarios}
\pix\bix^^{comentarios}
Voc\^e pode incluir coment\'arios em sua entrada do \TeX\.  Quando o 
\TeX\ v\^e um coment\'ario, apenas passa por cima dele, de forma que o 
que est\'a em um coment\'ario n\~ao afeta o seu documento tipografado de 
forma alguma.  Os coment\'arios s\~ao \'uteis para fornecer 
informa\c c\~oes extras sobre o que est\'a no seu arquivo de entrada.  
Por exemplo:
\csdisplay
% ========= Início da Seção `Ouriço' =========
|

{\indexchar % }%
Um coment\'ario come\c ca com um sinal de porcentagem (|%|) e se estende 
at\'e o final da linha de entrada.  \TeX\ ignora n\~ao apenas o 
coment\'ario, mas tamb\'em o fim de linha, de forma que os coment\'arios 
tem outro uso muito importante: conectar duas linhas para que o fim de 
linha ^^{quebra de linha//deletando} entre elas seja invis\'ivel para 
\TeX\ e n\~ao gere um espa\c co de sa\'ida ou um fim de linha.  Por 
exemplo, se voc\^e digitar:
\csdisplay
Um tolo com uma planilha%
ainda é um idiota.
|
voc\^e obter\'a:
\display{
Um tolo com uma planilha%
ainda é um idiota.
}
\eix^^{comentarios}


\subsection Pontua\c c\~ao

\null
\xrdef{espacamentodeponto}
\TeX\ normalmente adiciona algum espa\c co extra depois do que acha que 
\'e uma marca de ^{pontua\c c\~ao} no final de uma frase, a saber, 
`^|.|', `^|?|', ou `|!!|' 
\indexchar !
^^{ponto} ^^{ponto de interroga\c c\~ao} ^^{ponto de exclama\c c\~ao}
seguido por um espa\c co de entrada.  Entretanto, \TeX\ n\~ao adiciona o 
espa\c co extra se a marca de pontua\c c\~ao seguir uma letra 
mai\'uscula, porque assume que a letra mai\'uscula \'e uma inicial no 
nome de algu\'em.  Voc\^e pode for\c car o espa\c co extra onde n\~ao 
ocorreria digitando algo como:
\csdisplay
Um computador da IBM\null?
|
O |\null| n\~ao produz sa\'ida alguma, mas impede que \TeX\ associe a 
mai\'uscula `M' ao ponto de interroga\c c\~ao.  Por outro lado, voc\^e 
pode cancelar o espa\c co extra onde ele n\~ao deve estar digitando um 
espa\c co de controle ap\'os a marca de pontua\c c\~ao, por exemplo:
\csdisplay
Proc.\!visiblespace!.Royal Acad.\!visiblespace!.of Twits
|
de forma que voc\^e obter\'a:
\display{Proc.\ Royal Acad.\ of Twits}
\noindent
em vez de:
\display{Proc. Royal Acad. of Twits}

Algumas pessoas preferem n\~ao deixar mais espa\c co ap\'os a 
pontua\c c\~ao no final de uma frase.  Voc\~e pode obter esse efeito com 
o comando ^|\frenchspacing| (\xref\frenchspacing).  |\frenchspacing| \'e 
frequentemente recomendado para ^{bibliografias}.

Para ^{aspas} simples, voc\^e deve usar as aspas simples esquerda e 
direita (|`| e |'|) no seu teclado.  Para aspas duplas \`a esquerda e 
\`a direita, use duas aspas simples \`a esquerda ou duas aspas simples 
\`a direita (|``| ou |''|) em vez das aspas duplas (|"|) no teclado.  A 
aspa dupla do teclado de fato dar\'a a voc\^e as corretas aspas duplas 
em muitas fontes, mas as duas aspas simples \`a direita s\~ao o estilo 
\TeX\ preferido.  Por exemplo:

\vbox{%
\csdisplay
Não há `q' nesta frase.
``Fale, criança,'', disse o Unicórnio.
Ela disse, ``\thinspace`Chega!!', ele disse.''
|
}%
Essas tr\^es linhas produzem:
\display{\par\restoreplainTeX
N\~ao h\'a `q' nesta frase.
\par ``Fale, crian\c ca,'' disse o Unic\'ornio.
\par Ela disse, ``\thinspace`Chega!', ele disse.''
}
\noindent
O |\thinspace| na terceira linha de entrada impede que as aspas simples 
cheguem muito perto das aspas duplas.  Sem ele, voc\^e veria tr\^es 
aspas quase igualmente espa\c cadas em uma linha.

\TeX\ tem tr\^es tipos de ^{tra\c cos}:
\ulist\compact
\li Curtos (hifens) como este ( - ).  Voc\^e os obt\'em digitando~`^|-|'.
\li M\'edios (tra\c cos-n) como este ( -- ).  Voc\^e os obt\'em digitando~`^|--|'.
\li Longos (tra\c cos-m) como este ( --- ).  Voc\^e os obt\'em digitando~`^|---|'.
\endulist
\noindent
Normalmente voc\^e usaria hifens para indicar palavras compostas como 
``will-o'-the-wisp''; tra\c cos-n para indicar intervalos de p\'aginas 
como ``p\'aginas~81--87''; e tra\c cos-m para indicar uma quebra na 
continuidade---como essa.


\subsection Caracteres especiais

Certos caracteres tem um significado especial para \TeX, ent\~ao voc\^e 
n\~ao deve us\'a-los em texto comum.  Eles s\~ao:

\csdisplay
    $  #  &  %  _  ^  ~  {  }  \
|
^^|$//em texto comum|
^^|#//em texto comum|
^^|&//em texto comum|
^^|_//em texto comum|
^^|^//em texto comum|
^^|~//em texto comum|
^^|%//em texto comum|
^^|{//em texto comum|
^^|}//em texto comum|
{\recat!ttidxref[\//em texto comum]]
\noindent
Para produzi-los em seu documento tipografado, voc\^e precisa usar 
circunlocu\c c\~oes.  Para os cinco primeiros, voc\^e deve digitar:
^^|\$|
^^|\#|
^^|\&|
^^|\%|
^^|\_|
\csdisplay
    \$  \#  \&  \%  \_
|

\noindent
Para os outros, voc\^e precisa de algo mais elaborado:

\csdisplay
   \^{!visiblespace}   \~{!visiblespace}   $\{$   $\}$   $\backslash$
|


\subsection Grupos

\bix^^{grupos}
Um \emph{group} consiste de material contido nas chaves esquerda e 
direita correspondentes (|{| e |}|).  
^^|{//iniciando um grupo|
^^|}//finalizando um grupo|
Colocando um comando dentro de um grupo, voc\^e pode limitar seus 
efeitos ao material dentro do grupo.  Por exemplo, o comando |\bf| diz 
ao \TeX\ para definir algo no tipo {\bf negrito}.  Se voc\^e fosse 
colocar |\bf| em sua entrada e n\~ao fazer nada para neutralizar isso, 
tudo em seu documento seguindo o |\bf| seria colocado em negrito.  
Incluindo |\bf| em um grupo, voc\^e limita seu efeito ao grupo.  Por 
exemplo, se voc\^e digitar:
\csdisplay
N\'os temos {\bf poucas palavras em negrito} nesta frase.
|
\noindent
voc\^e obter\'a:
\display{N\'os temos {\bf poucas palavras em negrito} nesta frase.}

\noindent
Voc\^e tamb\'em pode usar um grupo para limitar o efeito de uma 
atribui\c c\~ao a um dos par\^ametros do \TeX.  Esses par\^ametros 
cont\'em valores que afetam como o \TeX\ tipografa seu documento.  Por 
exemplo, o valor do par\^ametro |\parindent| especifica o recuo no 
in\'icio de um par\'agrafo.  A atribui\c c\~ao |\parindent = 15pt| 
define o recuo para $15$ pontos da impressora.  Ao colocar essa 
atribui\c c\~ao no in\'icio de um grupo contendo alguns par\'agrafos, 
voc\^e pode alterar o recuo apenas desses par\'agrafos.  Se voc\^e n\~ao 
colocar a atribui\c c\~ao em um grupo, o recuo alterado ser\'a aplicado 
ao restante do documento (ou at\'e a pr\'oxima atribui\c c\~ao a 
|\parindent|, se houver uma posterior).

\xrdef{grupodechave}
Nem todos os pares de chaves indicam um grupo.  Em particular, as chaves 
associadas a um argumento para o qual as chaves \emph{n\~ao} s\~ao 
necess\'arias n\~ao indicam um grupo---elas apenas servem para delimitar 
o argumento.  Desses comandos que requerem chaves para seus argumentos, 
alguns tratam as chaves como definindo um grupo e os outros interpretam 
o argumento de alguma maneira especial que depende do comando.
\footnote {Mais precisamente, para comandos primitivos, as chaves ou 
definem um grupo ou eles envolvem tokens que n\~ao s\~ao processados no 
``est\^omago'' do \TeX.  Para |\halign| e |\valign| o grupo tem um 
efeito trivial, porque tudo dentro das chaves ou n\~ao chega ao 
``est\^omago'' (porque est\'a no modelo) ou \'e envolvido em outro grupo 
interno.
^^|\halign//agrupamento para|
^^|\valign//agrupamento para|
}
\eix^^{grupos}


\subsection F\'ormulas matem\'aticas

\bix^^{matem\'atica}
\xrdef{f\órmulamatem\'atica}
Uma f\'ormula matem\'atica pode aparecer no texto 
(\emph{texto matem\'atico}) ^^{texto matem\'atico} ou definir uma linha 
pr\'opria com espa\c co vertical extra ao redor dela 
(\emph{exibir matem\'atica}).  
^^{exibir matem\'atica}
Você envolve uma f\'ormula de texto entre sinais de cifr\~ao único (|$|) 
e uma f\'ormula exibida entre dois cifr\~oes (|$$|).
\ttidxref{$}\ttidxref{$$}
Por exemplo:

\csdisplay
Se $a<b$, ent\~ao a rela\c c\~ao $$e^a < e^b$$ \'e v\'alida.
|
\noindent
Essa entrada produz:
\display{\centereddisplays Se $a<b$, ent\~ao a rela\c c\~ao 
$$e^a < e^b$$ \'e v\'alida.}
\smallskip
\noindent
\chapterref{matem\'atica} descreve os comandos que s\~ao \'uteis em 
f\'ormulas matem\'aticas.
\eix^^{matem\'atica}


\section Como \TeX\ funciona

Para usar \TeX\ efetivamente, ajuda ter uma id\'eia de como o \TeX\ 
opera sua atividade de transmuta\c c\~ao de entrada em sa\'ida.  Voc\^e 
pode imaginar o \TeX\ como uma esp\'ecie de organismo com ``olhos'', 
``boca'', ``garganta'', ``est\^omago'' e ``intestinos''.  Cada parte do 
organismo transforma sua entrada de alguma maneira e passa a entrada 
transformada para o pr\'oximo est\'agio.

Os ^{olhos} transformam um arquivo de entrada em uma sequ\^encia de 
caracteres.  A ^{boca} transforma a sequ\^encia de caracteres em uma 
sequ\^encia de \emph{tokens}, ^^{tokens} onde cada token ou \'e um 
\'unico caractere ou uma sequ\^encia de controle.  
^^{sequ\^encias de controle//como tokens}
A garganta expande os tokens em uma sequ\^encia de 
\emph{comandos primitivos}, os quais tamb\'em s\~ao tokens.  
^^{expandindo tokens}
O ^{est\^omago} realiza as opera\c c\~oes especificadas pelos comandos 
primitivos, produzindo uma sequ\^encia de p\'aginas.  Finalmente, os 
^{intestinos} transformam cada p\'agina na forma requerida pelo 
\dvifile\ e a envia.  ^^{\dvifile//criado pelos intestinos do \TeX}
Essas a\c c\~oes est\~ao descritas com mais detalhes em 
\chapterref{conceitos} sob \conceptcit{\anatomy}.
^^{\anatomy}

A tipografia real ocorre no ``est\^omago''.  Os comandos instruem \TeX\ 
para tipografar tal e tal caractere em tal e tal fonte; para inserir um 
espa\c co entre palavras; para finalizar um par\'agrafo; e assim por 
diante.  Come\c cando com caracteres de tipografia individuais e outros 
elementos tipogr\'aficos simples, \TeX\ constr\'oi uma p\'agina 
^^{p\'aginas} como um ninho de ^{caixas} dentro de caixas dentro de 
caixas \seeconcept{caixa}.  Cada caractere de tipografia ocupa uma 
caixa, assim como uma p\'agina inteira.  Uma caixa pode conter n\~ao 
apenas caixas menores, mas tamb\'em \emph{cola} ^^{cola} (\xref{cola}) e 
algumas outras coisas.  A cola produz espa\c co entre as caixas menores.  
Uma propriedade importante da cola \'e que ela pode esticar e encolher; 
assim \TeX\ pode fazer uma caixa maior ou menor esticando ou encolhendo 
a cola dentro dela.

Grosso modo, uma linha \'e uma caixa contendo uma sequ\^encia de caixas 
de caracteres e uma p\'agina \'e uma caixa contendo uma sequ\^encia de 
caixas de linha.  H\'a cola entre as palavras de uma linha e entre as 
linhas de uma p\'agina.  \TeX\ estica ou encolhe a cola em cada linha, 
de modo a fazer com que a margem direita da p\'agina fique uniforme e a 
cola em cada p\'agina, de modo a tornar iguais as margens inferiores de 
p\'aginas diferentes.  Outros tipos de elementos tipogr\'aficos tamb\'em 
podem aparecer em uma linha ou em uma p\'agina, mas n\~ao vamos entrar 
neles aqui.

Como parte do processo de montagem de p\'aginas, o \TeX\ precisa quebrar 
os par\'agrafos em linhas e linhas em p\'aginas.  O ``est\^omago'' v\^e 
primeiro um par\'agrafo como uma longa linha, de fato.  Ele insere 
\emph{quebras de linha} ^^{quebra de linha} para transformar o 
par\'agrafo em uma sequ\^encia de linhas do comprimento certo, 
realizando uma an\'alise bastante elaborada para escolher o conjunto de 
quebras que faz o par\'agrafo aparentar melhor 
\seeconcept{quebra de linha}.  O ``est\^omago'' realiza um processo 
semelhante, por\'em mais simples, para transformar uma sequ\^encia de 
linhas em uma p\'agina.  Essencialmente, o ``est\^omago'' acumula linhas 
at\'e que nenhuma linha mais possa ser encaixada na p\'agina.  Em 
seguida, ele escolhe um \'unico local para quebrar a p\'agina, colocando 
as linhas antes da quebra na p\'agina atual e salvando as linhas ap\'os 
o intervalo para a pr\'oxima p\'agina \seeconcept{quebra de p\'agina}. 
^^{quebras de p\'agina//inseridas pelo ``est\^omago'' do \TeX}

Quando \TeX\ est\'a montando uma entidade a partir de uma lista de itens 
(caixas, cola, etc.), ele est\'a em um dos seis \emph{modos} ^^{modos} 
(\xref{modo}).  O tipo de entidade que est\'a sendo montado define o 
modo em que ele se encontra.  Existem dois modos comuns: modo horizontal 
comum para montar par\'agrafos (antes de serem divididos em linhas) e 
o modo vertical comum para montar p\'aginas.  Existem dois modos 
restritos: modo horizontal restrito para anexar itens horizontalmente 
para formar uma caixa horizontal e o modo vertical interno para anexar 
itens verticalmente para formar uma caixa vertical (diferente de uma 
p\'agina).  Finalmente, h\'a dois modos matem\'aticos: o modo 
matem\'atico de texto para montar f\'ormulas matem\'aticas dentro de um 
par\'agrafo e o modo matem\'atico de exibi\c c\~ao para montar 
f\'ormulas matem\'aticas que s\~ao exibidas em linhas pr\'oprias 
(consulte ``F\'ormulas matem\'aticas'', \xref{formulasmatematicas}).


\section Novo \TeX\ versus antigo {\TeX}

\xrdef{novotex}
Em 1989, Knuth fez uma grande revis\~ao em \TeX\ para adapt\'a-lo aos 
conjuntos de caracteres necess\'arios para suportar a tipografia para 
outros idiomas al\'em do ingl\^es. \space ^^{idiomas estrangeiros} A 
revis\~ao incluiu alguns recursos extras que poderiam ser adicionados 
sem perturbar mais nada.  Este livro descreve ``^{\newTeX}''.  Se voc\^e 
ainda estiver usando uma vers\~ao mais antiga do \TeX\ (vers\~ao $2.991$ 
ou anterior), voc\^e vai querer saber quais recursos do {\newTeX} voc\^e 
n\~ao pode usar.  Os seguintes recursos n\~ao est\~ao dispon\'iveis nas 
vers\~oes mais antigas:
\ulist\compact
\li ^|\badness| (\xref\badness)
\li ^|\emergencystretch| (\xref\emergencystretch)
\li ^|\errorcontextlines| (\xref\errorcontextlines)
\li ^|\holdinginserts| (\xref\holdinginserts)
\li ^|\language|, ^|\setlanguage|, e |\new!-lan!-guage|
(\pp\xrefn\language, \xrefn{\@newlanguage}) ^^|\newlanguage|
\li ^|\lefthyphenmin| e ^|\righthyphenmin| (\xref\lefthyphenmin)
\li ^|\noboundary| (\xref\noboundary)
\li ^|\topglue| (\xref\topglue)
\li A nota\c c\~ao |^^|$xy$ para d\'igitos hexadecimais (\xref{hexchars})
\endulist
\noindent
Recomendamos que voc\^e obtenha um novo \TeX\ se puder.


\section Recursos

\xrdef{recursos}
V\'arios recursos est\~ao dispon\'iveis para ajud\'a-lo a usar o \TeX.
\texbook\ \'e a fonte definitiva de informa\c c\~ao sobre \TeX:

\smallskip
{\narrower\noindent
^{Knuth, Donald E.}, \texbook.  Reading, Mass.: Addison-Wesley, 1984.\par}
\smallskip
\noindent
Certifique-se de obter a d\'ecima s\'etima impress\~ao (janeiro de 1990) 
ou posterior; as impress\~oes anteriores n\~ao cobrem os recursos do 
novo \TeX.

^{\LaTeX} \'e uma cole\c c\~ao muito popular de comandos projetados para 
simplificar o uso de \TeX.  Ela est\'a descrita em:
\smallskip
{\narrower\noindent\frenchspacing\spaceskip = 3.33pt plus 2pt minus 1.2pt
^{Lamport, Leslie}, {\sl The \LaTeX\ Document Preparation System}.
Reading, Mass.: Addison-Wesley, 1986.\par}
\smallskip
\noindent
^{\AMSTeX} \'e a cole\c c\~ao de comandos adotada pela American 
Mathematical Society como um padr\~ao para submeter manuscritos 
matem\'aticos eletronicamente.  Ela est\'a descrita em:
\smallskip
{\narrower\noindent
^{Spivak, Michael~D.}, {\sl The Joy of \TeX}. Providence, R.I.:
American Mathematical Society, 1986.
\par}
\smallskip
\noindent
Voc\^e pode se afiliar ao ^{\TUG} (TUG), que publica um boletim chamado 
{\it ^{TUGBoat}}.  O TUG \'e uma excelente fonte n\~ao apenas para 
informa\c c\~oes sobre \TeX, mas tamb\'em para cole\c c\~oes de macros, 
incluindo \AMSTeX.  Seu endere\c co \'e:
\smallskip
{\obeylines
^{\TUG}
c/o American Mathematical Society
P.O. Box 9506
Providence,  RI  02940
U.S.A.
}
\smallskip
\noindent
Finalmente, voc\^e pode obter c\'opias das macros ^|eplain.tex| 
descritas em \chapterref{eplain}, bem como as macros usadas na 
tipografia deste livro.  Elas est\~ao dispon\'iveis via Internet, por 
\ftp\ an\^onimo, a partir dos seguintes hosts:
{\obeylines\display{\tt
labrea.stanford.edu [36.8.0.47]
ics.uci.edu [128.195.1.1]
june.cs.washington.edu [128.95.1.4]}}

A vers\~ao eletr\^onica inclui macros adicionais que formatam a entrada 
para o programa de computador ^{\BibTeX}, escrito por Oren Patashnik na 
Universidade de Stanford, ^^{Patashnik, Oren} e imprime a sa\'ida a 
partir desse programa.  Se voc\^e encontrar bugs nas macros, ou pensar 
em melhorias, voc\^e pode enviar um e-mail para Karl em 
{\tt karl@cs.umb.edu}.

As macros tamb\'em est\~ao dispon\'iveis por US \$10.00 em disquetes de 
$5\frac1/4$\inches\ ou $3\frac1/2$\inches\ formato PC a partir de:
\smallskip
{\obeylines
Paul Abrahams
214 River Road
Deerfield,  MA  01342
\vskip\tinyskipamount
Email: {\tt Abrahams\%Wayne-MTS@um.cc.umich.edu}
}
\smallskip
\noindent
Esses endere\c cos est\~ao corretos em junho de 1990; por favor, esteja 
ciente de que eles podem mudar depois disso, particularmente os 
endere\c cos eletr\^onicos.

\endchapter\byebye
