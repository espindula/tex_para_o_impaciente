% Isto é parte do livro TeX for the Impatient.
% Copyright (C) 2003 Paul W. Abrahams, Kathryn A. Hargreaves, Karl Berry.
% Veja o arquivo fdl.tex para condições de reprodução.

\input macros.tex
\chapter{Usando \TeX}

\chapterdef{usandotex}

% Evite reclamações de caixa insuficiente acerca do parágrafo vazio que 
% precede o cabeçalho da primeira seção.
% 
\def\par{{\parfillskip = 0pt plus 1fil\endgraf}\let\par=\endgraf}
\vglue-\abovesectionskip % nós já pulamos o suficiente
\vskip0pt % Faz \combineskips funcionar.

\section Transformando entrada em tinta

\subsection Programas e arquivos de que voc\^e precisa

Para produzir um documento \TeX, voc\^e precisar\'a executar o programa 
\TeX\ e v\'arios programas relacionados.  Voc\^e tamb\'em precisar\'a de 
arquivos de suporte para \TeX\ e possivelmente para esses outros 
programas.  Neste livro, podemos falar sobre o \TeX, mas n\~ao podemos 
falar sobre os outros programas e os arquivos de suporte, exceto em 
termos muito gerais, porque eles dependem do seu ambiente \TeX\ local.  
As pessoas que lhe fornecem \TeX\ devem poder fornecer a voc\^e o que 
chamamos de \emph{informa\c c\~oes locais}. \pix^^{informa\c c\~ao local}
As informa\c c\~oes locais informam como iniciar \TeX, como usar os 
programas relacionados e como obter acesso aos arquivos de suporte.

A entrada para \TeX\ consiste em um arquivo de texto comum que voc\^e 
pode preparar com um ^{editor de texto}.  Um arquivo de entrada do \TeX, 
ao contr\'ario de um arquivo de entrada para um processador de texto 
t\'ipico, normalmente n\~ao cont\'em quaisquer ^{caracteres de controle} 
invis\'iveis.  Tudo o que o \TeX\ v\^e \'e vis\'ivel para voc\^e 
tamb\'em se voc\^e olhar para uma listagem do arquivo.

Seu arquivo de entrada pode acabar sendo pouco mais que um esqueleto que 
chama outros arquivos de entrada.  Os usu\'arios do \TeX\ costumam 
organizar grandes documentos, como livros, dessa maneira.  Voc\^e pode 
usar o comando ^|\input| (\xref\input) para incorporar um arquivo de 
entrada em outro.  Em particular, voc\^e pode usar |\input| para 
incorporar arquivos contendo \emph{defini\c c\~oes de macro}---%
^^{macros//em arquivos auxiliares}
defini\c c\~oes auxiliares que melhoram as capacidades do \TeX.  Se 
alguns arquivos de macro estiverem dispon\'iveis em sua instala\c c\~ao 
do \TeX\, a informa\c c\~ao local acerca do \TeX\ dever\'a lhe dizer 
como acessar os arquivos de macro e o que eles podem fazer por voc\^e.  
A forma padr\~ao de \TeX, a descrita neste livro, incorpora uma 
cole\c c\~ao de macros e outras defini\c c\~oes conhecidas como 
^{\plainTeX} (\xref{\plainTeX}).

Quando \TeX\ processa seu documento, ele produz um arquivo chamado 
^{\dvifile}.  A abreviatura ``|dvi|'' significa ``device independent'' 
(independente de dispositivo).  A abrevia\c c\~ao foi escolhida porque 
as informa\c c\~oes no \dvifile\ s\~ao independentes do dispositivo 
usado para imprimir ou exibir o documento.

Para imprimir seu documento ou visualiz\'a-lo com um 
\emph{visualizador}, ^^{visualizador}, voc\^e precisa processar o 
^{\dvifile} com um programa \emph{controlador de dispositivo\/} 
^^{controladores de dispositivo}(Um visualizador \'e um programa que 
permite que voc\^e veja em uma tela alguma aproxima\c c\~ao de como 
ser\'a a sa\'ida da tipografia).  Diferentes dispositivos de sa\'ida 
geralmente exigem controladores de dispositivos diferentes.  Depois de 
executar o controlador de dispositivo, voc\^e tamb\'em pode precisar 
transferir a sa\'ida do controlador de dispositivo para a impressora ou 
outro dispositivo de sa\'ida.
^^{impressoras} ^^{dispositivos de sa\'ida}  A informa\c c\~ao local 
acerca do \TeX\ devem informar como obter o controlador correto de 
dispositivo e us\'a-lo.

Como o \TeX\ n\~ao possui conhecimento embutido de fontes espec\'ificas, 
ele usa \emph{arquivos de fonte} ^^{arquivos de fonte} para obter 
informa\c c\~ao acerca das fontes usadas em seu documento.  Os arquivos 
de fonte tamb\'em devem fazer parte do seu ambiente local do \TeX.  Cada 
fonte normalmente requer dois arquivos: um contendo as dimens\~oes dos 
caracteres na fonte (o \emph{arquivo de métricas}) 
^^{arquivo de métricas}e um contendo as formas dos caracteres 
(o \emph{arquivo de forma}).^^{arquivo de forma}  Vers\~oes ampliadas de 
uma fonte compartilham o arquivo de m\'etricas, mas possuem arquivos de 
formas diferentes. ^^{magnifica\c c\~ao} Os arquivos de m\'etricas as 
vezes s\~ao chamados de ^{\tfmfile}s, e as diferentes variedades de 
arquivos de formas as vezes s\~ao chamadas de ^{\pkfile}s, ^{\pxlfile}s, 
e ^{\gffile}s.  Esses nomes correspondem aos nomes dos arquivos que o 
\TeX\ e seus programas associados usam.  Por exemplo, |cmr10.tfm| \'e o 
arquivo de m\'etricas para a fonte |cmr10| (Computer Modern Roman de 10 
pontos).

O pr\'oprio \TeX\ usa apenas o arquivo de m\'etricas, j\'a que ele n\~ao 
se importa com a apar\^encia dos caracteres, mas apenas com quanto 
espa\c co eles ocupam.  O controlador de dispositivo normalmente usa o 
arquivo de forma, j\'a que \'e respons\'avel por criar a imagem impressa 
de cada caractere de tipografia.  Alguns controladores de dispositivo 
tamb\'em precisam usar o arquivo de m\'etricas.  Alguns controladores de 
dispositivo podem utilizar fontes residentes em uma impressora e n\~ao 
precisam de arquivos de formas para essas fontes.
\secondprinting{\vfill\eject}


\subsection Executando o {\TeX}

\bix^^{executando \TeX}
Voc\^e pode executar o \TeX\ em um arquivo de entrada |screed.tex| 
^^{arquivos de entrada}
digitando algo como `|run tex|' ou apenas `|tex|' (verifique sua 
informa\c c\~ao local).  \TeX\ responder\'a algo como:
% 23/04/1990 é o 426° aniversário de Shakespeare, e o 26° de Karl.
\csdisplay
This is TeX, Version 3.0 (preloaded format=plain 90.4.23)
**
|
O ``formato pr\'e carregado'' aqui se refere a uma forma pr\'e 
compilada das macros do ^{\plainTeX} que vem com o \TeX.  Voc\^e pode 
agora digitar `|screed|' para que o \TeX\ processe seu arquivo.  Quando 
estiver pronto, voc\^e ver\'a algo como:
\csdisplay
(screed.tex [1] [2] [3] )
Output written on screed.dvi (3 pages, 400 bytes).
Transcript written on screed.log.
|
exibido no seu terminal ou impresso no registro da sua execu\c c\~a se 
n\~ao estiver trabalhando em um terminal.  A maior parte dessa sa\'ida 
\'e auto explicativa.  Os n\'umeros entre par\ênteses s\~ao n\'umeros de 
p\'aginas que o \TeX\ mostra quando envia cada p\'agina do seu documento 
ao \dvifile.  \TeX\ normalmente assumir\'a uma extens\~ao `|.tex|' para 
um nome de arquivo de entrada se o nome do arquivo de entrada que voc\^e 
deu n\~ao tiver uma extens\~ao.  Para algumas formas do \TeX\ voc\^e 
pode ser capaz de invocar \TeX\ diretamente para um arquivo de entrada 
digitando:
\csdisplay
tex screed
|
ou algo como isso.

Em vez de fornecer sua entrada \TeX\ a partir de um arquivo, voc\^e pode 
digit\'a-la diretamente no seu terminal.  Para fazer isso, digite 
`^|\relax|' em vez de `|screed|' no aviso `|**|'.  \TeX\ ir\'a agora 
avisar-lhe com um `|*|' para cada linha de entrada e interpretar cada 
linha de entrada conforme a vir.  Para finalizar a entrada, digite um 
comando como `|\bye|' que diz ao \TeX\ que voc\^e est\'a pronto.  A 
entrada direta as vezes \'e uma maneira pr\'atica de experimentar o 
\TeX.

Quando seu arquivo de entrada cont\'em outros arquivos de entrada 
incorporados, as informa\c c\~oes exibidas indicam quando o \TeX\ inicia 
e termina o processamento de cada arquivo incorporado.
^^{arquivos de entrada//incorporado}
\xrdef{arquivosentrada}
O \TeX\ exibe um par\^entese esquerdo e o nome do arquivo quando ele 
come\c ca a trabalhar em um arquivo, e exibe o par\^entese direito 
correspondente quando o trabalho est\'a feito com o arquivo.  Se voc\^e 
receber quaisquer ^{mensagens de erro} na sa\'ida exibida, voc\^e 
poder\'a combin\'a-las com um arquivo procurando pelo par\^entese 
esquerdo n\~ao fechado mais recente.

Para uma explica\c c\~ao mais completa de como executar o \TeX, veja 
\knuth{Chapter~6} e sua ^{informa\c c\~ao local}.
\eix^^{executando o \TeX}


\section Preparando um arquivo de entrada

Nesta se\c c\~ao explicamos algumas das conven\c c\~oes que voc\^e deve 
seguir ao preparar a entrada para \TeX\null.  Algumas das 
informa\c c\~oes dadas aqui tamb\'em aparecem nos exemplos em 
\chapterref{exemplos} deste livro.
^^{entrada, preparando}

\subsection Comandos e sequ\^encias de controle

\bix^^{comandos}
\bix^^{sequ\^encias de controle}
A entrada para \TeX\ consiste de uma sequ\^encia de comandos que 
informam ao \TeX\ como tipografar seu documento.  A maioria dos 
caracteres age como comandos de um tipo particularmente simples: 
``tipografe-me''.  A letra `|a|', por exemplo, \'e um comando para 
tipografar um 'a'.  Mas h\'a outro tipo de comando---uma 
\emph{sequ\^encia de controle}---que d\'a ao \TeX\ uma instru\c c\~ao 
mais elaborada.  Uma sequ\^encia de controle normalmente come\c ca com 
uma barra invertida (|\|), embora voc\^e possa alterar essa 
conven\c c\~ao se precisar.
\xrdef{@backslash}
Por exemplo, a entrada:

\csdisplay
Ela cravou um punhal (\dag) no cora\c c\~ao do vil\~ao.
|
cont\'em a sequ\^encia de controle |\dag|; essa sequ\~encia produz a 
sa\'ida de tipografia:
\display{%
Ela cravou um punhal (\dag) no cora\c c\~ao do vil\~ao.
}
\noindent Tudo neste exemplo, exceto o |\dag| e os espa\c cos funcionam 
como um comando ``tipografe-me''.  Vamos explicar mais sobre espa\c cos 
em \xrefpg{spaces}.

Existem dois tipos de sequ\^encias de controle: 
\emph{palavras de controle} 
^^{palavras de controle}
e \emph{s\'imbolos de controle}:
^^{s\'imbolos de controle}
\ulist\compact
\li Uma palavra de controle consiste de uma barra invertida seguida por 
uma ou mais letras, por exemplo, `|\dag|'.  O primeiro caractere que não 
for uma letra marca o fim da palavra de controle.
\li Um s\'imbolo de controle consiste em uma barra invertida seguida por 
um caractere que n\~ao \'e uma letra, por exemplo, `|\$|'.  O caractere 
pode ser um espa\c co ou at\'e mesmo o fim de uma linha (que \'e um 
caractere perfeitamente leg\'itimo).
\endulist
\noindent
Uma palavra de controle (mas n\~ao um s\'imbolo de controle) absorve 
qualquer espa\c co ou fim de linha que o segue.  
^^{sequ\^encias de controle//absorvendo espa\c cos}
Se voc\^e n\~ao quiser perder um espa\c co ap\'os uma palavra de 
controle, siga a sequ\^encia de controle com um ^{espa\c co de controle} 
(|\!visiblespace|) ou com `|{}|'.  Assim ou:
\csdisplay
As maravilhas do \TeX\!visiblespace!.nunca cessar\~ao!!
|
ou:\hfil\ 
\csdisplay
As maravilhas do \TeX{} nunca cessar\~ao!!
|
produz:
\display{%
As maravilhas do \TeX{} nunca cessar\~ao!
}
\noindent em vez de:
\display{%
As maravilhas do \TeX nunca cessar\~ao!
}
\noindent
que \'e o que voc\^e ganharia se deixasse de fora o `|\|\visiblespace' 
ou o `|{}|'.

N\~ao execute uma palavra de controle junto com o texto que a segue---o 
\TeX\ n\~ao vai saber onde a palavra de controle termina.  Por exemplo, 
a sequ\^encia de controle |\c| coloca um acento cedilha no caractere que 
o seguir.  A palavra francesa {\it gar\c con\/} deve ser escrita como 
`|gar\c!visiblespace!.con|', e n\~ao `|gar\ccon|'; se voc\^e escrever a 
\'ultima, o \TeX\ reclamar\'a acerca de uma sequ\^encia de controle 
indefinida |\ccon|.

Um s\'imbolo de controle, por outro lado, n\~ao absorve nada que o siga.  
Assim, voc\^e deve digitar `\$13.56' como `|\$13.56|', n\~ao 
`|\$!vs13.56|'; a \'ultima forma produzir\'a `\hbox{\$ 13.56}'.  No 
entanto, esses comandos de acentua\c c\~ao que s\~ao nomeados pelos 
s\'imbolos de controle s\~ao definidos de tal forma que produzem o 
efeito de absorver um espa\c co seguinte.  Assim, voc\^e pode digitar a 
palavra francesa {\it d\'eshabiller\/} ou como `|d\'eshabiller|' ou como 
`|d\'!visiblespace!.eshabiller|'.

Cada sequ\^encia de controle tamb\'em \'e um comando, mas n\~ao o 
contr\'ario.  
^^{comandos//versus sequ\^encias de controle}
^^{sequ\^encias de controle//versus comandos}
Por exemplo, a letra `|N|' \'e um comando, mas n\~ao \'e uma sequ\^encia 
de controle.  Neste livro, usualmente usamos ``comando'' em vez de 
``sequ\^encia de controle'' quando qualquer termo serviria.  N\'os 
usamos a ``sequ\^encia de controle'' quando queremos enfatizar aspectos 
da sintaxe do \TeX\ que n\~ao se aplicam a comandos em geral.

\eix^^{sequ\^encias de controle}
\eix^^{comandos}


\subsection Argumentos

\xrdef{arg1}
Alguns comandos precisam ser seguidos por um ou mais \emph{argumentos} 
^^{arguments} que ajudem a determinar o que o comando faz.  Por exemplo, 
o comando |\vskip|, que diz ao \TeX\ para pular para baixo (ou para 
cima) a p\'agina, espera um argumento especificando quanto espa\c co 
pular.  Para pular para baixo duas polegadas, voc\^e deve digitar 
`|\vskip 2in|', onde |2in| \'e o argumento de |\vskip|.

Comandos diferentes esperam diferentes tipos de argumentos.  Muitos 
comandos esperam dimens\~oes, tal como o |2in| no exemplo acima.  Alguns 
comandos, particularmente aqueles definidos por macros, esperam 
argumentos que sejam um caractere ou algum texto entre chaves.  No 
entanto, outros exigem que seus argumentos sejam colocados entre chaves, 
isto \'e, eles n\~ao aceitam argumentos de caracteres \'unicos.  A 
descri\c c\~ao de cada comando neste livro informa quais tipos de 
argumentos, se houver algum, o comando espera.  Em alguns casos, as 
chaves obrigat\'orias definem um grupo (consulte \xref{grupo de chave}).

\secondprinting{\vfill\eject}


\subsection Par\^ametros

\xrdef{introparms}
Alguns comandos s\~ao par\^ametros (\xref{par\^ametro}).
^^{par\^ametros//como comandos}
Voc\^e pode usar um par\^ametro de duas maneiras:
\olist
\li Voc\^e pode usar o valor de um par\^ametro como um argumento para 
outro comando.  Por exemplo, o comando \hbox{|\vskip\parskip|} provoca 
um salto vertical segundo o valor do par\^ametro de cola |\parskip| 
(pular par\'agrafo).
\li Voc\^e pode alterar o valor do par\^ametro atribuindo algo a ele.  
Por exemplo, a atribui\c c\~ao \hbox{|\hbadness=200|} faz com que o 
valor do par\^ametro num\'erico de |\hbadness| seja $200$.
\endolist
\noindent
N\'os tamb\'em usamos o termo ``par\^ametro'' para nos referirmos a 
entidades como |\pageno| que, na verdade, s\~ao registros, mas se 
comportam como par\^ametros.
^^{registros//como par\^ametros}

Alguns comandos s\~ao nomes de tabelas.  Esses comandos s\~ao usados 
como par\^ametros, exceto pelo fato de exigirem um argumento adicional 
que define uma entrada espec\'ifica na tabela.  Por exemplo, |\catcode| 
nomeia uma tabela de c\'odigos de categoria 
(\xref{c\'odigo de categoria}).  Assim, o comando \hbox{|\catcode`~=13|} 
define o c\'odigo de categoria do caractere `|~|' para $13$.


\subsection Spaces

\xrdef{spaces}
\bix^^{spaces}
You can freely use extra spaces in your input.  Under nearly all circumstances
\TeX\ treats several spaces in a row as being equivalent to a
single space.  For instance, it doesn't matter whether you put one space
or two spaces after a ^{period} in your input.  Whichever you do, \TeX\
performs its end-of-sentence maneuvers and leaves the appropriate
(in most cases) amount of space after the period.
\TeX\ also treats the end of an input line as equivalent to a space.
Thus you can end your input lines wherever it's convenient---%
\TeX\ makes input
lines into
paragraphs in the same way no matter where the line breaks are in your
input.

A blank line in your input marks the end of a paragraph.
^^{paragraphs//ending}
Several blank lines are equivalent to a single one.

\TeX\ ignores input spaces within math formulas (see below).  Thus you can
include or omit spaces anywhere within a math formula---\TeX\ doesn't care.
Even within a math formula, however,
you must not run a control word together with a following letter.

If you are defining your own macros, you need to be particularly careful about
where you put ends of line in their definitions.
It's all too easy to define a macro that produces an
^{unwanted space} in addition to whatever else it's supposed to produce.
We discuss this problem elsewhere since it's somewhat
technical; see \xrefpg{unwantedspace}.

A space or its equivalent between two words in your input doesn't simply turn
into a space character in your output.
A few of these input spaces turn into ends of lines
in the output,
since input lines generally don't correspond to output lines.
The others turn into spaces of variable width called ``glue'' (\xref{glue}),
which has a natural size (the size it ``wants to be'')
but can stretch or shrink.
When \TeX\ is typesetting a paragraph
that is supposed to have an even right margin (the usual
case), it adjusts the widths of the glue in each line
to get the lines to end at the margin.
(The last line of a paragraph is an exception, since it isn't ordinarily
required to end at the right margin.)

You can prevent an input space from turning into an end of line by using a
^{tie} (^|~|).
For example, you wouldn't want \TeX\ to put a line break between the
`Fig.' and `8' of `Fig.~8'.
By typing `|Fig.~8|' you can prevent such a line break.
\eix^^{spaces}
\needspace{2in}

\subsection Comments

\xrdef{comments}
\pix\bix^^{comments}
You can include comments in your \TeX\ input.
When \TeX\ sees a comment it just passes over it, so 
what's in a comment doesn't affect your typeset document in any way.
Comments are useful for
providing extra information about what's in your input file.
For example:
\csdisplay
% ========= Start of Section `Hedgehog' =========
|

{\indexchar % }%
A comment starts with a percent sign (|%|) and extends to the end of the
input line.
\TeX\ ignores not just the comment but the end of the line as well, so
comments have another very
important use: connecting two lines so that the end of line
^^{line breaks//deleting}
between them is invisible to \TeX\ and doesn't generate
an output space or an end of line.
For instance, if you type:
\csdisplay
A fool with a spread%
sheet is still a fool.
|
you'll get:
\display{
A fool with a spread%
sheet is still a fool.
}
\eix^^{comments}


\subsection Punctuation

\null
\xrdef{periodspacing}
\TeX\ normally adds some extra space after what it thinks is a
^{punctuation} mark at the end of a sentence,
namely, `^|.|', `^|?|', or `|!!|' \indexchar !
^^{period} ^^{question mark} ^^{exclamation point}
followed by an input space.
\TeX\ doesn't add
the extra space if the punctuation mark follows
a capital letter, though, because it assumes the capital
letter to be an initial in someone's name.
You can force the extra space where it wouldn't otherwise occur by
typing something like:
\csdisplay
A computer from IBM\null?
|
The |\null| doesn't produce any output, but it does prevent \TeX\
from associating the capital `M' with the question mark.
On the other hand, you can cancel the
extra space where it doesn't belong by typing a control space
after the punctuation mark, e.g.:
\csdisplay
Proc.\!visiblespace!.Royal Acad.\!visiblespace!.of Twits
|
so that you'll get:
\display{Proc.\ Royal Acad.\ of Twits}
\noindent rather than:
\display{Proc. Royal Acad. of Twits}

Some people prefer not to leave more space after punctuation at the
end of a sentence.  You can get this effect with the
^|\frenchspacing| command (\xref\frenchspacing).
|\frenchspacing| is often recommended for ^{bibliographies}.

For single ^{quotation marks}, you should use the left and right
single quotes
(|`| and |'|) on your keyboard.  For left and right
double quotation marks, use two left single
quotes or two right single quotes (|``| or |''|) rather
than the double quote (|"|) on your keyboard.
The keyboard double quote
will in fact give you a right double quotation mark in
many fonts, but the two right single quotes
are the preferred \TeX\ style.
For example:

\vbox{%
\csdisplay
There is no `q' in this sentence.
``Talk, child,'' said the Unicorn.
She said, ``\thinspace`Enough!!', he said.''
|
}%
These three lines yield:
\display{\par\restoreplainTeX
There is no `q' in this sentence.
\par ``Talk, child,'' said the Unicorn.
\par She said, ``\thinspace`Enough!', he said.''
}
\noindent
The |\thinspace| in the third input line prevents
the single quotation mark from coming
too close to the double quotation marks.
Without it, you'd just see three 
nearly equally spaced quotation marks in a row.

\TeX\ has three kinds of ^{dashes}:
\ulist\compact
\li Short ones (hyphens) like this ( - ). You get them by typing~`^|-|'.
\li Medium ones (en-dashes) like this ( -- ). You get them by typing~`^|--|'.
\li Long ones (em-dashes) like this ( --- ). You get them by typing~`^|---|'.
\endulist
\noindent
Typically you'd use hyphens to indicate compound words like
``will-o'-the-wisp'',
en-dashes to indicate
page ranges such as ``pages~81--87'', and em-dashes to indicate
a break in continuity---like this.


\subsection Special characters

Certain characters have special meaning to \TeX, so you shouldn't use them
in ordinary text.  They are:

\csdisplay
    $  #  &  %  _  ^  ~  {  }  \
|
^^|$//in ordinary text|
^^|#//in ordinary text|
^^|&//in ordinary text|
^^|_//in ordinary text|
^^|^//in ordinary text|
^^|~//in ordinary text|
^^|%//in ordinary text|
^^|{//in ordinary text|
^^|}//in ordinary text|
{\recat!ttidxref[\//in ordinary text]]
\noindent
In order to produce them in your typeset document,
you need to use circumlocutions.  For the first five,
you should instead type:
^^|\$|
^^|\#|
^^|\&|
^^|\%|
^^|\_|
\csdisplay
    \$  \#  \&  \%  \_
|

\noindent
For the others, you need something more elaborate:

\csdisplay
   \^{!visiblespace}   \~{!visiblespace}   $\{$   $\}$   $\backslash$
|


\subsection Groups

\bix^^{groups}
A \emph{group} 
consists of material enclosed in matching left and right braces (|{| and 
|}|).
^^|{//starting a group|
^^|}//ending a group|
By placing a command within a group, you can limit its effects to
the material within the group.  
For instance, the |\bf| command tells \TeX\ to set
something in {\bf boldface} type.  If you were to put |\bf| into your input
and do nothing else to counteract it, everything in your document following the
|\bf| would be set in boldface.
By enclosing |\bf| in a group,
you limit its effect to the group.  For example, if you type:
\csdisplay
We have {\bf a few boldface words} in this sentence.
|
\noindent you'll get:
\display{We have {\bf a few boldface words} in this sentence.}

\noindent You can also use a group to limit the effect of
an assignment to one of \TeX's parameters.
These parameters contain values that affect how \TeX\ typesets your document.
For example, the value of the |\parindent|
parameter specifies the indentation at the beginning of a paragraph.
The assignment |\parindent = 15pt|
sets the indentation to $15$ printer's points.
By placing this assignment at the beginning
of a group containing a few paragraphs, you can change
the indentation of just those paragraphs.  If you don't enclose
the assignment in a group,
the changed indentation will apply to the rest of the document (or up to the
next assignment to |\parindent|, if there's a later one).

\xrdef{bracegroup}
Not all pairs of braces indicate a group.
In particular, the braces associated with an argument for which the
braces are \emph{not} required don't indicate a group---they just
serve to delimit the argument.
Of those commands that do require braces for their arguments,
some treat the braces as defining a group
and the others interpret the argument in some special way that depends on
the command.\footnote
{More precisely, for primitive commands either
the braces define a group or they enclose tokens that aren't processed in
\TeX's stomach.
For |\halign| and |\valign| the group has a trivial
effect because everything within the braces either doesn't reach the stomach
(because it's in the template) or is enclosed in a further inner group.
^^|\halign//grouping for|
^^|\valign//grouping for|
}
\eix^^{groups}


\subsection Math formulas

\bix^^{math}
\xrdef{mathform}
A math formula can appear in text (\emph{text math})
^^{text math}
or set off on a line by itself
with extra vertical space around it (\emph{display math}).
^^{display math}
You enclose a text formula in single dollar signs (|$|)
and a displayed formula in double dollar signs (|$$|).
\ttidxref{$}\ttidxref{$$}
For example:

\csdisplay
If $a<b$, then the relation $$e^a < e^b$$ holds.
|
\noindent This input produces:
\display{\centereddisplays
If $a<b$, then the relation $$e^a < e^b$$ holds.}
\smallskip
\noindent \chapterref{math} describes the commands that are useful
in math formulas.
\eix^^{math}


\section How \TeX\ works

In order to use \TeX\ effectively, it helps to 
have some idea of how \TeX\ goes about
its activity of transmuting input into output.
You can imagine \TeX\ as a kind of organism with ``eyes'', 
``mouth'', ``gullet'',
``stomach'', and ``intestines''.
Each part of the organism transforms its input in some way and passes
the transformed input to the next stage.

The ^{eyes} transform an input file into a sequence of characters.
The ^{mouth} transforms the sequence of characters into a sequence of
\emph{tokens},
^^{tokens}
where each token is either a single character or a control sequence.
^^{control sequences//as tokens}
The gullet expands the tokens into a sequence of 
\emph{primitive commands}, which are also tokens.
^^{expanding tokens}
The ^{stomach} carries out the operations specified by the primitive commands,
producing a sequence of pages.
Finally, the ^{intestines} transform each page into the form required
for the \dvifile\ and send it there.
^^{\dvifile//created by \TeX's intestines}
These actions are described in more detail 
in \chapterref{concepts} under \conceptcit{\anatomy}.
^^{\anatomy}

The real typesetting goes on in the stomach.
The commands instruct \TeX\ to typeset such-and-such a character in
such-and-such a font, to insert an interword space, to end a paragraph, and
so on.
Starting with individual typeset characters and other simple typographic
elements, \TeX\ builds up a page ^^{pages} as a nest of
^{boxes} within boxes within boxes \seeconcept{box}.
Each typeset character occupies a box, and so does an entire page.
A box can contain not just smaller boxes but also \emph{glue} ^^{glue}
(\xref{glue}) and a few other things.
The glue produces
space between the smaller boxes.  
An important property of glue is that it can stretch and shrink;
thus \TeX\ can make a box
larger or smaller by stretching or shrinking
the glue within~it.

Roughly speaking, a line is a box containing a sequence of character boxes,
and a page is a box containing a sequence of line boxes.
There's glue between the words of a line and between the lines of a page.
\TeX\ stretches or shrinks
the glue on each line so as to make the right margin
of the page come out even and the glue on each page
so as to make the bottom margins of different pages be equal.
Other kinds of typographical elements can also appear in a line or in a page,
but we won't go into them here.

As part of the process of assembling pages, \TeX\ needs to break paragraphs
into lines and lines into pages.  The stomach first sees a paragraph as one
long line, in effect.  It inserts \emph{line breaks}
^^{line breaking}
in order to transform
the paragraph into a sequence of lines of the right length, performing a
rather elaborate analysis in order to choose the set of breaks
that makes the paragraph look best
\seeconcept{line break}.
The stomach carries out a similar
but simpler process in order to transform a sequence of lines into a page.
Essentially the stomach accumulates lines until no more lines can fit on the
page.  It then chooses a single place to break the page, putting the lines
before the break on the current page
and saving the lines after the break for the
next page \seeconcept{page break}. ^^{page breaks//inserted by \TeX's stomach}

When \TeX\ is assembling an entity from a list of items (boxes, glue, etc.),
it is in one of six
\emph{modes} ^^{modes} (\xref{mode}).
The kind of entity it is assembling defines the mode that it is in.  
There are two ordinary modes: ordinary horizontal mode for assembling
paragraphs (before they are broken into lines) 
and ordinary vertical mode for assembling pages.
There are two restricted modes:
restricted horizontal mode for appending items horizontally to form
a horizontal box
and internal vertical mode for appending items vertically to form
a vertical box (other than a page).
Finally, there are two math modes: text math mode for assembling math formulas
within a paragraph and display math mode for assembling math formulas that are
displayed on lines by themselves (see ``Math formulas'', \xref{mathform}).


\section New \TeX\ versus old {\TeX}

\xrdef{newtex}
In 1989 Knuth made a major revision to \TeX\ in order to 
adapt it to the
character sets needed to support typesetting for languages other than
English.\space ^^{foreign languages}
The revision included a few minor extra features that could be added
without disturbing anything else.
This book describes ``^{\newTeX}''.
If you're still using an older version
of \TeX\ (version $2.991$ or earlier),
you'll want to know what features of {\newTeX} you can't use.
The following features aren't available in the older versions:
\ulist\compact
\li ^|\badness| (\xref\badness)
\li ^|\emergencystretch| (\xref\emergencystretch)
\li ^|\errorcontextlines| (\xref\errorcontextlines)
\li ^|\holdinginserts| (\xref\holdinginserts)
\li ^|\language|, ^|\setlanguage|, and |\new!-lan!-guage|
(\pp\xrefn\language, \xrefn{\@newlanguage}) ^^|\newlanguage|
\li ^|\lefthyphenmin| and ^|\righthyphenmin| (\xref\lefthyphenmin)
\li ^|\noboundary| (\xref\noboundary)
\li ^|\topglue| (\xref\topglue)
\li The |^^|$xy$ notation for hexadecimal digits (\xref{hexchars})
\endulist
\noindent
We recommend that you obtain new \TeX\ if you can.

\section Resources

\xrdef{resources}
A number of resources are available to help you in using \TeX.
\texbook\ is the definitive source of information on \TeX:

\smallskip
{\narrower\noindent
^{Knuth, Donald E.}, \texbook.  Reading, Mass.: Addison-Wesley, 1984.\par}
\smallskip
\noindent
Be sure to get the seventeenth printing (January 1990) or later;
the earlier printings don't cover the features of new \TeX.

^{\LaTeX} is a very popular collection of commands designed to simplify the use
of \TeX.  It is described in:
\smallskip
{\narrower\noindent\frenchspacing\spaceskip = 3.33pt plus 2pt minus 1.2pt
^{Lamport, Leslie}, {\sl The \LaTeX\ Document Preparation System}.
Reading, Mass.: Addison-Wesley, 1986.\par}
\smallskip
\noindent
^{\AMSTeX} is the collection of commands adopted by the American Mathematical
Society as a standard for submitting mathematical man\-u\-scripts
electronically.
It is described in:
\smallskip
{\narrower\noindent
^{Spivak, Michael~D.}, {\sl The Joy of \TeX}. Providence, R.I.:
American Mathematical Society, 1986.
\par}
\smallskip
\noindent
You can join the ^{\TUG} (TUG), which publishes a newsletter
called {\it ^{TUGBoat}}.
TUG is an excellent source not only for information about \TeX\ but also
for collections of macros, including \AMSTeX.
Its address is:
\smallskip
{\obeylines
^{\TUG}
c/o American Mathematical Society
P.O. Box 9506
Providence,  RI  02940
U.S.A.
}
\smallskip
\noindent
Finally, you can obtain copies of the ^|eplain.tex| macros
described in \chapterref{eplain} as well as the macros used in typesetting
this book.
They are available through the Internet network by anonymous \ftp\ from the
following hosts:
{\obeylines\display{\tt
labrea.stanford.edu [36.8.0.47]
ics.uci.edu [128.195.1.1]
june.cs.washington.edu [128.95.1.4]}}

The electronic version includes additional macros 
that format input for the
^{\BibTeX}\ computer program, written by Oren Patashnik at Stanford
University, ^^{Patashnik, Oren}
and print the output from that program.
If you find bugs in the macros, or think of improvements, you can send
electronic mail to Karl at {\tt karl@cs.umb.edu}.

The macros are also available for US \$10.00 on $5\frac1/4$\inches\
or $3\frac1/2$\inches\ PC-format diskettes from:
\smallskip
{\obeylines
Paul Abrahams
214 River Road
Deerfield,  MA  01342
\vskip\tinyskipamount
Email: {\tt Abrahams\%Wayne-MTS@um.cc.umich.edu}
}
\smallskip
\noindent
These addresses are correct as of June 1990; please be aware that they may
change after that, particularly the electronic addresses.

\endchapter\byebye
