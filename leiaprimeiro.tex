% Isto é parte do livro TeX para o Impaciente.
% Copyright (C) 2003 Paul W. Abrahams, Kathryn A. Hargreaves, Karl Berry.
% Veja o arquivo fdl.tex para condições de reprodução.

\input macros.tex
\frontchapter{Leia isto primeiro}

% Não precisamos de nada, exceto \rm aqui.
{\font\rm = cmr10 scaled \magstephalf \baselineskip = 1.1\baselineskip
Se voc\^e \'e novo ao \TeX:
\ulist
\li Leia primeiro as Se\c c\~oes 
\chapternum{uselivro}--\chapternum{usingtex};
\li Veja os exemplos em \chapterref{examples} para coisas que se parecem 
com o que voc\^e quer fazer.  Procure qualquer comando relacionado em 
``Resumo da c\'apsula de comandos'', \chapterref{capsule}.  Use as 
refer\^encias de p\ágina l\'a para encontrar descri\c c\~oes mais 
completas desses comandos e outros que s\~ao similares;
\li Procure palavras desconhecidas em ``Conceitos'', 
\chapterref{concepts}, usando a lista na capa trazeira do livro para 
encontrar a explica\c c\~ao rapidamente;
\li Experimente e explore.
\endulist
\bigskip
\noindent
Se voc\^e j\'a \'e familiarizado com \TeX, ou se voc\^e est\'a editando 
ou de qualquer modo modificando um documento do \TeX\ que outra pessoa 
criou:
\ulist
\li Para um lembrete r\'apido do que um comando faz, olhe em 
\chapterref{capsule}, ``Resumo da c\'apsula de comandos''.  O resumo 
est\'a em ordem alfab\'etica e tem refer\^encias de p\'agina para 
descri\c c\~oes mais completas dos comandos;
\li Use os agrupamentos funcionais de descri\c c\~oes de comando para 
encontrar aqueles relacionados a um comando espec\'ifico que voc\^e j\'a 
conhece ou para localizar um comando que atenda a um prop\'osito 
espec\'ifico;
\li Use \chapterref{concepts}, ``Conceitos'', para obter uma 
explica\c c\~ao de qualquer conceito que voc\^e n\~ao entende, ou 
precisa entender com mais precis\~ao, ou tenha esquecido.  Use a lista 
na contracapa do livro para encontrar um conceito rapidamente.
\endulist
}
\pagebreak

\byebye
