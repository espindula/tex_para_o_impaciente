% This is part of the book TeX for the Impatient.
% Copyright (C) 2003 Paul W. Abrahams, Kathryn A. Hargreaves, Karl Berry.
% Copyright (C) 2004 Marc Chaudemanche pour la traduction fran�aise.
% See file fdl.tex for copying conditions.

\input fmacros
\chapter {Commandes pour \linebreak des op\'erations g\'en\'erales}

\chapterdef{general}

Cette section couvre 
des ^{caract\'eristiques de programmation} de \TeX\ et
toutes autres choses qui ne rentrent pas dans les cat\'egories
de commandes des chapitres pr\'ec\'edents.
Pour une explication des conventions utilis\'ees dans cette section,
voir \headcit{Descriptions des commandes}{cmddesc}.

\begindescriptions

%==========================================================================
\section {Nommer et modifier des polices}

\begindesc
\bix^^{polices//nommer et modifier}
\cts font {}
\aux\cts font {\<s\'equence de contr\^ole> = \<nom de police>}
\aux\cts font {\<s\'equence de contr\^ole> = \<nom de police> {\bt scaled} \<nombre>}
\aux\cts font {\<s\'equence de contr\^ole> = \<nom de police> {\bt at} \<dimension>}
\explain
Utilis\'ee seule, la s\'equence de contr\^ole |\font| d\'esigne la police courante.
|\font| n'est pas une vraie commande quand elle est utilis\'ee seule, 
puis\-qu'elle ne peut alors appara\^\i tre que comme argument d'une autre commande.

Pour les trois autres formes de |\font|,
\<nom de police> nomme un jeu de fichier qui d\'efinit une police.
Ces formes de |\font| sont des commandes.  Chacune de ces formes a deux 
effets~:
{\tighten
\olist
\li Elle d\'efinit \<s\'equence de contr\^ole> comme un nom qui s\'electionne
la police \<nom de police>, \'eventuellement magnifi\'ee (voir plus bas).
\li Elle demande \`a \TeX\ de charger le fichier de m\'etriques 
^^{fichiers de m\'etriques} de la police (^{\tfmfile}) pour \<nom de 
police>.
\endolist
}% end tighten

\noindent
Le nom d'un fichier de police indique normalement sa taille de dessin.
Par exemple, |cmr10| indique Computer Modern romain avec une
taille de dessin de $10$ points.
La taille des dessins d'une police est enregistr\'ee dans ses fichiers de m\'etriques.

Si ni |scaled| \<nombre> ni |at| \<dimension>
n'est pr\'esent, la police est utilis\'ee 
\`a sa taille de dessin---la taille pour laquelle elle est normalement faite.
Autrement, une version magnifi\'ee de la police est charg\'ee~:
\ulist
\li Si |scaled| \<nombre>
est pr\'esent, la police est magnifi\'ee d'un facteur de $\hbox{\<nombre>}/1000$.
\li Si |at| \<dimension> est pr\'esent, la police est 
agrandie de \<dimension> en la magnifiant
par $\hbox{\<dimension>}/ds$, o\`u $ds$ est la taille de dessin de
\<nom de police>.
\<dimension> et $ds$ sont toujours donn\'es en points.
\endulist
\noindent
Des magnifications de moins que $1$ sont possibles~; Elles r\'eduisent la taille.

Vous devez habituellement fournir un fichier de formes (\xref{shape}) pour chaque
magnification de police que vous chargez.
N\'eanmoins, certains ^{pilotes de p\'eriph\'eriques} peuvent utiliser des polices
qui sont r\'esidentes dans une imprimante. ^^{polices r\'esidentes}
De telles polices n'ont pas besoin de fichiers de formes.

Voir \conceptcit{police} et
\conceptcit{magnification} pour plus d'informa\-tion.

\example
\font\tentt = cmtt10
\font\bigttfont = cmtt10 scaled \magstep2
\font\eleventtfont = cmtt10 at 11pt
First we use {\tentt regular CM typewriter}.
Then we use {\eleventtfont eleven-point CM typewriter}.
Finally we use {\bigttfont big CM typewriter}.
|
\produces
\font\regttfont = cmtt10
\font\bigttfont = cmtt10 scaled \magstep 2
\font\eleventtfont = cmtt10 at 11pt
First we use {\regttfont regular CM typewriter}.
Then we use {\eleventtfont eleven-point CM typewriter}.
Finally we use {\bigttfont big CM typewriter}.
\endexample
\enddesc

\begindesc
\cts fontdimen {\<nombre> \<police>\param{dimension}}
\explain
^^{polices//param\`etres de}
Ce param\`etre sp\'ecifie des dimensions vari\'ees associ\'ees avec
la police nomm\'ee par la s\'equence de contr\^ole \<police>
(\`a distinguer de \<font\-name> qui nomme les fichiers de police).
Des valeurs de ces param\`etres sont sp\'ecifi\'es dans les fichiers de m\'etriques
pour \<police>, mais vous pouvez
r\'eparer ou changer leurs valeurs pendant une ex\'ecution de \TeX.
Les nombres et significations de ces param\`etres sont~:
\display{\halign{\hfil#\hfil\quad&#\hfil\cr
\it Nombre&\it Signification\cr
\noalign{\vskip 1\jot}%
1&inclinaison par point\cr
2&espace inter-mot\cr
3&\'etirement inter-mot\cr
4&r\'etr\'ecissement inter-mot\cr
5&hauteur de x (taille d'un |1ex|)\cr
6&largeur de cadratin (taille d'un |1em|)\cr
7&espace suppl\'ementaire\cr}}
\noindent
\TeX\ utilise l'inclinaison par point pour positionner des accents.
Il utilise aussi les param\`etres inter-mot pour produire des espaces inter-mots
(voir |\spaceskip|, \xref\spaceskip) et le param\`etre d'espace suppl\'ementaire
pour l'espa\-ce additionel apr\`es une virgule (voir |\xspaceskip|,
\xref\xspaceskip).
Les valeurs de ces param\`etres pour les
polices \plainTeX\ sont \'enum\'er\'ees dans la \knuth{page~433}{497}.
Les polices de symboles math\'ematiques ont $15$ param\`etres additionnels, 
dont nous ne parlerons pas ici.

Attention~: 
les assignements de ces param\`etres \emph{ne} sont \emph{pas} d\'efaits \`a la fin
d'un groupe.
Si vous voulez changer ces param\`etres localement, vous devrez
sauvegarder et restaurer leurs valeurs originales explicitement.
\example
Here's a line printed normally.\par
\fontdimen2\font = 3\fontdimen2\font
% Triple the interword spacing.
\noindent Here's a really spaced-out line.
|
\produces
Here's a line printed normally.\par
\dimen0 = \fontdimen2\font % to undo global assignment
\fontdimen2\font = 3\fontdimen2\font
% triple the interword spacing
\noindent Here's a really spaced-out line.
\fontdimen2\font = \dimen0
\endexample
\enddesc

\begindesc
\cts magnification {{\bt =} \<nombre>}
\cts mag {\param{nombre}}
\explain
\margin{{\tt\\mag} and {\tt\\magnification} combined.}
Un assignement de |\magnification| \'etablit 
le ``^{facteur d'\'echelle}'' $f$ qui d\'etermine
le ratio de \minref{magnification} de votre document \seeconcept{magnification}.
L'assignation de |\magni!-fication| doit appara\^\i tre avant que la premi\`ere page
de votre document ait \'et\'e envoy\'ee.

L'assignement met $f$ \`a \<nombre> et met
aussi |\hsize| et |\vsize|
^^|\hsize//d\'etermin\'e par {\tt\\magnification}|
^^|\vsize//d\'etermin\'e par {\tt\\magnification}|
respectivement \`a |6.5true in| et |8.9true in|,
Les valeurs appropri\'ees pour une page $8 \frac1/2$-%
par-$11$-pouce.
$f$ doit \^etre entre $0$ et $32768$.
Le ratio \minref{magnification} du
document est $f/1000$.
Un facteur d'\'echelle
de $1000$ procure une unit\'e de magnification, c'est-\`a-dire, il laisse 
la taille de votre document inchang\'ee.  
Il est d'usage d'utiliser des puissances de $1.2$ comme facteur d'\'echelle et
la plupart des librairies de polices sont bas\'ees sur de tels facteurs.  Vous 
pouvez utiliser les commandes ^|\magstep| et ^|\magstephalf| pour sp\'ecifier 
des magnifications par ces facteurs.

|\magnification| n'est pas un param\`etre. Vous ne pouvez pas l'utiliser
pour \emph{r\'ecuperer} le facteur d'\'echelle.  Si vous \'ecrivez quelque chose comme
|\dimen0 = \mag!-nifi!-cation|, \TeX\ s'en plaindra.

Le param\`etre |\mag| contient le facteur d'\'echelle.
Changer la valeur de |\mag| retaille les dimensions de la page, ce qui n'est pas
ce que vous voulez habituellement.
Aussi, il est pr\'ef\'erable de changer la magnification en
l'assignant \`a |\magnification| plut\^ot qu'\`a |\mag|.

\example
\magnification = \magstep2 
% magnify fonts by 1.44 (=1.2x1.2)
|
\endexample
\enddesc

\begindesc
\cts magstep {\<nombre>}
\explain
Cette commande d\'eveloppe le ratio de \minref{magnification} demand\'e pour
magnifier tout dans votre document 
(autre que des dimensions |true|)
par $1.2^r$, o\`u $r$ est
la valeur de \<nombre>. \<nombre> doit \^etre entre $0$ et $5$.
\example
\magnification = \magstep1 % Magnify by ratio of 1.2.
|
\endexample
\enddesc

\begindesc
\cts magstephalf {}
\explain
Cette commande d\'eveloppe le ratio de \minref{magnification} demand\'e pour
magnifier tout dans votre document 
(autre que des dimensions |true|)
par $\sqrt{1.2}$, 
c'est-\`a-dire, \`a mi chemin entre $1$ et $1.2$.
\example
\magnification = \magstephalf
|
\endexample
\eix^^{polices//nommer et modifier}
\enddesc

%==========================================================================
\section {Convertir l'information en tokens}

\subsection {Nombres}

\begindesc
\xrdef{convert}
\bix^^{nombres//convertir en caract\`eres}
%
\cts number {\<nombre>}
\explain
Cette commande produit la repr\'esentation d'un \minref{nombre}
comme une s\'e\-quence de
\minref{token}s de caract\`ere.  Le nombre peut \^etre un entier explicite,
un param\`etre \<nombre> ou un registre \<nombre>.
\example
\number 24 \quad \count13 = -10000 \number\count13
|
\produces
\number 24 \quad \count13 = -10000 \number\count13
\endexample
\enddesc

\begindesc
^^{chiffres romain}
\easy\cts romannumeral {\<nombre>}
\explain
Cette commande produit la repr\'esentation num\'erique romaine d'un \minref{nom\-bre}
comme une s\'equence de
\minref{token}s de caract\`ere.  Le nombre peut \^etre un entier explicite,
un param\`etre \<nombre> ou un registre \<nombre>.
Si le nombre est z\'ero ou n\'egatif, |\romannumeral| ne produit
pas de tokens.
\example
\romannumeral 24 \quad (\romannumeral -16)\quad
\count13 = 6000 \romannumeral\count13
|
\produces
\romannumeral 24 \quad (\romannumeral -16)\quad
\count13 = 6000 \romannumeral\count13
\endexample

\eix^^{nombres//convertir en caract\`eres}
\enddesc

%==========================================================================
\subsection {Information environmentale}

\begindesc
^^{heure du jour}
\cts time {\param{nombre}}
\explain
\TeX\ met ce param\`etre au 
nombre de minutes qui se sont d\'eroul\'ees depuis minuit (du jour courant).
A neuf heures, par exemple, |\time| est \`a $720$.
Cette commande et les trois suivantes utilisent l'heure et la date 
enregistr\'ee dans votre ordinateur.
\TeX\ les r\'ecup\`ere juste une fois, au d\'ebut de votre ex\'ecution, donc |\time|
\`a la fin de l'ex\'ecution \`a toujours la m\^eme valeur que |\time| au
d\'ebut de l'ex\'ecution (\`a moins que vous l'ayez chang\'ee explicitement).
\enddesc

\bix^^{date}
\begindesc
\cts day {\param{nombre}}
\explain
\TeX\ met ce param\`etre au jour courant du mois.  c'est 
un nombre entre $1$ et $31$.
|\day| est mis au d\'ebut de votre ex\'ecution (voir les commentaires sur
|\time| ci-dessus).
\enddesc

\begindesc
\cts month {\param{nombre}}
\explain
\TeX\ met ce param\`etre au mois courant.  C'est
un nombre entre $1$ et $12$.
|\month| est mis au d\'ebut de votre ex\'ecution (voir les commentaires sur
|\time| ci-dessus).
\enddesc

\begindesc
\cts year {\param{nombre}}
\explain
\TeX\ met ce param\`etre \`a l'ann\'ee courante ({apr\`es \sc J.-C.}).
C'est un nombre tel que $2004$.
|\year| est mis au d\'ebut de votre ex\'ecution (voir les commentaires sur
|\time| ci-dessus).

\eix^^{date}
\enddesc

\begindesc
^^{num\'ero de version}
\cts fmtname {}
\cts fmtversion {}
\explain
Ces commandes produisent le nom et le num\'ero de version 
du format \TeX,
c'est-\`a-dire \minref{\plainTeX} ou ^{\LaTeX}, que vous utilisez.
\example
This book was produced with the \fmtname\ format,
version~\fmtversion.
|
\produces
This book was produced with the \fmtname\ format,
version~\fmtversion.
\endexample
\enddesc

\begindesc
\cts jobname {}
\explain
Cette commande produit le nom de base 
du fichier avec lequel \TeX\ \`a \'et\'e invoqu\'e.
Par exemple, si votre fichier d'entr\'ee principal est |hatter.tex|,
|\jobname| se d\'eveloppera en |hatter|.
|\jobname| est plus pratique quand vous 
cr\'eez un fichier auxiliaire devant \^etre associ\'e avec un document.
^^{fichiers auxiliaires}
\example
\newwrite\indexfile  \openout\indexfile = \jobname.idx
% For input file `hatter.tex', open index file `hatter.idx'.
|
\endexample\enddesc

%==========================================================================
\subsection {Valeurs des variables}

\begindesc
\cts meaning {\<token>}
\explain
^^{tokens//montrer la signification de}
Cette commande produit
la signification de \<token>.  Elle est pratique pour diagnostiquer en sortie.
Vous pouvez utiliser la commande ^|\the| (\xref\the) d'une mani\`ere similaire  
pour obtenir l'information sur les valeurs des \minref{registre}s et autres
entit\'es de \TeX.
\example
[{\tt \meaning\eject}] [\meaning\tenrm] [\meaning Y]
|
\produces
[{\tt \meaning\eject}] [\meaning\tenrm] [\meaning Y]
\endexample\enddesc

\begindesc
\cts string {\<s\'equence de contr\^ole>}
\explain
^^{s\'equences de contr\^ole//convertir en chaines}
Cette commande produit 
les caract\`eres qui forment le nom \<s\'equence de contr\^ole>,
en incluant le \minref{caract\`ere d'\'echappement}.
Le caract\`ere d'\'echap\-pement est repr\'esent\'e par la valeur courante de
^|\escapechar|.
^^{caract\`ere d'\'echappement//repr\'esent\'e par \b\tt\\escapechar\e}
\TeX\ donne les caract\`eres dans la liste d'un code de cat\'egorie de $12$ (outre).

Vous pouvez proc\'eder \`a l'op\'eration inverse avec
la commande ^|\csname| (\xref \csname),
qui transforme une chaine en une s\'equence de contr\^ole.
\example
the control sequence {\tt \string\bigbreak}
|
\produces
the control sequence {\tt \string\bigbreak}
\endexample\enddesc

\begindesc
\cts escapechar {\param{nombre}}
\explain
Ce param\`etre sp\'ecifie le code \ascii\ \minrefs{\ascii} du
caract\`ere que \TeX\ utilise pour repr\'esenter le \minref{caract\`ere d'\'echappement}
^^{caract\`ere d'\'echappement//repr\'esent\'e par \b\tt\\escapechar\e}
quand il
convertit un nom de s\'equence de contr\^ole en une s\'equence de tokens de caract\`ere.
Cette conversion \`a lieu quand vous utilisez la commande |\string| et aussi quand
\TeX\ produit des messages de diagnostic.  La valeur par d\'efaut du caract\`ere
d'\'echappement est $92$, la code de caract\`ere {\ascii} pour un ^{backslash}.
Si |\escapechar| n'est pas dans la fourchette $0$--$255$,
\TeX\ n'inclut pas de caract\`ere d'\'echappement dans le r\'esultat de la conversion.
\example
\escapechar = `!!
the control sequence {\tt \string\bigbreak}
|
\produces
\escapechar = `!
the control sequence {\tt \string\bigbreak}
\endexample
\enddesc

\begindesc
\cts fontname {\<police>}
\explain
^^{polices//noms de}
Cette commande produit le nom de fichier
pour la \<police>.  Le nom de fichier est le \<font\-name> qui a \'et\'e utilis\'e 
pour d\'efinir \<police>. 
\example
\font\myfive=cmr5 [\fontname\myfive]
|
\produces
\font\myfive=cmr5 [\fontname\myfive]
\endexample
\enddesc


%==========================================================================
\section {Groupement}

\begindesc
\bix^^{groupes}
%
\cts begingroup {}
\cts endgroup {}
\explain
Ces deux commandes d\'ebutent et terminent un \minref{groupe}.
Un |\begingroup| ne s'accorde pas avec une accolade droite,
ni un |\endgroup| avec une accolade gauche.

\TeX\ traite |\begingroup| et |\endgroup| comme tout autre
\minref{s\'equence de contr\^ole} quand il scanne son entr\'ee.  En particulier,
vous pouvez d\'efinir une \minref{macro} qui contient un |\begingroup|
mais pas de |\endgroup|, et inversement.
^^{macros//utiliser \b\tt\\begingroup\e\ et \b\tt\\endgroup\e\ dans des}
Cette technique est souvent pratique
quand vous d\'efinissez des paires de macros, une qui \'etablit
un environnement et l'autre qui termine cet environnement.
Vous ne pouvez, de plus, pas utiliser |\begingroup| et |\endgroup| comme 
substituts pour des
accolades autres que celles qui entourent un groupe.
\example
\def\a{One \begingroup \it two }
\def\enda{\endgroup four}
\a three \enda
|
\produces
\def\a{One \begingroup \it two }
\def\enda{\endgroup four}
\a three \enda
\endexample
\enddesc

\begindesc
\makecolumns 4/2:
\easy%
\ctsact { \xrdef{@lbrace}
\cts bgroup {}
\ctsact } \xrdef{@rbrace}
\cts egroup {}
\explain
Les accolades gauches et droites sont des commandes qui d\'ebutent et terminent un
\minref{groupe}.
Les \minref{s\'equences de contr\^ole} |\bgroup| et |\egroup| sont \'equivalentes
\`a `|{|' et `|}|', sauf que
\TeX\ traite |\bgroup| et |\egroup| comme tout autre
\minref{s\'equence de contr\^ole} quand il scanne son entr\'ee.

|\bgroup| et |\egroup| peuvent \^etre pratiques quand vous 
d\'efinissez des paires de macros, une dans laquelle
d\'ebute une construction d\'elimit\'ee par accolade (pas n\'ecessairement un groupe)
et l'autre dans laquelle ce termine cette construction.
^^{macros//utiliser \b\tt\\bgroup\e\ et \b\tt\\egroup\e\ dans des}
Vous ne pouvez pas d\'efinir de telles macros en utilisant des accolades ordinaires---si vous essayez,
vos d\'efinitions de macro contiendront des accolades non appair\'ees
et seront de ce fait inacceptables par \TeX.
Normalement vous ne devez utilisez ces commandes que quand vous ne pouvez utiliser
d'accolades ordinaires.

\example
Braces define the {\it boundaries\/} of a group.
|
\produces
Braces define the {\it boundaries\/} of a group.
\nextexample
\def\a{One \vbox\bgroup}
% You couldn't use { instead of \bgroup  here because
% TeX would not recognize the end of the macro
\def\enda#1{{#1\egroup} two}
% This one is a little tricky, since the \egroup actually
% matches a left brace and the following right brace
% matches the \bgroup.  But it works!!
\a \enda{\hrule width 1in}
|
\produces
\def\a{One \vbox\bgroup}
% You couldn't use { instead of \bgroup  here because
% TeX would not recognize the end of the macro
\def\enda#1{{#1\egroup} two}
% This one is a little tricky, since the \egroup actually 
% matches a left brace and the following right brace
% matches the \bgroup.  But it works!
\a \enda{\hrule width 1in}
\endexample
\enddesc

\begindesc
\cts global {}
\explain
Cette commande rend la d\'efinition 
ou l'\minref{assignement} suivant \minref{global} \seeconcept{global} 
pour qu'il devienne effectif ind\'ependamment des fronti\`eres du \minref{groupe}.
Vous pouvez appliquer un pr\'efix |\global| \`a toute sorte de d\'efinition
ou \minref{assignement},
en incluant une d\'efinition de \minref{macro}  ou un assignement de \minref{registre}.
\example
{\global\let\la = \leftarrow}
$a \la b$
|
\produces
% for safety's sake we fake this one!
\let\la = \leftarrow
$a \la b$
\endexample
\enddesc

\begindesc
\cts globaldefs {\param{nombre}}
\explain
Ce param\`etre contr\^ole si \TeX\ prend ou non des d\'efinitions
et autres assignements comme \'etant
\minref{globaux}~:
\ulist
\li Si |\globaldefs| est \`a z\'ero (comme il l'est par d\'efaut), une d\'efinition est globale
si et seulement si il est pr\'ec\'ed\'e par |\global| soit explicitement soit implicitement.
(Les commandes ^|\gdef| et ^|\xdef| (\xref\gdef) ont un pr\'efixe 
|\global| implicite).
\li Si |\globaldefs| est sup\'erieur \`a z\'ero, tout assignement et
d\'efinition est implicitement pr\'efix\'e par ^|\global|.
\li Si |\globaldefs| est inf\'erieur \`a z\'ero, tout pr\'efixe ^|\global| est ignor\'e.
\endulist
\enddesc

\begindesc
\margin{Order of {\tt\\aftergroup} and {\tt\\afterassignment} changed.}
\cts aftergroup {\<token>}
\explain
Quand \TeX\ rencontre cette commande dans une entr\'ee,
il sauve \<token>.
Apr\`es la fin du \minref{groupe} courant,
il r\'eins\`ere \<token> dans l'entr\'ee et le d\'eveloppe.
Si un groupe contient plusieurs |\aftergroup|s, les tokens correspondants
sont \emph{tous} ins\'er\'es suivant la fin du groupe, dans l'ordre
dans lequel ils apparaissent \`a l'origine.

L'exemple qui suit vous montre comment vous pouvez utiliser |\after!-group| 
pour diff\'erer le traitement d'un token que vous avez g\'en\'er\'e dans un
\minref{test conditionnel}.
\example
\def\neg{negative} \def\pos{positive}
% These definitions are needed because \aftergroup applies 
% to a single token, not to a sequence of tokens or even 
% to a brace-delimited text.
\def\arith#1{Is $#1>0$? \begingroup
   \ifnum #1>-1 Yes\aftergroup\pos
   \else No\aftergroup\neg\fi
   , it's \endgroup. }
\arith 2
\arith {-1}
|
\produces
\def\neg{negative} \def\pos{positive}
% These definitions are needed because \aftergroup applies 
% to a single token, not a sequence of tokens or even 
% a group.
\def\arith#1{Is $#1>0$? \begingroup
   \ifnum #1>-1 Yes\aftergroup\pos
   \else No\aftergroup\neg\fi
   , it's \endgroup. }
\arith 2
\arith {-1}
\endexample
\eix^^{groupes}
\enddesc

\begindesc
\cts afterassignment {\<token>}
\explain
Quand \TeX\ rencontre cette commande il sauve \<token> dans un endroit sp\'ecial.  
Apr\`es qu'il ait ex\'ecut\'e l'\minref{assignment} suivant, il ins\`ere
\<token> dans l'entr\'ee et le d\'eveloppe.  Si vous appelez |\afterassignment|
plus d'une fois avant un assignement, seul le dernier appel a un effet.
Une utilisation de |\afterassignment|
est dans l'\'ecriture de \minref{macro}s pour des commandes devant \^etre \'ecrites
sous la forme d'assignements, comme dans l'exemple suivant.

Voir la \knuth{page~279}{325} pour un description pr\'ecise
du fonctionnement d'|\afterassignment|.
\example
\def\setme{\afterassignment\setmeA\count255}
\def\setmeA{$\number\count255\advance\count255 by 10
   +10=\number\count255$}
Some arithmetic: \setme = 27
% After expanding \setme, TeX sets \count255 to 27 and
% then calls \setmeA.
|
\produces
\def\setme{\afterassignment\setmeA\count255}
\def\setmeA{$\number\count255\advance\count255 by 10
+10=\number\count255$}
Some arithmetic: \setme = 27
% After expanding \setme, TeX sets \count255 to 27 and
% then calls \setmeA.
\endexample
\enddesc


%==========================================================================
\section {Macros}

%==========================================================================
\subsection {D\'efinir des macros}

\begindesc
\bix^^{macros}
\bix^^{macros//d\'efinir des}
\xrdef{mac1}% begin the section on macros
%
\cts def {\<s\'equence de contr\^ole> \<texte de param\`etre> \rqbraces{\<texte de remplacement>}}
\explain
Cette commande d\'efinit \<s\'equence de contr\^ole> comme une \minref{macro} avec les
\<texte de param\`etre> et \<texte de remplacement> sp\'ecifi\'es.
Voir \xrefpg{macro} pour une explication compl\`ete sur la fa\c con d'\'ecrire une 
d\'efinition de macro.
\example
\def\add#1+#2=?{#1+#2&=
   \count255=#1 \advance\count255 by #2 \number\count255\cr}
$$\eqalign{
   \add 27+9=?
   \add -5+-8=?}$$
|
\dproduces
\def\add#1+#2=?{#1+#2&=
   \count255=#1 \advance\count255 by #2 \number\count255\cr}
$$\eqalign{
   \add 27+9=?
   \add -5+-8=?}$$
\endexample
\enddesc

\begindesc
\cts edef {\<s\'equence de contr\^ole> \<texte de param\`etre> \rqbraces{\<texte de remplacement>}}
\explain
Cette commande d\'efinit une macro de la m\^eme mani\`ere g\'en\'erale que |\def|.
La diff\'erence est que \TeX\ d\'eveloppe le \<texte de remplacement> 
d'un |\edef| imm\'ediatement (mais sans rien ex\'ecuter).
Ainsi toutes d\'efini\-tions du \<texte de remplacement> sont d\'evelopp\'ees, mais 
des assignements et commandes qui produisent des choses telles que bo\^\i tes et 
des ressorts sont laiss\'es tels quels.  Par exemple, une commande |\hbox| dans 
le \<texte de remplacement> d'un |\edef| reste comme une commande et n'est pas
transform\'ee en une bo\^\i te quand \TeX\ ex\'ecute la d\'efinition.
Il n'est pas toujours \'evident de savoir ce qui est d\'evelopp\'e et ce qui ne l'est pas, mais vous
trouverez une liste compl\`ete des s\'equences de contr\^ole d\'eveloppables sur les
\knuth{pages~212--215}{249--252}.

Vous pouvez inhiber le d\'eveloppement d'une s\'equence de contr\^ole qui devrait 
\^etre d\'evelopp\'ee en utilisant |\no!-expand| (\xref\noexpand). ^^|\noexpand|
Vous pouvez diff\'erer le d\'eveloppement d'une s\'equence de contr\^ole en utilisant la commande
^|\expandafter| (\xref\expandafter).

Les commandes |\write|, |\message|, |\errmessage|, |\wlog| et |\csname|
d\'eveloppent leurs 
listes de token en utilisant les m\^eme r\`egles que |\edef| utilise pour d\'evelopper son
texte de remplacement.
^^|\write//d\'evelopp\'e par les r\`egles d'{\tt\\edef}|
^^|\message//d\'evelopp\'e par les r\`egles d'{\tt\\edef}|
^^|\errmessage//d\'evelopp\'e par les r\`egles d'{\tt\\edef}|
^^|\wlog//d\'evelopp\'e par les r\`egles d'{\tt\\edef}|
^^|\csname//d\'evelopp\'e par les r\`egles d'{\tt\\edef}|
\example
\def\aa{xy} \count255 = 1
\edef\bb{w\ifnum \count255 > 0\aa\fi z}
% equivalent to \def\bb{wxyz}
\def\aa{} \count255 = 0 % leaves \bb unaffected
\bb
|
\produces
\def\aa{xy} \count255 = 1
\edef\bb{w\ifnum \count255 > 0\aa\fi z}
% equivalent to \def\bb{wxyz}
\def\aa{} \count255 = 0 % leaves \bb unaffected
\bb
\endexample
\enddesc

\begindesc
\cts gdef {\<s\'equence de contr\^ole> \<texte de param\`etre> \rqbraces{\<texte de remplacement>}}
\explain
Cette commande est \'equivalente \`a |\global\def|.
\enddesc

\begindesc
\cts xdef {\<s\'equence de contr\^ole> \<texte de param\`etre> \rqbraces{\<texte de remplacement>}}
\explain
Cette commande est \'equivalente \`a |\global\edef|.
\enddesc

\begindesc
\cts long {}
\explain
Cette commande est utilis\'ee comme pr\'efixe d'une d\'efinition de \minref{macro}.
Elle dit \`a \TeX\ que les arguments de la macro peuvent inclure
des tokens |\par| (\xref{\@par}), qui indiquent normalement la fin d'un paragraphe.
^^|\par//dans des arguments de macro|
Si \TeX\
essaye de d\'evelopper une macro d\'efinie sans |\long| et qu'un des arguments
de la macro comprend un token |\par|,
\TeX\ se plaindra par un ``runaway argument''.  Le but
de ce comportement est de vous fournir des protections contre des arguments de
macro non termin\'es.
|\long| vous donne un moyen d'outrepasser la protection.
\example
\long\def\aa#1{\par\hrule\smallskip#1\par\smallskip\hrule}
\aa{This is the first line.\par
This is the second line.}
% without \long, TeX would complain
|
\produces
\medskip
\long\def\aa#1{\par\hrule\smallskip#1\par\smallskip\hrule}
\aa{This is the first line.\par
This is the second line.}
% without \long, TeX would complain
\endexample
\enddesc

\begindesc
\cts outer {}
\explain
\null ^^{outer}
Cette commande est utilis\'ee comme pr\'efixe d'une d\'efinition de \minref{macro}.
Elle dit \`a \TeX\ que la macro est externe (\xref{outer})
et ne peut pas \^etre utilis\'ee dans certains contextes.
Si la macro est utilis\'ee dans un contexte interdit, \TeX\ se plaindra.

\example
\outer\def\chapterhead#1{%
   \eject\topglue 2in \centerline{\bf #1}\bigskip}
% Using \chapterhead in a forbidden context causes an
% error message.
|
\endexample
\enddesc

\begindesc
\cts chardef {\<s\'equence de contr\^ole>=\<charcode>}
\explain
^^{caract\`eres//d\'efinis par \b\tt\\chardef\e}
Cette commande d\'efinit \<s\'equence de contr\^ole> 
comme \'etant \<charcode>.
Bien que |\chardef| soit plus souvent utilis\'ee pour d\'efinir des caract\`eres, vous pouvez aussi
l'utiliser pour donner un nom \`a un nombre dans la fourchette $0$--$255$ m\^eme 
quand vous n'utilisez pas ce nombre comme un code de caract\`ere.
\example
\chardef\percent = `\% 21\percent, {\it 19\percent}
% Get the percent character in roman and in italic
|
\produces
\chardef\percent = `\%
21\percent, {\it 19\percent}
% You'll get the percent character in roman and in italic
\endexample
\enddesc

\begindesc
^^{caract\`eres math\'ematiques}
^^{mathcodes}
\cts mathchardef {\<s\'equence de contr\^ole>=\<mathcode>}
\explain
Cette commande d\'efinit la \<s\'equence de contr\^ole> comme un caract\`ere math\'ematique
avec le \minrefs{mathcode}\<mathcode> donn\'e.
La s\'equence de contr\^ole ne sera l\'egale qu'en mode math\'ematique.
\example
\mathchardef\alphachar = "010B % As in plain TeX.
$\alphachar$
|
\produces
\mathchardef\alphachar = "010B % As in plain TeX.
$\alphachar$
\endexample
\eix^^{macros//d\'efinir des}
\enddesc

%==========================================================================
\subsection {Autre d\'efinitions}

\begindesc
\cts let {\<s\'equence de contr\^ole> = \<token>}
\explain
^^{s\'equences de contr\^ole//d\'efinies avec \b\tt\\let\e}
\minrefs{token}
Cette commande fait que 
\<s\'equence de contr\^ole> prend la signification actuelle de \<token>.
M\^eme si vous red\'efinissez \<token> plus tard, la signification de \<s\'equence de contr\^ole>
ne changera pas.  Bien que \<token> soit le plus souvent une s\'equence de contr\^ole,
il peut aussi \^etre un token de \minref{caract\`ere}.
\enddesc

\begindesc
\cts futurelet {\<s\'equence de contr\^ole> \<token$_1$> \<token$_2$>}
\explain
Cette commande dit \`a \TeX\ de donner \`a \<token$_2$> la signification de
\<s\'equence de contr\^ole> (comme elle le serait avec |\let|) et ensuite 
d'ex\'ecu\-ter \<token$_1$> et \<token$_2$> normalement.
|\futurelet| est pratique \`a la fin d'une d\'efinition de macro
parce qu'elle vous donne un moyen de regarder au-del\`a du token que \TeX\ 
est en train d'ex\'ecuter avant de continuer. 
\example
\def\predict#1{\toks0={#1}\futurelet\next\printer}
% \next will acquire the punctuation mark after the
% argument to \predict
\def\printer#1{A \punc\ lies ahead for \the\toks0. }
\def\punc{%
   \ifx\next;semicolon\else
      \ifx\next,comma\else
         ``\next''\fi\fi}
\predict{March}; \predict{April}, \predict{July}/
|
\produces
\def\predict#1{\toks0={#1}\futurelet\next\printer}
\def\printer#1{A \punc\ lies ahead for \the\toks0. }
\def\punc{%
   \ifx\next;semicolon\else
      \ifx\next,comma\else
         ``\next''\fi\fi
   }
\predict{March};
\predict{April},
\predict{July}/
\endexample
\enddesc

\begindesc
\cts csname {\<liste de token> {\bt \\endcsname}}
\xrdef{\endcsname}
\explain
Cette commande produit une s\'equence de contr\^ole \`a partir de \<liste de token>.
Elle procure un moyen de synth\'etiser des s\'equences de contr\^ole,
en incluant celles que vous ne pouvez normalement pas \'ecrire.
\<liste de token> peut elle-m\^eme inclure des s\'equences de contr\^ole~: 
elle est d\'eveloppable de la m\^eme mani\`ere que le texte de remplacement d'une 
d\'efinition |\edef| (\xref\edef).
Si le d\'eveloppement final
produit quelque chose qui n'est pas un caract\`ere, \TeX\ se plaindra.
|\csname| transforme une liste de tokens en s\'equence de contr\^ole~;
vous pouvez faire l'inverse avec ^|\string| \ctsref\string.
\example
\def\capTe{Te}
This book purports to be about \csname\capTe X\endcsname.
|
\produces
\def\capTe{Te}
This book purports to be about \csname\capTe X\endcsname.
\endexample
\enddesc

%==========================================================================
\subsection {Contr\^oler le d\'eveloppement}

\begindesc
\bix^^{macros//contr\^oler leur d\'eveloppement}
\cts expandafter {\<token$_1$> \<token$_2$>}
\explain
Cette commande dit \`a \TeX\ de d\'evelopper \<token$_1$> en accord avec ses r\`egles
de d\'eveloppement de \minref{macro} \emph{apr\`es} avoir d\'evelopp\'e \<token$_2$>
d'un niveau.  C'est pratique quand \<token$_1$> est quelque chose comme `|{|'
^^|{//avec {\tt\\expandafter}|
ou ^|\string| qui inhibe les expansions de \<token$_2$>,
mais que vous voulez d\'evelopper \<token$_2$> quand m\^eme.
\example
\def\aa{xyz}
\tt % Use this font so `\' prints that way.
[\string\aa]  [\expandafter\string\aa]
[\expandafter\string\csname TeX\endcsname]
|
\produces
\def\aa{xyz}
\tt
[\string\aa]  [\expandafter\string\aa]
[\expandafter\string\csname TeX\endcsname]
\endexample
\enddesc

\begindesc
\cts noexpand {\<token>}
\explain
Cette commande demande \`a \TeX\ de 
supprimer le d\'eveloppement de \<token> si \<token> est une 
\minref{s\'equence de contr\^ole} qui peut \^etre d\'evelopp\'ee.
Si \<token> ne peut pas \^etre d\'evelopp\'ee, c'est-\`a-dire, si c'est une lettre,
\TeX\ agit comme si le |\noexpand| n'\'etait pas l\`a
et ex\'ecute \<token> normalement.
En d'autres mots le d\'eveloppement de `|\noexpand|\<token>'
est simplement \<token> quoi que \<token> devienne.
\example
\def\bunny{rabbit}
\edef\magic{Pull the \noexpand\bunny\ out of the hat!! }
% Without \noexpand, \bunny would always be replaced
% by `rabbit'
\let\oldbunny=\bunny \def\bunny{lagomorph} \magic
\let\bunny=\oldbunny \magic
|
\produces
\def\bunny{rabbit}
\edef\magic{Pull the \noexpand\bunny\ out of the hat! }
% Without \noexpand, \bunny would always be replaced
% by `rabbit'
\let\oldbunny=\bunny \def\bunny{lagomorph} \magic
\let\bunny=\oldbunny \magic
\endexample
\enddesc

\begindesc
\cts the {\<token>}
\explain
Cette commande se d\'eveloppe g\'en\'eralement en une liste de tokens de \minref{caract\`ere}s
que repr\'esente \<token>.  \<token> peut \^etre parmi ce qui suit~:

\ulist\compact
\li un \minref{param\`etre} \TeX, par exemple, |\parindent| ou |\deadcycles|
^^{param\`etres//utiliser \b\tt\\the\e\ avec}
\li un \minref{registre}, par exemple, |\count0|
^^{registres//avec \b\tt\\the\e}
\margin{Item for special registers removed}
\li un code associ\'e avec un caract\`ere entr\'e, par exemple, |\catcode`(|
\li un param\`etre de police, par exemple, |\fontdimen3\sevenbf|
\li le ^|\hyphenchar| ou le ^|\skewchar| d'une police, 
|\skewchar\teni|, par exemple
\li ^|\lastpenalty|, ^|\lastskip| ou ^|\lastkern| (valeurs d\'eriv\'ees du
der\-nier \'el\'ement d'une liste horizontale \minrefs{horizontal list}
ou \minref{liste verticale} courante)
\li une s\'equence de contr\^ole d\'efinie par ^|\chardef| ou
^|\mathchardef|
\endulist

\noindent
De plus, |\the| peut d\'evelopper des tokens non-caract\`ere dans les deux cas
suivants~: 
\ulist\compact
\li |\the| \<police>, qui d\'eveloppe la s\'equence de contr\^ole la plus r\'ecem\-ment 
d\'efinie qui s\'electionne la m\^eme police que la s\'equence de contr\^o\-le \<police>
\li |\the| \<variable de token>, qui d\'eveloppe une copie de la valeur de la
variable, par exemple, |\the\everypar|
\endulist

Voir les \knuth{pages~214--215}{251} pour une description plus d\'etaill\'ee de ce que
|\the| fait selon les cas.
\example
The vertical size is currently \the\vsize.
The category code of `(' is \the\catcode `(.
|
\produces
The vertical size is currently \the\vsize.
The category code of `(' is \the\catcode `(.
\endexample
\enddesc

{\tighten
\see \headcit{Convertir l'information en tokens}{convert} ainsi que la commande
|\showthe| (\xref\showthe).
\par}

\eix^^{macros//contr\^oler leur d\'eveloppement}

%==========================================================================
\subsection {Tests conditionnels}

\begindesc
\xrdef{conds}
\bix^^{tests conditionnels}
%
\ctspecial if {\<token$_1$> \<token$_2$>}\ctsxrdef{@if}
\explain
{\emergencystretch=1em
Cette commande teste si \<token$_1$> et \<token$_2$>
ont le m\^eme code de \minref{caract\`ere}, ind\'ependamment de leurs
\minref{codes de cat\'egorie}.
Avant d'ex\'ecuter le test, \TeX\ d\'eveloppe les tokens suivant le |\if|
jusqu'\`a ce qu'il obtienne deux tokens qui ne peuvent se d\'evelopper plus.  
Ces deux tokens deviennent \<token$_1$> et \<token$_2$>.
Le d\'eveloppement 
inclut le remplacement d'une s\'equence de contr\^ole |\let| \'egale \`a un token de caract\`ere
par ce token de caract\`ere.
Une \minref{s\'equence de contr\^ole} qui ne peut plus se d\'evelopper est
consid\'er\'ee comme ayant un code de caract\`ere $256$.\par}
\example
\def\first{abc}
\if\first true\else false\fi;
% ``c'' is left over from the expansion of \first.
% It lands in the unexecuted ``true'' part.
\if a\first\ true\else false\fi;
% Here ``bc'' is left over from the expansion of \first
\if \hbox\relax true\else false\fi
% Unexpandable control sequences test equal with ``if''
|
\produces
\def\first{abc}
\if\first true\else false\fi;
% ``c'' is left over from the expansion of \first.
% It lands in the unexecuted ``true'' part.
\if a\first\ true\else false\fi;
% Here ``bc'' is left over from the expansion of \first
\if \hbox\relax true\else false\fi
% Unexpandable control sequences test equal with ``if''
\endexample
\enddesc

\begindesc
\ctspecial ifcat {\<token$_1$> \<token$_2$>}\ctsxrdef{@ifcat}
\explain
^^{codes de cat\'egorie//tester}
Cette commande teste si \<token$_1$> et \<token$_2$>
ont le m\^eme \minref{code de cat\'ego\-rie}.
Avant d'ex\'ecuter le test, \TeX\ d\'eveloppe les tokens suivant le |\if|
jusqu'a ce qu'il obtienne deux tokens qui ne puissent plus se d\'evelopper.  
Ces deux tokens deviennent \<token$_1$> et \<token$_2$>.
Le d\'eveloppement 
inclut le remplacement d'une s\'equence de contr\^ole |\let| \'egale \`a un token de caract\`ere
par ce token de caract\`ere.
Une \minref{s\'equence de contr\^ole} qui ne peut plus se d\'evelopper est
consid\'er\'ee comme ayant un code de caract\`ere $16$.
\example
\ifcat axtrue\else false\fi;
\ifcat ]}true\else false\fi;
\ifcat \hbox\day true\else false\fi;
\def\first{12345}
\ifcat (\first true\else false\fi
% ``2345'' lands in the true branch of the test 
|
\produces
\ifcat axtrue\else false\fi;
\ifcat ]}true\else false\fi;
\ifcat \hbox\day true\else false\fi;
\def\first{12345}
\ifcat (\first true\else false\fi
% ``2345'' lands in the true branch of the test 
\endexample
\enddesc

\begindesc
\ctspecial ifx {\<token$_1$> \<token$_2$>}\ctsxrdef{@ifx}
\explain
Cette commande teste si \<token$_1$> et \<token$_2$> s'accordent.
Contrairement \`a |\if| et |\ifcat|, |\ifx| \emph{ne}  d\'eveloppe \emph{pas} les tokens
suivant |\ifx|, donc \<token$_1$> et \<token$_2$> sont les deux
tokens imm\'ediatement apr\`es |\ifx|.
Il y a trois cas~:
\olist
\li Si un token est une \minref{macro} et pas l'autre,
les tokens ne s'accordent pas.
\li Si aucun token n'est une macro, Les tokens s'accordent si~:
\olist
\li les deux tokens sont des caract\`eres (ou des s\'equences de contr\^ole d\'ecrivant des caract\`eres) et
que leurs codes de \minref{caract\`ere} et de \minref{cat\'egorie} s'accordent, ou
\li Les deux tokens font r\'ef\'erence \`a la m\^eme commande \TeX, police, etc.
\endolist
\li Si les deux tokens sont des macros, les tokens s'accordent si~:
\olist\compact
\li leurs d\'eveloppements de ``premier niveau'', c'est-\`a-dire,
leurs textes de remplacement, sont identiques, et
\li Ils ont le m\^eme statut concernant ^|\long| (\xref\long)
et ^|\outer| (\xref\outer).
\endolist
Notez en particulier que \emph{deux s\'equences de contr\^ole non d\'efinies
s'accordent}.
\endolist
\noindent
Ce teste est g\'en\'eralement plus utile que |\if|.
\example
\ifx\alice\rabbit true\else false\fi;
% true since neither \rabbit nor \alice is defined
\def\a{a}%
\ifx a\a true\else false\fi;
% false since one token is a macro and the other isn't
\def\first{\a}\def\second{\aa}\def\aa{a}%
\ifx \first\second true\else false\fi;
% false since top level expansions aren't the same
\def\third#1:{(#1)}\def\fourth#1?{(#1)}%
\ifx\third\fourth true\else false\fi
% false since parameter texts differ
|
\produces
\ifx\alice\rabbit true\else false\fi;
% true since neither \rabbit nor \alice is defined
\def\a{a}%
\ifx a\a true\else false\fi;
% false since one token is a macro and the other isn't
\def\first{\a}\def\second{\aa}\def\aa{a}%
\ifx \first\second true\else false\fi;
% false since top level expansions aren't the same
\def\third#1:{(#1)}\def\fourth#1?{(#1)}%
\ifx\third\fourth true\else false\fi
% false since parameter texts differ
\endexample
\enddesc

\begindesc
\ctspecial ifnum {\<number$_1$> \<relation> \<number$_2$>}\ctsxrdef{@ifnum}
\explain
^^{nombres//comparer}
Cette commande teste si \<number$_1$> et \<number$_2$>
satisfont \<relation>, qui peut \^etre soit `|<|', `|=|' ou `|>|'.
Les nombres peuvent \^etre des constantes telles que |127|, des registres de compteur tels que
|\pageno| ou |\count22| ou des param\`etres num\'eriques tels que |\hbadness|.
Avant d'ex\'ecuter le test, \TeX\ d\'eveloppe les tokens suivant le |\ifnum|
jusqu'a ce qu'il obtienne une s\'equence de tokens ayant la forme \<number$_1$> 
\<relation> \<number$_2$> suivi par un token
qui ne peut \^etre une partie de \<number$_2$>.
\example
\count255 = 19 \ifnum \count255 > 12 true\else false\fi
|
\produces
\count255 = 19 \ifnum \count255 > 12 true\else false\fi
\endexample
\enddesc

\begindesc
\ctspecial ifodd {\<nombre>}\ctsxrdef{@ifodd}
\explain
^^{nombres//tester la parit\'e}
Cette commande teste si \<nombre> est impair.
Avant d'ex\'ecuter le test, \TeX\ d\'eveloppe les tokens suivant le |\ifodd|
jusqu'a ce qu'il obtienne une s\'equence de tokens ayant la forme \<nombre>,
suivi par un token qui ne peut \^etre une partie de \<nombre>.
\example
\count255 = 19
\ifodd 5 true\else false\fi
|
\produces
\ifodd 5 true\else false\fi
\endexample
\enddesc

\begindesc
\ctspecial ifdim {\<dimen$_1$> \<relation> \<dimen$_2$>}\ctsxrdef{@ifdim}
\explain
^^{dimensions//comparer}
Cette commande teste si \<dimen$_1$> et \<dimen$_2$>
satisfont \<relation>, qui doit \^etre soit `|<|', `|=|' ou `|>|'.
Les dimensions peuvent \^etre des constantes telles que |1in|, des registres de dimension
tels que |\dimen6| ou des para\-m\`etres de dimension tels que |\parindent|.
Avant d'ex\'ecuter le test, \TeX\ d\'eveloppe les tokens suivant le |\ifdim|
jusqu'a ce qu'il obtienne une s\'equence de tokens ayant la forme 
\<dimen$_1$> \<relation> \<dimen$_2$> 
suivi par un token qui ne peut \^etre une partie de \<dimen$_2$>.

\example
\dimen0 = 1000pt \ifdim \dimen0 > 3in true\else false\fi
|
\produces
\dimen0 = 1000pt \ifdim \dimen0 > 3in true\else false\fi
\endexample
\enddesc

\begindesc
\ctspecial ifhmode {}\ctsxrdef{@ifhmode}
\ctspecial ifvmode {}\ctsxrdef{@ifvmode}
\ctspecial ifmmode {}\ctsxrdef{@ifmmode}
\ctspecial ifinner {}\ctsxrdef{@ifinner}
\explain
^^{mode horizontal//tester}
^^{mode vertical//tester}
^^{mode math\'ematique//tester}
^^{mode restreint//tester}
Ces commandes testent dans quel \minref{mode} est \TeX~:
\ulist
\li |\ifhmode| est vrai si \TeX\ est en mode horizontal ordinaire ou r\'eduit.
\li |\ifvmode| est vrai si \TeX\ est en mode vertical ordinaire ou interne.
\li |\ifmmode| est vrai si \TeX\ est en mode math\'ematique de texte ou d'affichage.
\li |\ifinner| est vrai si \TeX\ est dans un mode ``interne''~:
horizontal r\'eduit, vertical interne ou math\'ematique de texte.
\endulist
\example
\def\modes{{\bf
   \ifhmode
      \ifinner IH\else H\fi
   \else\ifvmode
      \ifinner \hbox{IV}\else \hbox{V}\fi
   \else\ifmmode \hbox{M}\else
      error\fi\fi\fi}}
Formula $\modes$; then \modes, 
   \hbox{next \modes\ and \vbox{\modes}}.
\par\modes
|
\produces
\def\modes{{\bf
   \ifhmode
      \ifinner IH\else H\fi
   \else\ifvmode
      \ifinner \hbox{IV}\fi
   \else\ifmmode \hbox{M}\else
      error\fi\fi\fi}}
Formula $\modes$; then \modes, 
   \hbox{next \modes\ and \vbox{\modes}}.
\par\noindent{\bf V} % sorry folks, we have to fake this one
\endexample
\enddesc

\begindesc
\ctspecial ifhbox {\<registre>}\ctsxrdef{@ifhbox}
\ctspecial ifvbox {\<registre>}\ctsxrdef{@ifvbox}
\ctspecial ifvoid {\<registre>}\ctsxrdef{@ifvoid}
\explain
^^{hbox//tester}
^^{vbox//tester}
^^{\boites//tester si elle est vide}
Ces commandes testent le contenu du
registre de bo\^\i te num\'ero \<reg\-ister>.
Soit \<registre> \`a $n$.  Alors~:
\ulist
\li |\ifhbox| est vrai si |\box|$\,n$ est une \minref{hbox}.
\li |\ifvbox| est vrai si |\box|$\,n$ est une \minref{vbox}.
\li |\ifvoid| est vrai si |\box|$\,n$ est vide, c'est-\`a-dire, ne contient pas
de bo\^\i te.
\endulist
\example
\setbox0 = \vbox{} % empty but not void
\setbox1 = \hbox{a}
\setbox2 = \box1 % makes box1 void
\ifvbox0 true\else false\fi;
\ifhbox2 true\else false\fi;
\ifvoid1 true\else false\fi
|
\produces
\setbox0 = \vbox{}
\setbox1 = \hbox{a}
\setbox2 = \box1 % empties box1
\ifvbox0 true\else false\fi;
\ifhbox2 true\else false\fi;
\ifvoid1 true\else false\fi
\endexample
\enddesc

\begindesc
\ctspecial ifeof {\<nombre>}\ctsxrdef{@ifeof}
\explain
^^{fin de fichier, tester}
\minrefs{file}
Cette commande teste un flux d'entr\'ee de fin de fichier.
Elle est vraie si le flux d'entr\'ee \<nombre> n'a pas \'et\'e ouvert,
ou a \'et\'e ouvert et que le fichier associ\'e a \'et\'e enti\`erement lu
(ou n'existe pas).
\enddesc

\begindesc
\ctspecial ifcase
{\<nombre>\<texte du cas$_0$> {\bt \\or }\<texte du cas$_1$> {\bt \\or}
   $\ldots$ {\bt \\or} \<texte du cas$_n$>\hfil\break
\hglue 3pc{\bt \\else} \<texte alternatif> {\bt \\fi}}
\ctsxrdef{@ifcase}
\ctsxrdef{@or}
\explain
^^{test de cas}
Cette commande introduit un test avec des cas multiples num\'erot\'es.
Si \<nom\-bre> a la valeur $k$, \TeX\ d\'eveloppera \<texte de cas$_k$> s'il
existe et \<texte alternatif> autrement.  Vous pouvez omettre le |\else|---%
dans ce cas, \TeX\ ne d\'eveloppera rien si aucun des cas n'est satisfait.
\example
\def\whichday#1{\ifcase #1<day 0>\or Sunday\or Monday%
   \or Tuesday\or Wednesday\or Thursday\or Friday%
   \or Saturday\else Nonday\fi
   \ is day \##1. }
\whichday2 \whichday3 \whichday9
|
\produces
\def\whichday#1{\ifcase #1<day 0>\or Sunday\or Monday%
   \or Tuesday\or Wednesday\or Thursday\or Friday%
   \or Saturday\else Nonday\fi
   \ is day \##1. }
\whichday2 \whichday3 \whichday9
\endexample
\enddesc

\begindesc
\ctspecial iftrue {}\ctsxrdef{@iftrue}
\ctspecial iffalse {}\ctsxrdef{@iffalse}
\explain
Ces commandes sont \'equivalentes \`a des tests qui sont toujours vrai ou toujours
faux.  Le principal usage de ces commandes est de d\'efinir des macros qui gardent
trace du r\'esultat d'un test.
\example
\def\isbigger{\let\bigger=\iftrue}
\def\isnotbigger{\let\bigger=\iffalse}
% These \let's MUST be buried in macros!!  If they aren't,
% TeX erroneously tries to match them with \fi.
\def\test#1#2{\ifnum #1>#2 \isbigger\else\isnotbigger\fi}
\test{3}{6}
\bigger$3>6$\else$3\le6$\fi
|
\produces
\def\isbigger{\let\bigger=\iftrue}
\def\isnotbigger{\let\bigger=\iffalse}
% These \let's MUST be buried in macros!
% If they aren't, TeX erroneously tries to match them with \fi
\def\test#1#2{\ifnum #1>#2 \isbigger\else\isnotbigger\fi}
\test{3}{6}
\bigger$3>6$\else$3\le6$\fi
\endexample
\enddesc

\begindesc
\ctspecial else {} \ctsxrdef{@else}
\explain
Cette commande introduit l'alternative ``faux'' d'un test conditionel.
\enddesc

\begindesc
\ctspecial fi {} \ctsxrdef{@fi}
\explain
Cette commande termine le texte d'un test conditionel.
\enddesc

\begindesc
\ctspecial newif {{\bt \\if}\<nom de test>}\ctsxrdef{@newif}
\explain
Cette commande nomme un trio de s\'equence de contr\^ole avec comme noms |\alpha!-true|,
|\alphafalse|,
et |\ifalpha|, o\`u |alpha| est \<test name>.
Vous pouvez les utiliser pour d\'efinir vos propres tests en
cr\'eant une variable logique qui enregistre des informations
true\slash false:
\ulist\compact
\li |\alphatrue| met la variable logique |alpha| \`a vrai.
\li |\alphafalse| met la variable logique |alpha| \`a faux
\li |\ifalpha| est un test conditionnel qui est vrai si la variable logique
|alpha| est vraie et faux autrement.
\endulist
La variable logique |alpha| n'existe pas r\'eellement, mais \TeX\ agit comme
si elle l'\'etait.  Apr\`es |\newif\ifalpha|, la variable logique est initialis\'ee
\`a faux.

|\newif| est un commande externe, donc vous ne pouvez l'utiliser dans une d\'efinition
de macro.
\example
\newif\iflong  \longtrue
\iflong Rabbits have long ears.
\else Rabbits don't have long ears.\fi
|
\produces
\newif\iflong
\longtrue
\iflong Rabbits have long ears.\else Rabbits don't have long ears.\fi
\endexample
\eix^^{tests conditionnels}
\enddesc


%==========================================================================
\subsection {Actions r\'ep\'et\'ees}

{\def\test{{\bt \\if}$\Omega$}%
\begindesc
\bix^^{actions r\'ep\'et\'ees}
\bix^^{boucles}
\cts loop {$\alpha$ {\test} $\beta$ {\bt \\repeat}}
\ctspecial repeat {}\ctsxrdef{@repeat}
\explain
Ces commandes procurent une construction de boucle pour \TeX.
Ici, $\alpha$ et $\beta$ sont des s\'equences de commandes arbitraires
et \test\ est un des tests conditionnels d\'ecrites dans
\headcit{Tests conditionnels}{conds}.
Le |\repeat| r\'ep\`ete le |\fi| correspondant au test,
donc vous n'aurez pas besoin d'\'ecrire un |\fi| explicite pour terminer le test.
Ni, malheureusement, associer un |\else| au test.
Si vous voulez utiliser le test dans le sens oppos\'e, vous devez
r\'earranger le test ou
d\'efinir un test auxiliaire avec |\newif| (voir au-dessus) et utiliser ce
test dans le sens que vous voulez (voir le second exemple ci-dessous).

\TeX\ d\'eveloppe |\loop| comme suit~:
\olist
\li $\alpha$ est d\'evelopp\'e.
\li {\test} est ex\'ecut\'e.  Si le r\'esultat est faux, la boucle est termin\'ee.
\li $\beta$ est d\'evelopp\'e.
\li Le cycle est r\'ep\'et\'e.
\endolist
\example
\count255 = 6
\loop
   \number\count255\ 
   \ifnum\count255 > 0
      \advance\count255 by -1
\repeat
|
\produces
\count255 = 6
\loop
   \number\count255\ 
   \ifnum\count255 > 0
      \advance\count255 by -1
\repeat
\nextexample
\newif\ifnotdone % \newif uses \count255 in its definition
\count255=6
\loop
   \number\count255\ 
   \ifnum\count255 < 1 \notdonefalse\else\notdonetrue\fi
   \ifnotdone
      \advance\count255 by -1
\repeat
|
\produces
\newif\ifnotdone
\count255=6
\loop
   \number\count255\ 
   \ifnum\count255 < 1 \notdonefalse\else\notdonetrue\fi
   \ifnotdone
      \advance\count255 by -1
\repeat
%
\eix^^{actions r\'ep\'et\'ees}
\eix^^{boucles}
%
\endexample
\enddesc
} % end scope of definition of \test

%==========================================================================
\subsection {Ne rien faire}

\begindesc
\cts relax {}
\explain
Cette commande demande \`a \TeX\ de ne rien faire.  C'est utile dans un contexte o\`u
vous devez procurer une commande mais il n'y a rien que vous voulez que \TeX\ fasse.
\example
\def\medspace{\hskip 12pt\relax}
% The \relax guards against the possibility that
% The next tokens are `plus' or `minus'.
|

\endexample
\enddesc

\begindesc
\cts empty {}
\explain
Cette commande se d\'eveloppe en aucun tokens du tout.
Elle diff\`ere de |\relax| car elle dispara\^\i t apr\`es un d\'eveloppement de macro.
%
\xrdef{mac2}% end the section on macros
\eix^^{macros}
\enddesc

%==========================================================================
\section {Registres}

%==========================================================================
\subsection {Utiliser des registres}

\begindesc
\bix^^{registres}
\makecolumns 11/2:
\cts count {\<registre> {\bt =} \<nombre>}
\cts dimen {\<registre> {\bt =} \<dimen>}
\cts skip {\<registre> {\bt =} \<glue>}
\cts muskip {\<registre> {\bt =} \<muglue>}
\cts toks {\<registre> {\bt =} \<token variable>}
\aux\cts toks {\<registre> {\bt =} \rqbraces{\<list. de tok.>}}
\aux\cts count {\<registre>}
\aux\cts dimen {\<registre>}
\aux\cts skip {\<registre>}
\aux\cts muskip {\<registre>}
\aux\cts toks {\<registre>}
\explain
^^{assignements//de registres}
Les six premi\`eres commandes list\'ees ici assignent quelque chose \`a un registre.
Les |=| dans les assignations sont optionnels.
Les cinq s\'equences de contr\^ole restantes ne sont pas de vraies commandes
car elles n'apparais\-sent que comme partie d'un argument.
Elles pr\'ecisent le contenu des registres sp\'ecifi\'es.
Bien que vous ne puissiez utiliser ces s\'equences de contr\^ole elles-m\^eme comme commandes
dans un texte,  vous pouvez utili\-ser ^|\the| pour les convertir en texte 
et ainsi pouvoir composer leurs valeurs.

Vous pouvez nommer et r\'eserver de nouveaux registres
avec la commande ^|\newcount| et ses relatifs
(\xref{\@newcount}).
Utiliser ces commandes est un moyen s\^ur d'obtenir des registres qui 
ne sont pas cens\'es avoir d'usages conflictuels.

^^{registres count}
Un registre |\count| contient un entier, qui peut \^etre soit positif soit
n\'egatif.
Des entiers peuvent \^etre aussi grands que vous n'aurez probablement jamais besoin qu'ils soient.\footnote
{Voici le seul exercice du livre~: trouvez le plus grand entier
que \TeX\ acceptera.}
\TeX\ utilise les registres compteur $0$--$9$ pour garder trace des
num\'eros de page (voir la \knuth{page~119}{139}).
|\count255| est le seul registre compteur possible pour une utilisation
sans r\'eservation.
\example
\count255 = 17 \number\count255
|
\produces
\count255 = 17 \number\count255
\endexample

\medskip\noindent
^^{registre dimension}
Un registre |\dimen| contient une dimension.
Les registres de |\dimen0| \`a |\dimen255| sont possibles
pour un brouillon.

\example
\dimen0 = 2.5in
\hbox to \dimen0{$\Leftarrow$\hfil$\Rightarrow$}
|
\produces
\dimen0 = 2.5in
\hbox to \dimen0{$\Leftarrow$\hfil$\Rightarrow$}
\doruler{\8\8\8}3{in}
\endexample

\medskip\noindent
^^{registres skip}
Un registre |\skip| contient les dimensions de ressort.
Diff\'erent d'un registre |\dimen|, il
enregistre un montant d'\'etirement et de r\'etr\'ecissement autant qu'une taille naturelle.
Les registres de |\skip0| \`a |\skip9| et |\skip255| peuvent \^etre
utilis\'es sans r\'eservation.

\example
\skip2 = 2in
$\Rightarrow$\hskip \skip2 $\Leftarrow$
|
\produces
\skip2 = 2in
$\Rightarrow$\hskip \skip2 $\Leftarrow$\par
\noindent\hphantom{$\Rightarrow$}\ruler{\8\8}2{in}
\endexample

\medskip\noindent
^^{registres muskip}
Un registre |\muskip| est comme un registre |\skip|,
mais le ressort qu'il contient est toujours mesur\'ee en ^|mu|
\seeconcept{unit\'e math\'ematique}.
La taille d'un |mu| d\'epend de la police courante.
Par exemple, elle est normalement un peu
plus petite dans un exposant que dans un texte ordinaire.
Les registres de |\muskip0| \`a |\muskip9| et |\muskip255| peuvent \^etre
utilis\'es sans r\'eservation.

\example
\muskip0 = 24mu % An em and a half, no stretch or shrink.
$\mathop{a \mskip\muskip0 b}\limits^{a \mskip\muskip0 b}$
% Note the difference in spacing.
|
\produces
\muskip0 = 24mu % an em and a half
$\mathop{a \mskip\muskip0 b}\limits^{a \mskip\muskip0 b}$
% Note the difference in spacing
\endexample

\medskip\noindent
^^{registres token}
Vous pouvez assigner soit un token variable
(un registre ou un param\`etre) ou une liste de token
\`a un registre |\toks|.
Quand vous assignez une liste de token \`a un registre de token,
les tokens de la liste de token \emph{ne} sont \emph{pas} d\'evelopp\'es.

Une fois que les tokens d'une liste de token ont \'et\'e ins\'er\'es dans un texte
en utilisant ^|\the|, ils sont
d\'evelopp\'es comme des tokens qui seraient lus directement.
Ils ont les codes de cat\'egorie qu'ils ont re\c cus quand \TeX\ les \`a vu
une premi\`ere fois dans la~\hbox{source}.

\example
\toks0 = {the \oystereaters\ were at the seashore}
% This assignment doesn't expand \oystereaters.
\def\oystereaters{Walrus and Carpenter}
\toks1 = \toks0
% the same tokens are now in \toks0 and \toks1
Alice inquired as to whether \the\toks1.
|
\produces
\toks0 = {the \oystereaters\ were at the seashore}
% This assignment doesn't expand \oystereaters
\def\oystereaters{Walrus and Carpenter}
\toks1 = \toks0
% the same tokens are now in \toks0 and \toks1
Alice inquired as to whether \the\toks1.
\endexample
\enddesc

\begindesc
\cts maxdimen {}
\explain
^^{dimensions//maximum}
Cette s\'equence de contr\^ole d\'esigne une \<dimen> qui est la
plus grande dimension acceptable par \TeX\ (environ 4,50 m\`etres).
Ce n'est pas une vraie commande parce qu'elle ne peut appara\^\i tre 
que comme partie d'un argument d'une autre commande.
\example
\maxdepth = \maxdimen % Remove restrictions on \maxdepth.
|
\endexample
\enddesc

\see |\advance| (\xref\advance), |\multiply|,
|\divide| (\xref\divide), |\set!-box|, |\box| (\xref\box).

%==========================================================================
\subsection {Nommer et r\'eserver des registres, etc.}

\begindesc
\bix^^{registres//r\'eserver}
\makecolumns 11/2:
\ctspecial newcount \ctsxrdef{@newcount}
\ctspecial newdimen \ctsxrdef{@newdimen}
\ctspecial newskip \ctsxrdef{@newskip}
\ctspecial newmuskip \ctsxrdef{@newmuskip}
\ctspecial newtoks \ctsxrdef{@newtoks}
\ctspecial newbox \ctsxrdef{@newbox}
\ctspecial newread \ctsxrdef{@newread}
\ctspecial newwrite \ctsxrdef{@newwrite}
\ctspecial newfam \ctsxrdef{@newfam}
\ctspecial newinsert \ctsxrdef{@newinsert}
\ctspecial newlanguage \ctsxrdef{@newlanguage}
\explain
Ces commandes
r\'eservent et nomment une entit\'e du type indiqu\'e~:
\ulist
{\tolerance = 5000
\fontdimen3\the\font = 2\fontdimen3\the\font % allow spaces to stretch more
\li |\new!-count|, |\newdimen|, |\newskip|, |\new!-mu!-skip|, |\new!-toks|,
et |\new!-box| r\'eservent chacun un \minref{registre} du type indiqu\'e.
}
^^{registres count//r\'eserv\'es par \b\tt\\newcount\e}
^^{registres de dimension//r\'eserv\'es par \b\tt\\newdimen\e}
^^{registres skip//r\'eserv\'es par \b\tt\\newskip\e}
^^{registres muskip//r\'eserv\'es par \b\tt\\newmuskip\e}
^^{registres token//r\'eserv\'es par \b\tt\\newtoks\e}
^^{registres de \boite//r\'eserv\'es par \b\tt\\newbox\e}
\li |\newread| et |\newwrite| r\'eservent un flot d'entr\'ee et
de sortie \minrefs{input stream}\minrefs{output stream}
respectivement.
^^{flots d'entr\'ee//r\'eserv\'es par \b\tt\\newread\e}
^^{flots de sortie//r\'eserv\'es par \b\tt\\newwrite\e}
\li |\newfam| r\'eserve une \minref{famille} de police math\'ematique.
^^{famille//r\'eserv\'ee par \b\tt\\newfam\e}
\li |\newinsert| r\'eserve un caract\`ere d'insertion.
(R\'eserver un caract\`ere d'insertion entra\^\i ne r\'eserver plusieurs registres diff\'erents.)
^^{insertions//num\'eros r\'eserv\'es par \b\tt\\newinsert\e}
\li |\newlanguage| r\'eserve un jeu de motifs de c\'esure.
\endulist
Vous devez utiliser ces commandes \`a chaque fois que vous avez besoin d'une
de ces entit\'es, dans une r\'egion tr\`es locale,
pour \'eviter des conflits de num\'erotation.

Il y a une diff\'erence importante entre ces commandes~:
\ulist
\li Les s\'equences de contr\^ole d\'efinies par
|\newcount|, |\newdimen|, |\newskip|, |\newmuskip| et |\newtoks|
d\'esignent chacune une entit\'e du type appropri\'e.
Par exemple, apr\`es la commande~:
\csdisplay
\newdimen\listdimen
|
la s\'equence de contr\^ole |\listdimen| peut \^etre utilis\'ee comme une dimension.
\li Les s\'equences de contr\^ole d\'efinies par
|\newbox|, |\newread|, |\new!-write|, |\newfam|, |\newinsert|
et |\newlanguage|  \'evaluent chacune
le \emph{num\'ero} d'une entit\'e du type appropri\'e.
Par exemple, apr\`es la commande~:
\csdisplay
\newbox\figbox
|
La s\'equence de contr\^ole |\figbox| doit \^etre utilis\'ee en conjonction avec
une commande de type |\box|, par exemple~:
\csdisplay
\setbox\figbox = \vbox{!dots}
|
\endulist
\enddesc

\begindesc
\cts countdef {\<s\'equence de contr\^ole> {\bt =} \<registre>}
\cts dimendef {\<s\'equence de contr\^ole> {\bt =} \<registre>}
\cts skipdef {\<s\'equence de contr\^ole> {\bt =} \<registre>}
\cts muskipdef {\<s\'equence de contr\^ole> {\bt =} \<registre>}
\cts toksdef {\<s\'equence de contr\^ole> {\bt =} \<registre>}
\explain
Ces commandes d\'efinissent \<s\'equence de contr\^ole> pour faire r\'ef\'erence au 
\minref{registre} de la cat\'egorie indiqu\'ee dont le num\'ero est \<registre>.
Normalement vous devez utiliser les commandes du groupe pr\'ec\'edent
(|\newcount|, etc.) en pr\'ef\'erence \`a ces commandes pour \'eviter
des conflits de num\'ero.  Les commandes du groupe pr\'ec\'edent sont
d\'efinies en termes de commandes \hbox{de ce groupe}.
\example
\countdef\hatters = 19 % \hatters now refers to \count19
\toksdef\hares = 200 % \hares now refers to \toks200
|
\endexample
\enddesc

\see |\newif| (\xref{\@newif}), |\newhelp| (\xref{\@newhelp}).
\eix^^{registres//r\'eserver}

%==========================================================================
\subsection {Faire de l'arithm\'etique dans des registres}

\begindesc
\bix^^{arithm\'etique}
\bix^^{registres//arithm\'etique dans des}
%
\cts advance {\<registre compteur> {\bt by} \<nombre>}
\aux\cts advance {\<registre dimension> {\bt by} \<dimension>}
\aux\cts advance {\<registre saut> {\bt by} \<ressort>}
\aux\cts advance {\<registre muskip> {\bt by} \<muglue>}
\explain
Ces commandes ajoutent une quantit\'e compatible a un registre.  Pour \<ressort>
ou \<muglue> les trois composants (valeur naturelle, \'etirement et r\'etr\'ecissement)
\minrefs{glue} sont ajout\'es.
Toutes les quantit\'es peuvent \^etre n\'egatives.  
Pour ces calculs
(et autres assignements en fait), \<ressort> peut \^etre converti en une
\<dimension> en enlevant l'\'etirement et le r\'etr\'ecissement et une \<dimension> 
peut \^etre convertie en \<nombre> en prenant sa valeur en points d'\'echelle 
\seeconcept{dimension}.
Vous pouvez omettre le mot |by| dans ces commandes---\TeX\ les comprendra quand
m\^eme.
\example
\count0 = 18 \advance\count0 by -1 \number\count0\par
\skip0 = .5in \advance\skip0 by 0in plus 1in % add stretch
\hbox to 2in{a\hskip\skip0 b}
|
\produces
\count0 = 18 \advance\count0 by -1 \number\count0\par
\skip0 = .5in \advance\skip0 by 0in plus 1in % add stretch
\hbox to 2in{a\hskip\skip0 b}
\doruler{\8\8}2{in}
\endexample
\enddesc

\begindesc
\cts multiply {\<registre> {\bt by} \<nombre>}
\cts divide {\<registre> {\bt by} \<nombre>}
\explain
Ces commandes multiplient et divisent la valeur de \<registre>
par \<nombre> (qui peut \^etre n\'egatif).
Le registre peut \^etre un registre ^|\count|, ^|\dimen|, ^|\skip| 
ou ^|\muskip|.
Pour un registre ^|\skip| ou ^|\muskip| (\xref\skip),
les trois composantes du \minref{ressort} du registre sont modifi\'es.
Vous pouvez omettre le mot |by| dans ces commandes---\TeX\ les comprendra quand
m\^eme.

Vous pouvez aussi obtenir un multiple d'une \<dimension> en la pr\'ec\'edant par un \<nombre>
\minrefs{number}
ou une constante d\'ecimale, par exemple,
|-2.5\dimen2|.
Vous pouvez aussi utiliser cette notation pour des \<ressort>, mais attention---le r\'esultat
est une \<dimension>, pas un \<ressort>.
Ainsi |2\baselineskip| donne une \<dimension> qui est deux fois la taille naturelle
de |\baselineskip|, sans \'etirement ni r\'etr\'ecissement.
\example
\count0 = 9\multiply \count0 by 8 \number\count0 ;
\divide \count0 by 12 \number\count0 \par
\skip0 = 20pt plus 2pt minus 3pt \multiply \skip0 by 3
Multiplied value of skip0 is \the\skip0.\par
\dimen0 = .5in \multiply\dimen0 by 6
\hbox to \dimen0{a\hfil b}
|
\produces
\count0 = 9\multiply \count0 by 8 \number\count0 ;
\divide \count0 by 12 \number\count0 \par
\skip0 = 20pt plus 2pt minus 3pt \multiply \skip0 by 3
Multiplied value of skip0 is \the\skip0.\par
\dimen0 = .5in \multiply\dimen0 by 6
\hbox to \dimen0{a\hfil b}
\doruler{\8\8\8}3{in}
\endexample

\eix^^{arithm\'etique}
\eix^^{registres//arithm\'etique dans des}
\eix^^{registres}
\enddesc

%==========================================================================
\section {Terminer l'ex\'ecution}

\begindesc
^^{fin d'ex\'ecution}
\easy\ctspecial bye \ctsxrdef{@bye}
\explain
Cette commande dit \`a \TeX\ de remplir et produire la derni\`ere page, d'imprimer
toutes les \minref{insertion}s en suspens et de terminer l'ex\'ecution.
C'est le moyen usuel de terminer votre fichier source.
\enddesc

\begindesc
\cts end {}
\explain
Cette commande dit \`a \TeX\ de produire la derni\`ere page et de terminer l'ex\'ecution.
Elle ne remplit pas la page,  
donc il est pr\'ef\'erable d'utiliser |\bye| plut\^ot que |\end|.
\enddesc

%==========================================================================
\section {Entr\'ee et sortie}

%==========================================================================
\subsection {Op\'erations sur des fichiers d'entr\'ee}

\begindesc
\bix^^{fichiers}
\bix^^{fichiers d'entr\'ee}
\easy\cts input {\<nom de fichier>}
\explain
\minrefs{file}\minrefs{file name}
Cette commande demande \`a \TeX\ de lire son entr\'ee \`a partir du fichier \<nom de fichier>.
Quand ce fichier est \'epuis\'e, \TeX\ retourne lire \`a partir du source d'entr\'ee
pr\'ec\'edent.  Vous pouvez inclure des fichiers d'entr\'ee sur autant de niveaux que vous voulez
(dans les limites du raisonnable).

Quand vous saisissez un grand document, c'est souvent une bonne id\'ee de
structurer votre fichier principal comme une suite de commandes |\input| faisant r\'ef\'erence
aux parties subsidiaires du document.  De cette fa\c con, vous pouvez ex\'ecuter les
parties individuelles facilement quand vous travaillez sur des \'epreuves.  C'est aussi une bonne
pratique de mettre toutes vos d\'efinitions de \minref{macro} dans un fichier s\'epar\'e et
appeler ce fichier avec une commande |\input| comme premi\`ere action de votre
fichier principal.

\TeX\ utilise des r\`egles diff\'erentes pour balayer les noms de fichier que 
pour lire les \minref{token}s en g\'en\'eral (voir \xref{nom de fichier}).
Si votre impl\'ementation attend des noms de fichier avec extensions (habituellement
pr\'ec\'ed\'ees par un point), alors \TeX\ procure l'extension par d\'efaut |.tex|.
\example
\input macros.tex
\input chap1 % equivalent to chap1.tex
|
\endexample
\enddesc

\begindesc
\cts endinput {}
\explain
Cette commande dit \`a \TeX\ d'arr\^eter de lire l'entr\'ee du fichier courant 
quand il rencontrera la prochaine fin de ligne.
\enddesc

\begindesc
\cts inputlineno {}
\explain
Cette commande donne un nombre (pas une cha\^\i ne) donnant le num\'ero de ligne 
de la ligne courante, devant \^etre le num\'ero qui appara\^\i trait dans un message d'erreur 
si une erreur arrive \`a cette endroit.
\enddesc

\begindesc
\cts openin {\<nombre> {\bt =} \<nom de fichier>}
\explain
Cette commande demande \`a \TeX\ d'ouvrir le fichier nomm\'e \<nom de fichier>
et de le rendre lisible via le flot d'entr\'ee d\'esign\'e par \<nombre>.
^^{flots d'entr\'ee//ouverture}
\<nombre> doit \^etre entre $0$ et $15$.
Une fois que vous avez ouvert un fichier et l'avez connect\'e \`a un flot d'entr\'ee,
vous pouvez lire le fichier en utilisant la commande |\read| 
avec le num\'ero du flot d'entr\'ee.  

vous pouvez associer plus d'un flot d'entr\'ee avec le m\^eme
fichier.  vous pouvez alors lire \`a plusieurs positions diff\'erentes dans
le fichier, une pour chaque flot d'entr\'ee.

Vous devez allouer les num\'eros de flots pour |\openin| en utilisant
|\new!-read| (\xref{\@newread}).
\example
\newread\auxfile  \openin\auxfile = addenda.aux
% \auxfile now denotes the number of this opening
% of addenda.aux.
|
\endexample
\enddesc

\begindesc%\secondprinting{\vglue-.5\baselineskip\vskip0pt}
\cts closein {\<nombre>}
\explain
Cette commande demande \`a \TeX\ de fermer le \minref{flot d'entr\'ee} num\'erot\'e
\<nombre>, c'est-\`a-dire,
clore l'association entre le flot d'entr\'ee et son fichier.
Le flot d'entr\'ee avec ce num\'ero devient alors disponible pour \^etre utilis\'e avec un
fichier diff\'erent.
Vous devez fermer un flot d'entr\'ee une fois que vous avez fini d'utiliser son fichier.
\example
\closein\auxfile
|
\endexample
\enddesc

\begindesc%\secondprinting{\vglue-.5\baselineskip\vskip0pt}
\cts read {\<nombre> {\bt to} \<s\'equence de contr\^ole>}
\explain
^^{flots d'entr\'ee//lecture avec \b\tt\\read\e}
^^{lire un fichier}
Cette commande demande \`a \TeX\ de lire une ligne du fichier
associ\'e avec le \minref{flot d'entr\'ee}
d\'esign\'e par \<nombre> et assigne les tokens de cette ligne \`a
\<s\'equence de contr\^ole>.  La \minref{s\'equence de contr\^ole} devient alors une 
\minref{macro} sans param\`etre.  Aucun d\'eveloppement de macro ne prend place
durant l'op\'eration de lecture.  Si la ligne contient des accolades ouvrantes
non appair\'ees, \TeX\ lira des lignes suppl\'ementaires jusqu'a ce que les accolades
soient toutes referm\'ees.  Si \TeX\ atteint la fin du fichier sans fermer toutes les
accolades, il se plaindra.

Si \<nombre> est plus grand que $15$ ou n'a pas \'et\'e associ\'e avec un fichier
en utilisant ^|\openin|, \TeX\ affichera l'incitation `\<s\'equence de contr\^ole> |=|'
sur votre terminal et attendra que vous saisissiez une ligne d'entr\'ee.
Il assignera alors la ligne d'entr\'ee \`a \<s\'equence de contr\^ole>.
Si \<nombre> est inf\'erieur \`a z\'ero, il lira une ligne d'entr\'ee de votre
terminal mais omettra l'incitation.
\example
\read\auxfile to \holder
% Expanding \holder will produce the line just read.
|
\endexample
\eix^^{fichiers d'entr\'ee}
\enddesc

%\secondprinting{\vfill\eject}


%==========================================================================
\subsection {Op\'erations sur des fichiers de sortie}

\bix^^{fichiers de sortie}
\begindesc
\cts openout {\<nombre> {\bt =} \<nom de fichier>}
\explain
^^{flots de sortie//ouverture}
Cette commande demande \`a \TeX\ d'ouvrir le fichier nomm\'e \<nom de fichier>
et de le rendre disponible pour l'\'ecriture  via le \minref{flux de sortie}
d\'esign\'e par \<nombre>.
\<nombre> doit \^etre entre $0$ et $15$.
Une fois que vous avez ouvert un fichier et l'avez connect\'e \`a un flot de sortie,
vous pouvez \'ecrire dans le fichier en utilisant la commande |\write| 
avec le num\'ero du flot de sortie. 

Un |\openout| g\'en\`ere un \'el\'ement extraordinaire qui devient une partie d'une bo\^\i te.
Le |\openout| ne prend pas effet avant que \TeX\ n'envoit cette bo\^\i te
dans le \dvifile,
\`a moins que vous ayez fait pr\'ec\'eder |\openout| de ^|\immediate|.

\TeX\ ne se plaindra pas si vous associez plus d'un flux de sortie avec le
m\^eme fichier, mais vous aurez un sacr\'e bazar dans le fichier si vous essayez~!

Vous devez allouer des num\'eros de flux pour |\openout| en utilisant
|\newwrite| (\xref{\@newwrite}).
\example
\newwrite\auxfile  \openout\auxfile = addenda.aux
% \auxfile now denotes the number of this opening
% of addenda.aux.
|
\endexample
\enddesc

\begindesc
\cts closeout {\<nombre>}
\explain
^^{flots de sortie//fermeture}
Cette commande demande \`a \TeX\ de fermer le \minref{flot de sortie} num\'erot\'e
\<nombre>. c'est-\`a-dire,
cesser l'association entre le flot de sortie et son fichier.
Le flot de sortie avec ce num\'ero devient alors disponible pour \^etre utilis\'e avec 
un fichier diff\'erent.
Vous devez fermer un flot de sortie une fois que vous avez fini d'utiliser son fichier.

Un |\closeout| g\'en\`ere un \'el\'ement extraordinaire qui devient une partie d'une bo\^\i te.
Le |\closeout| ne prend pas effet avant que \TeX\ n'envoit pas cette bo\^\i te
dans le \dvifile,
\`a moins que vous ayez fait pr\'ec\'eder |\closeout| de ^|\immediate|.
\example
\closeout\auxfile
|
\endexample
\enddesc

\begindesc
\cts write {\<nombre> \rqbraces{\<liste de token>}}
\explain
^^{flots de sortie//\'ecriture}
^^{\ecrire\ un fichier}
Cette commande demande \`a \TeX\ d'\'ecrire \<liste de token> dans le fichier
associ\'e avec le \minref{flot de sortie}
d\'esign\'e par \<nombre>.
Il g\'en\`ere un \'el\'ement extraordinaire qui devient une partie d'une bo\^\i te.
L'\'ecriture actuelle ne prend pas de place tant que \TeX\ n'envoie cette bo\^\i te
dans le \dvifile,
\`a moins que vous pr\'ec\'ediez le |\write| par ^|\immediate|.

Pour un |\write| qui n'est pas imm\'ediat, \TeX\ ne d\'eveloppe pas de macros de
\<liste de token> tant que la liste de token n'est pas \'ecrite dans le fichier.
Les d\'eveloppements de macro suivent les m\^emes r\`egles que |\edef| (\xref\edef).
En particulier, toute s\'equence de contr\^ole qui n'est pas le  
nom d'une macro est \'ecrite comme un
^|\escapechar| suivi par le nom de la s\'equence de contr\^ole 
et un espace.  tous les tokens `|#|'
de \<liste de token> sont doubl\'es, c'est-\`a-dire, \'ecrits comme `|##|'.

Si \<nombre> n'est pas dans la fourchette de $0$ \`a $15$, \TeX\ \'ecrit
\<liste de token> dans le fichier log.
^^{fichier log//\'ecrit par \b\tt\\write\e}
Si \<nombre> est sup\'erieur \`a $15$ ou n'est pas associ\'e avec un flot de
sortie, \TeX\ \'ecrit aussi \<liste de token> sur le terminal.
\example
\def\aa{a a}
\write\auxfile{\hbox{$x#y$} \aa}
% Writes the string `\hbox {$x##y$} a a' to \auxfile.
|
\endexample
\enddesc

\begindesc
\cts immediate {}
\explain
Cette commande doit pr\'ec\'eder un |\openout|, un |\closeout| ou un |\write|.
^^|\write//avec {\tt\\immediate}|
^^|\openout//avec {\tt\\immediate}|
^^|\closeout//avec {\tt\\immediate}|
Elle demande \`a \TeX\ d'ex\'ecuter l'op\'eration de fichier sp\'ecifi\'ee sans d\'elai.
\example
\immediate\write 16{I'm stuck!!}
% has the same effect as \message
|
\endexample\enddesc

\begindesc
\cts special {\rqbraces{\<liste de token>}}
\explain
Cette commande demande \`a \TeX\  
d'\'ecrire \<liste de token> directement dans le \dvifile\ quand il enverra
une page.
Une utilisation typique de |\special| sera de dire au driver d'impression d'incorporer
le contenu d'un fichier graphique nomm\'e dans la page de sortie.
^^{drivers de p\'eriph\'erique//instructions par \b\tt\\special\e}
La commande |\special| produit un \'el\'ement extraordinaire qui associe
\<liste de token> avec une position particuli\`ere sur la page, plus pr\'ecis\'ement,
la position qu'une bo\^\i te de taille nulle aurait eu si un telle bo\^\i te
serait apparue \`a la place de la commande |\special|.
Toute utilisation que vous devez faire de |\special| d\'epend strictement des
^{drivers d'impression} que vous avez \`a votre disposition.
\example
\special{graphic expic} 
% Display the graphics file `expic' here.
|
\endexample
\enddesc

\begindesc
\cts newlinechar {\param{nombre}}
\explain
Ce param\`etre contient un caract\`ere qui indique une nouvelle ligne sur la
sortie.  Quand \TeX\ rencontre ce caract\`ere
en lisant l'argument d'une commande
|\write|, |\message| ou
|\errmessage|, il d\'ebute une nouvelle ligne.
Si |\newlinechar| n'est pas dans la fourchette $0$--$255$,
Il n'y a aucun caract\`ere indiquant
une nouvelle ligne en sortie.
\PlainTeX\ met |\newlinechar| \`a $-1$.
\example
\newlinechar = `\^^J
\message{This message appears^^Jon two lines.}
|
\logproduces
This message appears
on two lines.
|
\endexample
\enddesc

\see |\newread|, |\newwrite| (\xref{\@newwrite}).
\eix^^{fichiers}
\eix^^{fichiers de sortie}

%==========================================================================
\subsection {Interpr\'eter des caract\`eres entr\'es}

\begindesc
\cts catcode {\<charcode> \tblentry{nombre}}
\explain
^^{codes de cat\'egorie//en table de \b\tt\\catcode\e}
Cette entr\'ee de table contient le \minref{code de cat\'egorie} du caract\`ere
dont le code \ascii\ est \<charcode>.
Les codes de cat\'egorie sont list\'es dans la \xrefpg{catcodes}.
En changeant le code de cat\'egorie d'un caract\`ere vous pouvez demande \`a \TeX\ de traiter
ce caract\`ere diff\'eremment.
\example
\catcode `\[ = 1 \catcode `\] = 2
% Make [ and ] act like left and right braces.
|
\endexample
\enddesc

\begindesc
\cts active {}
\explain
Cette commande contient le
code de cat\'egorie pour un caract\`ere actif, soit, le nombre $13$.
\example
\catcode `\@ = \active % Make @ an active character.
|
\endexample
\enddesc

\begindesc
\cts mathcode {\<charcode> \tblentry{nombre}}
\explain
Cette entr\'ee de table contient le \minref{mathcode} du caract\`ere
dont le code \ascii\ est \<charcode> \seeconcept{mathcode}.
Le mathcode sp\'ecifie l'interpr\'etation du caract\`ere en mode math\'ematique.
\example
\mathcode\> = "313E % as in plain TeX
% The > character has class 3 (relation), family 1 (math 
% italic), and character code "3E
|
\endexample
\enddesc

\begindesc
\margin{{\tt\\delcode} was explained in two places.  This explanation
combines them.  (The other place was in the math section.)}
\cts delcode {\<charcode>\tblentry{number}}
\explain
^^{codes d\'elimiteurs}
Cette entr\'ee de table sp\'ecifie le code \minref{delimiter} pour le caract\`ere entr\'ee
dont le code \ascii\ est \<charcode>.
Le code d\'elimiteur dit \`a \TeX\ comment trouver le meilleur caract\`ere de sortie utiliser
pour composer le caract\`ere d'entr\'ee indiqu\'e comme d\'elimiteur.

\<nombre> est normalement \'ecrit en notation hexad\'ecimale.
Supposez que \<nombre> soit le nombre hexad\'ecimal $s_1s_2s_3\,
l_1l_2l_3$.  Alors, quand le caract\`ere est utilis\'e comme un d\'elimiteur,
\TeX\ prend le caract\`ere ayant une petite variante
$s_1s_2s_3$ et une grande variante $l_1l_2l_3$.  Ici $s_1s_2s_3$ indique
le caract\`ere math\'ematique trouv\'e en position $s_2s_3$ de la famille $s_1$ et
de m\^eme pour $l_1l_2l_3$.  C'est la m\^eme convention que celle
utilis\'ee pour ^|\mathcode| (\xref \mathcode),
sauf que |\mathcode| sp\'ecifie aussi une classe.
\example
\delcode `( = "028300  % As in plain TeX.
|
\endexample
\enddesc

\begindesc
\cts endlinechar {\param{nombre}}
\explain
Ce param\`etre
contient le code de caract\`ere pour le  caract\`ere que
\TeX\ attend \`a la fin de chaque ligne d'entr\'ee.
^^{ligne d'entr\'ee}
Une valeur qui ne serait pas dans le fourchette $0$--$255$
indique qu'aucun caract\`ere ne doit \^etre attendu.
\PlainTeX\ laisse |\end!-line!-char| \`a
|`\^^M| (le code {\ascii} pour \asciichar{return}).
\enddesc

\begindesc
\cts ignorespaces {}
\explain
Cette commande demande \`a \TeX\ de lire et de d\'evelopper des tokens jusqu'\`a ce qu'il en trouve 
un qui ne soit pas un \minref{token} espace, ignorant tous les tokens espace 
qu'il trouve autrement.
|\ignorespaces| est souvent pratique \`a la fin d'une \minref{macro} comme un moyen
de rendre la macro insensible \`a tout espace ou fin de ligne
qui peut suivre son propre appel.
(Une ligne vide apr\`es |\ignorespaces| continue \`a produire un token |\par|,
de tout fa\c con.)
\example
\def\aa#1{yes #1\ignorespaces}
\aa{may}
be
|
\produces
\def\aa#1{yes #1\ignorespaces}
\aa{may}
be
\endexample
\enddesc

%==========================================================================
\section {Contr\^oler l'interaction avec \TeX}

\begindesc
\bix^^{contr\^oler \TeX}
\bix^^{ex\'ecuter \TeX}
\cts errorstopmode {}
\explain
Cette commande demande \`a \TeX\
de stopper pour une interaction \`a chaque fois qu'il trouve une erreur.
C'est le mode op\'eratoire normal.
\enddesc

\begindesc
\cts scrollmode {}
\explain
Cette commande demande \`a \TeX\
de ne pas stopper pour le plupart des erreurs,
mais de continuer \`a afficher les messages d'erreur sur votre terminal.
Saisir `|S|' ou `|s|' en r\'eponse \`a un message d'erreur vous met
en mode scroll.
\enddesc

\begindesc
\cts nonstopmode {}
\explain
Cette commande demande \`a \TeX\
de ne pas stopper pour des erreurs, m\^eme celles r\'eclamant des fichiers qu'il
ne peut trouver, mais de continuer \`a afficher les messages d'erreur sur votre terminal.
Saisir `|R|' ou `|r|' en r\'eponse \`a un message d'erreur vous met
en mode nonstop.
\enddesc

\begindesc
\cts batchmode {}
\explain
Cette commande demande \`a \TeX\
de ne pas stopper pour des erreurs et de supprimer toute sortie sur votre terminal.
Saisir `|Q|' ou `|q|' en r\'eponse \`a un message d'erreur vous met
en mode batch.
\enddesc

\begindesc
\cts pausing {\param{nombre}}
\explain
Si ce param\`etre est sup\'erieur \`a z\'ero, \TeX\ s'arr\^etera
\`a chaque ligne d'entr\'ee pour vous donner une opportunit\'e de 
la remplacer par une ligne diff\'erente.  Si vous saisissez une modification,
\TeX\ utilisera cette ligne au lieu de l'originale~; si vous r\'epondez
par \asciichar{return}, \TeX\ utilisera l'originale.

Mettre |\pausing| \`a $1$ peut \^etre pratique comme moyen de patcher un document quand 
\TeX\ l'ex\'ecute.  Par exemple,
vous pouvez utiliser cette facilit\'e pour ins\'erer des commandes ^|\show| (voir plus loin).
\eix^^{ex\'ecuter \TeX}
\eix^^{contr\^oler \TeX}
\enddesc

%==========================================================================
\section {Aide au diagnostic}

\subsection{Afficher des donn\'ees internes}

\begindesc
\bix^^{tracer}
\bix^^{debugger}
\bix^^{aides au diagnostique}
\cts show {\<token>}
\cts showthe {\<argument>}
\cts showbox {\<nombre>}
\cts showlists {}
\explain
Ces commandes enregistrent de l'information dans le fichier log de votre ex\'ecution \TeX~:
\ulist
\li |\show| enregistre la signification de \<token>.\minrefs{token}
^^{tokens//affich\'es par \b\tt\\show\e}
\li |\showthe| enregistre 
 quels tokens seront produit par 
|\the| \<arg\-u\-ment> (voir \xref \the).
\li |\showbox| enregistre le contenu du \minref{registre} de la \minref{bo\^\i te} 
num\'erot\'e \<nom\-bre>.
Le nombre de
points affich\'es dans la log indique le nombre de
niveaux d'inclusion de bo\^\i tes internes.
\li |\showlists| enregistre
le contenu de chaque liste que \TeX\ est en train de construire.
(les listes sont empil\'ees les unes sur les autres.)
Voir les \knuth{pages~88--89}{103} pour plus d'information sur l'interpr\'etation
de la sortie de |\showlists|.
\endulist
Pour |\show| et |\showthe|, \TeX\ affiche aussi l'information sur votre
^{terminal}.
Pour |\showbox| et |\showlists|, \TeX\ n'affiche l'information sur votre
terminal que
si ^|\tracingonline| (\xref \tracingonline) est sup\'erieur \`a z\'ero~;
si ^|\tracingonline| est \`a z\'ero ou moins (le cas par d\'efaut),
l'information n'est pas affich\'ee.

\`A chaque fois que \TeX\ rencontre une commande de type |\show| il
stoppe pour un interaction.  La requ\^ete d'interaction ne doit \emph{pas}
indiquer une erreur, mais doit vous donner l'opportunit\'e de demander \`a \TeX\ de
vous montrer autre chose.  Si vous ne voulez pas voir autre chose, pressez juste
\asciichar{return}.

Vous pouvez contr\^oler la quantit\'e de sortie produite par |\showbox| en mettant
|\show!-box!-breadth| et |\show!-box!-depth| (\xref\showboxbreadth).
^^|\showboxbreadth| ^^|\showboxdepth|
Ces param\`etres ont des valeurs par d\'efaut \`a $5$ et $3$ respectivement, 
C'est pourquoi
juste cinq \'el\'ements apparaissent pour chaque bo\^\i te d\'ecrite
dans la sortie de log ci-dessous.  (Le `|..etc.|' indique des \'el\'ements additionnels
des bo\^\i tes qui ne sont pas affich\'es.)
\example
\show a
\show \hbox
\show \medskip
\show &
|
\logproduces
> the letter a.
> \hbox=\hbox.
> \medskip=macro:
->\vskip \medskipamount .
> alignment tab character &.
|
\nextexample
\showthe\medskipamount
\toks27={\hbox{Joe's\quad\ Diner}}
\showthe\toks27
|
\logproduces
> 6.0pt plus 2.0pt minus 2.0pt.
> \hbox {Joe's\quad \ Diner}.
|
\nextexample
\setbox 3=\vbox{\hbox{A red dog.}\hrule A black cat.}
\showbox 3
|
\logproduces
> \box3=
\vbox(16.23332+0.0)x53.05565
.\hbox(6.94444+1.94444)x46.41675
..\tenrm A
..\glue 3.33333 plus 1.66498 minus 1.11221
..\tenrm r
..\tenrm e
..\tenrm d
..etc.
.\rule(0.4+0.0)x*
.\hbox(6.94444+0.0)x53.05565
..\tenrm A
..\glue 3.33333 plus 1.66498 minus 1.11221
..\tenrm b
..\tenrm l
..\tenrm a
..etc.
|
\endexample
\vfil\eject
\example
\vbox{A \hbox
   {formula 
       $x \over y\showlists$}}
|
\logproduces
### math mode entered at line 3
\mathord
.\fam1 y
this will be denominator of:
\fraction, thickness = default
\\mathord
\.\fam1 x
### restricted horizontal mode entered at line 2
\tenrm f
\tenrm o
\tenrm r
\tenrm m
\kern-0.27779
\tenrm u
\tenrm l
\tenrm a
\glue 3.33333 plus 1.66666 minus 1.11111
spacefactor 1000
### horizontal mode entered at line 1
\hbox(0.0+0.0)x20.0
\tenrm A
\glue 3.33333 plus 1.66498 minus 1.11221
spacefactor 999
### internal vertical mode entered at line 1
prevdepth ignored
### vertical mode entered at line 0
prevdepth ignored
|
\endexample
\enddesc

\see |\showboxbreadth|, |\showboxdepth| \ctsref\showboxbreadth.

\subsection{Sp\'ecifier ce qui est trac\'e}

\begindesc
\cts tracingonline {\param{nombre}}
\explain
Si ce param\`etre est sup\'erieur \`a z\'ero,
\TeX\ affichera les r\'esultats de la trace
(en incluant ^|\showbox| et ^|\showlists|)
sur votre terminal en plus de les enregistrer dans le fichier log.
\enddesc

\begindesc
\cts tracingcommands {\param{nombre}}
\explain
Si ce param\`etre est \'egal \`a $1$ ou plus,
\TeX\ enregistrera dans le fichier log plus de commandes qu'il ex\'ecute.
Si ^|\tracingonline| est plus grand que z\'ero, cette information appara\^\i tra aussi
sur votre terminal.
composer le premier caract\`ere d'un mot compte comme une commande,
mais (dans le but de la trace seulement) 
les actions de composition des caract\`eres suivants
et de toute ponctuation les suivant
ne compte pas comme des commandes.
Si |\tracingcommands| est \`a $2$ ou plus,
\TeX\ enregistrera aussi des commandes qui sont d\'evelopp\'ees
plut\^ot qu'ex\'ecut\'ees, par exemple, des tests conditionnels et leurs r\'esultats.
\example
\tracingcommands = 1 If $x+y>0$ we quit.\par
On the other hand, \tracingcommands = 0
|
\logproduces
{vertical mode: the letter I}
{horizontal mode: the letter I}
{blank space  }
{math shift character $}
{math mode: the letter x}
{the character +}
{the letter y}
{the character >}
{the character 0}
{math shift character $}
{horizontal mode: blank space  }
{the letter w}
{blank space  }
{the letter q}
{blank space  }
{\par}
{vertical mode: the letter O}
{horizontal mode: the letter O}
{blank space  }
{the letter t}
{blank space  }
{the letter o}
{blank space  }
{the letter h}
{blank space  }
{\tracingcommands}
|
\endexample
\enddesc

\begindesc
\cts tracinglostchars {\param{nombre}}
\explain
Si ce param\`etre est plus grand que z\'ero,
\TeX\ enregistrera une indication dans le fichier log \`a chaque fois
qu'il d\'elaissera un caract\`ere de sortie parce que ce caract\`ere n'existe pas
dans la police courante.
Si ^|\tracingonline| est plus grand que z\'ero, cette information appara\^\i tra aussi
sur votre terminal.
\PlainTeX\ le met par d\'efaut \`a $1$ (contrairement aux autres).
\example
\tracinglostchars = 1
A {\nullfont few} characters.
|
\logproduces
Missing character: There is no f in font nullfont!!
Missing character: There is no e in font nullfont!!
Missing character: There is no w in font nullfont!!
|
\endexample
\enddesc

\begindesc
\cts tracingmacros {\param{nombre}}
\explain
Si ce param\`etre est \`a $1$ ou plus,
\TeX\ enregistrera dans le fichier log le d\'eveloppement et les arguments
de toutes les macros qu'il ex\'ecute.
^^{macros//tracer des}
Si |\tracingmacros| est \`a $2$ ou plus,
\TeX\ enregistrera, en plus,
tout d\'eveloppement de listes de \minref{token} telles que
|\output| or |\everycr|.
Si ^|\tracingonline| est plus grand que z\'ero, cette information appara\^\i tra aussi
sur votre terminal.
\example
\def\a{first \b, then \c}
\def\b{b} \def\c{c}
\tracingmacros = 2
Call \a once.
|
\logproduces
\a ->first \b , then \c 

\b ->b

\c ->c
|
\endexample
\enddesc

\begindesc
\cts tracingoutput {\param{nombre}}
\explain
Si ce param\`etre est sup\'erieur \`a z\'ero,
\TeX\ enregistrera dans le fichier log le contenu de toutes les bo\^\i tes qu'il
envoit dans le \dvifile.
^^{\dvifile//\boite\ enregistr\'e dans le fichier}
Si ^|\tracingonline| est sup\'erieur \`a z\'ero, cette information appara\^\i tra aussi
sur votre terminal.
Le nombre de
points affich\'e dans la log indique le nombre de
niveaux d'inclusion de bo\^\i tes sur cette ligne.
Vous pouvez contr\^oler la quantit\'e de trace en fixant
^|\showboxbreadth| et ^|\showboxdepth| (\xref\showboxbreadth).

mettre |\tracingoutput| \`a $1$ peut \^etre tr\`es pratique quand vous essayez
de d\'eterminer pourquoi vous obtenez un ^{espace suppl\'ementaire} sur une page.

\example
% This is the entire file.
\tracingoutput = 1 \nopagenumbers
One-line page. \bye
|
\logproduces
Completed box being shipped out [1]
\vbox(667.20255+0.0)x469.75499
.\vbox(0.0+0.0)x469.75499, glue set 13.99998fil
..\glue -22.5
..\hbox(8.5+0.0)x469.75499, glue set 469.75499fil
...\vbox(8.5+0.0)x0.0
...\glue 0.0 plus 1.0fil
..\glue 0.0 plus 1.0fil minus 1.0fil
.\vbox(643.20255+0.0)x469.75499, glue set 631.2581fill
..\glue(\topskip) 3.05556
..\hbox(6.94444+1.94444)x469.75499, glue set 386.9771fil
...\hbox(0.0+0.0)x20.0
...\tenrm O
...\tenrm n
...\tenrm e
...\tenrm -
...etc.
..\glue 0.0 plus 1.0fil
..\glue 0.0 plus 1.0fill
.\glue(\baselineskip) 24.0
.\hbox(0.0+0.0)x469.75499, glue set 469.75499fil
..\glue 0.0 plus 1.0fil
|
\endexample
\enddesc

\begindesc
\cts tracingpages {\param{nombre}}
\explain
Si ce param\`etre est sup\'erieur \`a z\'ero,
\TeX\ enregistrera dans le fichier log ses calculs sur le co\^ut des
diff\'erentes coupures de page qu'il essaie.
^^{coupures de page//tracer}
Si |\tracing!-online| ^^|\tracingonline|
est sup\'erieur \`a z\'ero, cette information appara\^\i tra aussi
sur votre terminal.
\TeX\ produit une ligne de cette sortie
\`a chaque fois qu'il commence \`a placer une bo\^\i te ou une \minref{insertion}
sur la liste de page courante, et aussi \`a chaque fois qu'il ex\'ecute une
point de coupure potentiel pour la page.
Examiner cette sortie peut \^etre utile quand vous essayez de d\'eterminer
la cause d'une mauvaise coupure de page.
Voir les \knuth{pages~112--114}{130--133} pour une illustration et une explication de
cette sortie.

Certaines distributions de \TeX\ ignorent la valeur de |\tracingpages|
ainsi elles peuvent s'ex\'ecuter plus rapidement.
Si vous devez utiliser ce param\`etre, soyez certain d'utiliser une distribution qui
lui r\'epond.
\enddesc

\begindesc
\cts tracingparagraphs {\param{nombre}}
\explain
Si ce param\`etre est sup\'erieur \`a z\'ero,
\TeX\ enregistrera dans le fichier log ses calculs sur le co\^ut des
diff\'erentes coupures de ligne qu'il essaie.
^^{coupures de ligne//tracer}
Si ^|\tracingonline| 
est sup\'erieur \`a z\'ero, cette information appara\^\i tra aussi
sur votre terminal.
\TeX\  produit cette sortie quand il atteint la fin de chaque paragraphe.
Voir les \knuth{pages~98--99}{115--116} pour une illustration et une explication de
cette sortie.

Certaines distributions de \TeX\ ignorent la valeur de |\tracing!-para!-graphs|
ainsi elles peuvent s'ex\'ecuter plus rapidement.
Si vous devez utiliser ce param\`etre, soyez certain d'utiliser une distribution qui
lui r\'epond.
\enddesc

\begindesc
\cts tracingrestores {\param{nombre}}
\explain
Si ce param\`etre est sup\'erieur \`a z\'ero,
\TeX\ enregistrera dans le fichier log les valeurs qu'il restitue quand il
rencontre la fin d'un \minref{groupe}.
Si ^|\tracingonline| 
est sup\'erieur \`a z\'ero, cette information appara\^\i tra aussi
sur votre terminal.

Certaines distributions de \TeX\ ignorent la valeur de |\tracing!-restores|
ainsi elles peuvent s'ex\'ecuter plus rapidement.
Si vous devez utiliser ce param\`etre, soyez certain d'utiliser une distribution qui
lui r\'epond.
\enddesc

\begindesc
\cts tracingstats {\param{nombre}}
\explain
Si ce param\`etre est \`a $1$ ou plus, \TeX\ incluera
un rapport sur les ressources qu'il utilise pour ex\'ecuter votre travail
(voir la \knuth{page~300}{350--351} pour une liste et l'explication de ces ressources).
Du reste, si |\tracingstats| est \`a $2$ ou plus,
\TeX\ fera un rapport sur ses utilisations de m\'emoire \`a chaque fois qu'il fera un
^|\shipout| (\xref \shipout) pour une page.
Le rapport appara\^\i t \`a la fin du fichier log. 
^^{fichier log//tracer des statistiques dans le}
Si ^|\tracingonline| est sup\'erieur \`a z\'ero, cette information appara\^\i tra aussi
sur votre terminal.
Si vous avez des probl\`emes avec \TeX\ d\'epassant une de ses
capacit\'es, l'information procur\'ee par |\tracingstats| peut vous aider \`a 
mettre le doigt sur la cause de vos difficult\'es.

Certaines distributions de \TeX\ ignorent la valeur de |\tracingstats|
ainsi elles peuvent s'ex\'ecuter plus rapidement.
Si vous devez utiliser ce param\`etre, soyez certain d'utiliser une distribution qui
lui r\'epond.

L'exemple suivant montre un extrait de 
la sortie de trace que vous obtenez sur une distribution 
de \TeX.  Elle peut \^etre diff\'erente sur d'autres distributions.
{\codefuzz = 1in
\example
\tracingstats=1
|
\logproduces
Here is how much of TeX's memory you used:
 4 strings out of 5540
 60 string characters out of 72328
 5956 words of memory out of 262141
 921 multiletter control sequences out of 9500
 14794 words of font info for 50 fonts, out of 72000 for 255
 14 hyphenation exceptions out of 607
 7i,4n,1p,68b,22s stack positions out of 300i,40n,60p,3000b,4000s
|
\endexample
}% end scope of codefuzz
\enddesc

\begindesc
\cts tracingall {}
\explain
Cette commande demande \`a \TeX\ de
brancher toutes les formes de trace disponibles.
Elle met aussi ^|\tracingonline| \`a $1$ pour que la sortie de trace apparaisse
sur votre terminal.
\enddesc

\begindesc
\cts showboxbreadth {\param{nombre}}
\explain
Ce param\`etre sp\'ecifie le nombre maximum d'\'el\'ements de liste
que \TeX\ affiche pour un niveau d'une bo\^\i te quand il produit
la sortie pour ^|\showbox| ou ^|\tracingoutput|.
\PlainTeX\ met |\showboxbreadth| \`a $5$.
\enddesc

\begindesc
\cts showboxdepth {\param{nombre}}
\explain
Ce param\`etre sp\'ecifie le
niveau de la plus profonde liste que \TeX\ affiche quand
il produit la sortie pour ^|\showbox| ou ^|\showlists|.
\PlainTeX\ met |\showboxdepth| \`a $3$.
\eix^^{tracer}
\eix^^{debugger}
\eix^^{aides au diagnostique}
\enddesc

%==========================================================================
\subsection {Envoyer des messages}

\begindesc
\bix^^{messages, envoyer}
\bix^^{messages d'erreur}
\cts message {\rqbraces{\<liste de token>}}
\cts errmessage {\rqbraces{\<liste de token>}}
\explain
Cette commande affiche le message donn\'e par \<liste de token> sur votre
terminal et l'entre aussi dans la log.  Toutes les \minref{macro}s du
message sont d\'evelopp\'ees, mais aucune commande n'est ex\'ecut\'ee.  C'est la m\^eme r\`egle
que \TeX\ utilise pour |\edef| (\xref \edef).

Pour |\errmessage|, \TeX\ fait une pause 
de la m\^eme fa\c con qu'il le ferait pour un de ses propres messages d'erreur
et affiche les tokens |\errhelp| si vous demandez de l'aide.

Vous pouvez g\'en\'erer des messages multi-ligne en utilisant le caract\`ere ^|\newlinechar|
(\xref \newlinechar).
\example
\message{Starting a new section.}
|
\endexample
\enddesc

\begindesc
\cts wlog {\rqbraces{\<liste de token>}}
\explain
Cette commande \'ecrit \<liste de token> dans le fichier log.
^^{fichier log//\'ecrire par \b\tt\\wlog\e}
\minrefs{log file}
\TeX\ d\'eveloppe \<liste de token> en accord avec les m\^emes r\`egles qu'il utilise
pour |\edef| (\xref\edef).
\example
\wlog{Take two aspirins and call me in the morning.}
|
\logproduces
Take two aspirins and call me in the morning.
|
\endexample
\enddesc

\begindesc
\cts errhelp {\param{liste de token}}
\explain
Ce param\`etre contient la liste de token que \TeX\ affiche
quand vous demandez de l'aide en r\'eponse \`a une commande |\errmessage|.
Nous vous recommandons quand 
vous g\'en\'erez un message d'erreur avec |\errmessage|, de
mettre |\errhelp| \`a une cha\^\i ne qui d\'ecrit la nature de 
l'erreur et utiliser |\newhelp| pour produire cette cha\^\i ne.
Vous pouvez utiliser le caract\`ere ^|\newlinechar| pour produire des messages multi-ligne.
\enddesc

\begindesc
\ctspecial newhelp \ctsxrdef{@newhelp} {\<s\'equence de contr\^ole>
   \rqbraces{\<texte d'aide>}}
\explain
Cette commande assigne le ^{message d'aide} donn\'e par \<texte d'aide> \`a
\<s\'equence de contr\^ole>.  Il procure un moyen efficace de d\'efinir
le ^{texte d'aide} qui compl\`ete un message d'erreur.
Avant d'\'emettre le message d'erreur avec la commande |\errmessage|,
vous devez assigner \<s\'equence de contr\^ole> \`a ^|\errhelp|.  Le texte d'aide
appara\^\i tra alors si l'utilisateur saisi `|H|' 
ou `|h|' en r\'eponse au message d'erreur.
\example
\newhelp\pain{Your input includes a token that I find^^J
   to be offensive. Don't bother me again with this^^J
   document until you've removed it.}
\errhelp = \pain \newlinechar = `\^^J
% ^^J will start a new line
\errmessage{I do not appreciate receiving this token}
|
\logproduces
!! I do not appreciate receiving this token.
l.8 ...t appreciate receiving this token.}
                                          
? H
\Your input includes a token that I find
 to be offensive. Don't bother me again with this
 document until you've removed it. 
|
\endexample
\enddesc

\begindesc
\cts errorcontextlines {\param{nombre}}
\explain
Ce param\`etre d\'etermine le nombre de paires de lignes de contexte,
sans compter la premi\`ere et la derni\`ere, que \TeX\ imprime quand il
rencontre une erreur.  En le mettant \`a $0$ vous pouvez vous d\'ebarrasser de longs
messages d'erreur.  
Vous pouvez forcer le contexte entier en saisissant quelque chose comme~:
\csdisplay
I\errorcontextlines=100\oops
|
en r\'eponse \`a une erreur,
puisque la s\'equence de contr\^ole non d\'efinie |\oops| causera une autre erreur.
\PlainTeX\ met |\error!-context!-lines| \`a $5$.
\enddesc

\see |\write| (\xref \write), |\escapechar| (\xref \escapechar).
\eix^^{messages, envoyer}
\eix^^{messages d'erreur}

%==========================================================================
\section {Initialiser \TeX}

\begindesc
\cts dump {}
\explain
Cette commande, qui ne doit pas appara\^\i tre \`a l'int\'erieur d'un groupe, met en 
r\'eserve le contenu de la m\'emoire de \TeX\ dans un ^{fichier format} (\xref{fichier format}).
En utilisant ^|virtex|, une forme sp\'eciale ``vierge'' de \TeX,
vous pouvez alors recharger le fichier format \`a grande vitesse et 
continuer dans le m\^eme \'etat que \TeX\ \'etait au moment du dump.
|\dump| termine aussi une ex\'ecution.  Puisque |\dump| ne peut \^etre utilis\'e
qu'avec ^|initex|, pas en format de production de \TeX, elle n'est utile qu'aux
gens qui installent \TeX.
\enddesc

\begindesc
\cts everyjob {\param{liste de token}}
\explain
Ce param\`etre contient une liste de \minref{token} que \TeX\ d\'eveloppe au
d\'ebut de chaque ex\'ecution.  Parce qu'un assignement \`a |\everyjob| ne peut affecter
l'ex\'ecution courante (au moment o\`u vous avez fait l'assignement, c'est d\'ej\`a trop
tard), elle n'est utile qu'ux gens qui pr\'eparent des fichiers de format.
\enddesc


\enddescriptions \endchapter \byebye

