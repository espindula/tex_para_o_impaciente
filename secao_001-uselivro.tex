% Isto é parte do livro TeX para o Impaciente.
% Copyright (C) 2003 Paul W. Abrahams, Kathryn A. Hargreaves, Karl Berry.
% Veja o arquivo fdl.tex para condições de reprodução.

\input macros.tex
\chapter{Usando este livro}

\chapterdef{uselivro}

Este livro \'e um guia e manual do tipo ``fa\c ca voc\^e mesmo'' para o 
\TeX.  Aqui nesta se\c c\~ao informamos como usar o livro para o 
m\'aximo de vantagem.  Recomendamos que voc\^e primeiro leia ou fa\c ca 
uma varredura, na sequ\^encia,~\chapternum{uselivro} 
at\'e~\chapternum{examples}, as quais informam o que voc\^e precisa 
saber para come\c car a usar o \TeX.  Se voc\^e j\'a tem experi\^encia 
usando \TeX, ainda ser\'a \'util saber que tipos de informa\c c\~ao 
est\~ao nessas se\c c\~oes do livro.  As 
se\c c\~oes~\chapternum{concepts}--~\chapternum{tips}, as quais ocupam a 
maioria do resto do livro, s\~ao desenhadas para serem acessadas 
aleatoriamente.  No entanto, se voc\^e \'e o tipo de pessoa que gosta de 
ler manuais de refer\^encia, descobrir\'a que \'e poss\'ivel prosseguir 
sequencialmente se estiver disposto a fazer muitos desvios no in\'icio.  
Em ~\chapterref{usingtex}, ``Usando \TeX'', explica-se como produzir um 
documento do \TeX\ a partir de um arquivo de entrada do \TeX{}.  Tamb\'em 
descrevemos as conven\c c\~oes para a prepara\c c\~ao desse arquivo de entrada, 
explana-se um pouco acerca de como \TeX{} funciona, e informamos sobre 
recursos adicionais que est\~ao dispon\'iveis.  A leitura dessa se\c c\~ao 
ajudar\'a a entender os exemplos na pr\'oxima se\c c\~ao.  ~\chapterref{examples}, 
``Exemplos'', cont\'em uma sequ\^encia de exemplos que ilustram o uso de 
\TeX.  Cada exemplo consiste de uma p\'agina de sa\'ida junto com a entrada 
que usamos para cri\'a-la.  Esses exemplos orientar\~ao e ajudar\~ao a 
localizar o material mais detalhado que voc\^e precisar\'a conforme 
prosseguir.  Ao ver quais comandos s\~ao usados na entrada, voc\^e saber\'a 
onde procurar informa\c c\~oes mais detalhadas sobre como obter os efeitos 
mostrados na sa\'ida.  Os exemplos tamb\'em podem servir como modelos para 
documentos simples, embora devamos adverti-lo de que, como tentamos 
armazenar v\'arios comandos \TeX\ em um pequeno n\'umero de p\'aginas, os 
exemplos n\~ao s\~ao necessariamente ilustra\c c\~oes de um design de documento 
bom ou completo.  Ao ler a explica\c c\~ao de um comando, voc\^e pode encontrar 
alguns termos t\'ecnicos n\~ao familiares.  Em~\chapterref{concepts}, 
``Conceitos'', n\'os definimos e explicamos esses termos.  N\'os tamb\'em 
discutimos outros t\'opicos que n\~ao s\~ao abordados em outras partes no 
livro.  A contracapa do livro cont\'em uma lista de todos os conceitos e 
as p\'aginas em que s\~ao descritos.  Sugerimos que voc\^e fa\c ca uma c\'opia 
dessa lista e mantenha-a por perto para que voc\^e possa identificar e 
procurar imediatamente um conceito desconhecido.

Os comandos do \TeX\ s\~ao seu vocabul\'ario principal, e a maior parte deste 
livro \'e dedicada a explic\'a-los.  Nas se\c c\~oes~\chapternum{paras} 
at\'e~\chapternum{general} n\'os descrevemos os comandos.  Voc\^e 
encontrar\'a informa\c c\~ao geral acerca das descri\c c\~oes dos 
comandos em \xrefpg{cmddesc}.  As descri\c c\~oes dos comandos s\~ao 
organizadas funcionalmente, mais ou menos como um dicion\'ario de 
sin\^onimos.  Portanto, se voc\^e sabe o que deseja fazer, mas n\~ao 
sabe qual comando faz isso para voc\^e, pode usar o \'indice para 
gui\'a-lo ao grupo de comandos certo.  Os comandos que consideramos 
particularmente \'uteis e f\'aceis de entender s\~ao indicados com uma 
m\~ao apontada~(\hand).

\chapterref{capsule}, ``Resumo da c\'apsula de comandos'', \'e um 
\'indice especializado que complementa as descri\c c\~oes mais completas 
nas se\c c\~oes~\chapternum{paras}-\chapternum{general}.  O \'indice 
lista os comandos do \TeX\ em ordem alfab\'etica, com uma breve 
explica\c c\~ao de cada comando e uma refer\^encia \`a p\'agina onde \'e 
descrito mais completamente.  O resumo da c\'apsula o ajudar\'a quando 
voc\^e quiser apenas um lembrete r\'apido do que um comando faz.

\TeX\ \'e um programa complexo que ocasionalmente trabalha sua vontade 
de maneira misteriosa.  Em \chapterref{tips}, ``Dicas e T\'ecnicas'', 
n\'os fornecemos conselhos sobre como resolver uma variedade de 
problemas espec\'ificos que voc\^e pode encontrar de tempos em tempos.  
E se voc\^e est\'a perplexo com as mensagens de erro do \TeX, voc\^e 
encontrar\'a socorro em \chapterref{errors}, ``Entendendo as mensagens 
de erro''.  As guias cinzas na lateral do livro ajudar\~ao a localizar 
rapidamente partes do livro.  Elas dividem o livro nas seguintes partes 
principais:
\olist
\li explica\c c\~oes gerais e exemplos
\li conceitos
\li descri\c c\~oes de comandos (cinco guias mais curtas)
\li conselhos, mensagens de erro e macros do |eplain.tex|
\li resumo da c\'apsula de comandos
\li \'indice
\endolist

Em muitos lugares, fornecemos refer\^encias de p\'aginas para 
^{\texbook} (veja \xrefpg{resources} para uma cita\c c\~ao).  Essas 
refer\^encias se aplicam \`a d\'ecima s\'etima edi\c c\~ao do \texbook.  
Para outras edi\c c\~oes, algumas refer\^encias podem estar desativadas 
para uma p\'agina ou duas.


\section Syntactic conventions

Em qualquer livro sobre como preparar entrada para um computador, \'e 
necess\'ario indicar claramente os caracteres literais que devem ser 
digitados e distinguir esses caracteres do texto explicativo.  N\'os 
usamos a fonte ``Computer Modern typewriter'' para 
{\tt entrada literal como esta}, e tamb\'em para os nomes de comandos do 
\TeX{}.  Quando existir qualquer possibilidade de confus\~ao, colocamos 
a entrada do \TeX\ entre aspas simples `{\tt como isto}'.  No entanto, 
ocasionalmente usamos par\^enteses quando estamos indicando caracteres 
\'unicos, como (|`|) (voc\^e pode ver o porqu\^e).

Para o bem dos seus olhos, normalmente colocamos espa\c cos onde voc\^e 
deve colocar espaços.  Em alguns lugares onde precisamos enfatizar o 
espa\c co, entretanto, usamos um carácter `\visiblespace' 
{\catcode `\ =\other\pix^^| |} para indic\'a-lo.  Naturalmente, esse 
carácter \'e chamado de \emph{espa\c co vis\'ivel}.  \pix^^{spaces//visible}


\section Descri\c c\~oes dos comandos

\xrdef{cmddesc}
As se\c c\~oes ~\chapternum{paras}--\chapternum{general} cont\'em uma 
descri\c c\~ao do qu\^e aproximadamente cada comando do \TeX\ faz.  
^^{commands}Ambos, os comandos primitivos ^^{primitive//command} e 
aqueles do ^{\plainTeX} s\~ao abordados.  Os comandos primitivos s\~ao 
aqueles constru\'idos dentro do programa de computador \TeX, enquanto 
que os comandos do \plainTeX{} est\~ao definidos em um arquivo padr\~ao 
de defini\c c\~oes auxiliares (veja \xref\plainTeX).  Os \'unicos 
comandos que n\'os omitimos foram aqueles que s\~ao usados puramente 
localmente na defini\c c\~ao do \plainTeX\ (Sequ\^encias B\'asicas de 
Controle, \knuth{Appendix~B}).  Os comandos est\~ao organizados conforme 
a seguir:
\ulist\compact
\li ``Comandos para a composi\c c\~ao de par\'agrafos'', 
\chapterref{paras}, lida com caracteres, palavras, linhas e par\'agrafos 
inteiros;
\li ``Comandos para a composi\c c\~ao de p\'aginas'', \chapterref{pages}, 
lida com p\'aginas, seus componentes, e a rotina de sa\'ida;
\li ``Comandos para os modos horizontal e vertical'', 
\chapterref{hvmodes}, tem formas correspondentes ou id\^enticas para 
ambos os modos horizontais (par\'agrafos e hboxes) e modos verticais 
(p\'aginas e vboxes).  Esses comandos fornecem caixas, espa\c cos, 
regras, l\'ideres e alinhamentos;
\li ``Comandos para a composi\c c\~ao de f\'ormulas matem\'aticas'', 
\chapterref{math}, fornecem recursos para a constru\c c\~ao de 
f\'ormulas matem\'aticas;
\li ``Comandos para opera\c c\~oes gerais'', \chapterref{general}, 
fornecem recursos de programa\c c\~ao do \TeX e tudo o mais que n\~ao se 
encaixa em nenhuma das outras se\c c\~oes.
\endulist
Voc\^e deve pensar nessas categorias como sendo mais sugestivas do que 
rigorosas, porque os comandos n\~ao se encaixam perfeitamente nessas (ou 
em quaisquer outras) categorias.

Dentro de cada se\c c\~ao, as descri\c c\~oes dos comandos est\~ao 
organizadas por fun\c c\~ao.  Quando v\'arios comandos est\~ao 
intimamente relacionados, eles est\~ao descritos como um grupo; caso 
contr\'ario, cada comando tem sua pr\'opria explica\c c\~ao.  A 
descri\c c\~ao de cada comando inclui um ou mais exemplos e a sa\'ida 
produzida por cada exemplo, quando os exemplos s\~ao apropriados (para 
alguns comandos eles n\ão s\~ao).  Quando voc\^e estiver olhando para 
uma se\c c\~ao contendo comandos funcionalmente relacionados, 
certifique-se de verificar o final de uma subse\c c\~ao para um item 
``ver tamb\'em'' que o direciona para comandos relacionados que est\~ao 
descritos em outro lugar.

Alguns comandos est\~ao intimamente relacionados a certos conceitos.  
Por exemplo, os comandos |\halign| e |\valign| est\~ao relacionados a 
``alinhamento''; o comando |\def| est\'a relacionado a ``macro''; e os 
comandos |\hbox| e |\vbox| est\~ao relacionados a ``caixa''.  Nesses 
casos, n\'os geralmente demos uma descri\c c\~ao b\'asica dos pr\'oprios 
comandos e explicamos as id\'eias subjacentes.

Os exemplos associados aos comandos foram tipografados com 
^|\parindent|, o recuo de par\'agrafo definido como zero, de modo que 
os par\'agrafos normalmente não são recuados.  Essa conven\c c\~ao faz 
com que os exemplos sejam mais f\'aceis de ler.  Naqueles exemplos onde 
o recuo de par\'agrafo \'e essencial, n\'os configuramos explicitamente 
para um valor diferentes de zero.

A mão apontada na frente de um comando ou de um grupo de comandos indica 
que julgamos que esse comando ou grupo de comandos \'e particularmente 
\'util e f\'acil de entender.

Muitos comandos experam ^{argumentos} de um tipo ou de outro 
(\xref{arg1}).  Os argumentos de um comando fornecem ao \TeX\ 
informa\c c\~oes adicionais necess\'arias para executar o comando.  Cada 
argumento \'e indicado por um termo em it\'alico, entre colchetes 
angulares, que indica que tipo de argumento \'e:

\display{%
\halign{\<#>\quad&#\hfil\cr
argument&token \'unico ou algum texto entre chaves\cr
charcode&um c\'odigo de caractere, ou seja, um inteiro entre $0$ e 
$255$\cr
dimen&uma dimens\~ao, isto \'e, um comprimento\cr
glue&cola (com estiramento e encolhimento opcionais)\cr
number&um inteiro opcionalmente assinado (n\'umero inteiro)\cr
register&um n\'umero de registro entre $0$ e $255$\cr
}}
^^{\<argument>}
^^{\<charcode>}
^^{\<dimen>}
^^{\<glue>}
^^{\<number>}
^^{\<register>}

\noindent
Todos esses termos est\~ao explicados com mais detalhes em 
\chapterref{concepts}.  Al\'em disso, as vezes usamos termos como 
\<lista de tokens> que s\~ao auto explicativos ou explicados na 
descri\c c\~ao do comando.  Alguns comandos tem formatos especiais que 
exigem chaves ou palavras espec\'ificas.  Esses est\~ao definidos na 
mesma fonte em negrito que usamos para os cabe\c calhos de comando.

Alguns comandos s\~ao par\^ametros (\xref{introparms}) ou entradas de 
tabela.  ^^{parameters//as commands}
Isso \'e indicado na listagem do comando.  Voc\^e pode usar um 
par\^ametro como argumento ou atribuir um valor ao comando.  O mesmo 
vale para entradas de tabela.  N\'os usamos o termo ``par\^ametro'' para 
nos referirmos a entidades como |\pageno| que, na verdade, s\~ao 
registros, mas se comportam exatamente como par\^ametros.
^^{registers//parameters as}


\endchapter
\byebye
