% This is part of the book TeX for the Impatient.
% Copyright (C) 2003 Paul W. Abrahams, Kathryn A. Hargreaves, Karl Berry.
% See file fdl.tex for copying conditions.

\input macros
\frontchapter{Pref\'acio}

{\tighten
O \TeX\ de Donald Knuth, um sistema computadorizado de tipografia, 
fornece quase tudo o que \'e necess\'ario para tipografia de alta 
qualidade de nota\c c\~oes matem\'aticas, bem como de texto comum.  \'E 
particularmente not\'avel por sua flexibilidade, sua excelente 
hifeniza\c c\~ao e sua capacidade de escolher quebras de linha 
esteticamente satisfat\'orias.  Devido \`as suas extraordin\'arias 
capacidades, o \TeX\ tornou-se o principal sistema tipogr\'afico para 
matem\'atica, ci\^encia e engenharia e foi adotado como padr\~ao pela 
American Mathematical Society.  Um programa companheiro, ^{\Metafont}, 
pode construir formas de letras arbitr\'arias incluindo, em particular, 
quaisquer s\'imbolos que possam ser necess\'arios em matem\'atica.  
Ambos \TeX\ e \Metafont\ est\~ao amplamente acess\'iveis dentro da 
comunidade cient\'ifica e de engenharia e foram implementados em uma 
variedade de computadores.  \TeX\ n\~ao \'e perfeito---falta suporte 
integrado a gr\'aficos, e alguns efeitos, como ^{revision bars}, s\~ao 
muito dif\'iceis de produzir---mas esses inconvenientes s\~ao muito 
compensados por suas vantagens.
\par}

\thisbook\/ destina-se a servir cientistas, matem\'aticos e tip\'ografos 
t\'ecnicos para quem \TeX\ \'e uma ferramenta \'util e n\~ao um 
interesse primordial, bem como pessoas de computa\c c\~ao que tem um 
forte interesse em \TeX\ por sua pr\'opria causa.  N\'os tamb\'em 
pretendemos que ele sirva tanto aos novatos em \TeX\ e \`aqueles que 
j\'a est\~ao familiarizados com \TeX.  Assumimos que nossos leitores 
est\~ao confort\'aveis em trabalhar com computadores e querem obter as 
informa\c c\~oes de que precisam o mais r\'apido poss\'ivel.  Nosso 
objetivo \'e fornecer essas informa\c c\~oes de forma clara, concisa e 
acess\'ivel.
{\tighten Este livro, portanto, fornece um holofote luminoso, uma 
bengala robusta e mapas detalhados para explorar e usar o \TeX.  Ele 
permitir\'a que voc\^e domine \TeX\ em um ritmo r\'apido via consulta e 
experi\^encia, mas n\~ao o conduzir\'a pela m\~ao ao longo de todo o 
sistema \TeX.  Nossa abordagem \'e fornecer a voc\^e um manual para 
\TeX\ que facilita a recupera\c c\~ao de qualquer informa\c c\~ao que 
voc\^e precise.  N\'os explicamos o repert\'orio completo de comandos do 
\TeX\ e os conceitos que os fundamentam.  Voc\^e n\~ao ter\'a que 
desperdi\c car seu tempo arando com materiais que voc\^e nem precisa nem 
quer.  \par}

Nas se\c c\~oes iniciais n\'os tamb\'em fornecemos orienta\c c\~ao 
suficiente para que voc\^e possa come\c car se voc\^e n\~ao tiver usado 
\TeX\ antes.  Presumimos que voc\^e tenha acesso a uma 
implementa\c c\~ao de \TeX\ e que voc\^e saiba como usar um editor de 
texto, mas n\~ao presumimos muito sobre seu hist\'orico.  Como este 
livro est\'a organizado para pronta refer\^encia, voc\^e continuar\'a a 
ach\'a-lo \'util a medida que se tornar mais familiarizado com o \TeX.  
Se voc\^e preferir come\c car com um tour cuidadosamente guiado, 
recomendamos que voc\^e primeiro leia o ^{\texbook} de Knuth (veja 
\xrefpg{resources} para uma cita\c c\~ao), passando por cima das 
se\c c\~oes de curva perigosa (``dangerous bend''), e ent\~ao retorne 
para este livro para informa\c c\~oes adicionais e para refer\^encia 
quando voc\^e come\c car a usar \TeX.  (As se\c c\~oes de curva perigosa 
do \texbook\ cobrem t\'opicos avan\c cados).


A estrutura de \TeX\ \'e realmente muito simples: um documento de 
entrada do \TeX\ consiste em texto comum intercalado com comandos que 
d\~ao ao \TeX\ instru\c c\~oes adicionais sobre como tipografar o seu 
documento.  Coisas como f\'ormulas matem\'aticas cont\'em muitos desses 
comandos, enquanto o texto expositivo cont\'em relativamente poucos 
deles.  A parte que desperdi\c ca tempo de aprender \TeX\ \'e aprender 
os comandos e conceitos que fundamentam suas descri\c c\~oes.  Assim, 
n\'os devotamos a maior parte do livre para definir e explanar os 
comandos e os conceitos.  N\'os tamb\'em fornecemos exemplos mostrando a 
sa\'ida da tipografia de \TeX\ e a correspondente entrada, dicas sobre 
solucionar problemas comuns, informa\c c\~ao acerca de mensagens de erro 
e assim por diante.  N\'os fornecemos refer\^encias cruzadas extensa por 
n\'umero de p\'agina e um \'incide completo.

N\'os organizamos as descri\c c\~oes dos comandos de forma que voc\^e 
pode procurar por eles pela fun\c c\~ao ou alfabeticamente.  A 
organiza\c c\~ao funcional \'e o que voc\^e precisa quando voc\^e sabe o 
que deseja fazer, mas n\~ao sabe que comando pode faz\ê-lo para voc\^e.  
A organiza\c c\~ao alfabetica \'e o que voc\^e precisa quando sabe o 
nome de um comando, mas n\~ao sabe exatamente o que ele faz.

N\'os devemos adverti-lo de que n\~ao tentamos fornecer uma 
defini\c c\~ao completa de \TeX.  Para isso voc\^e precisar\'a do 
^{\texbook}, que \'e a fonte original de informa\c c\~ao sobre \TeX.  
\texbook\ tamb\'em cont\'em um monte de informa\c c\~ao sobre os pontos 
finos do uso de \TeX, particularmente sobre o assunto da composi\c c\~ao 
de f\'ormulas matem\'aticas.  N\'os o recomendamos enfaticamente.

Em 1989, Knuth fez uma grande revis\~ao no \TeX\ com a finalidade de 
adapt\'a-lo aos conjuntos de caracteres de $8$ bits, necess\'arios para 
suportar a tipografia para outras linguagens que n\~ao o Ingl\^es.  A 
descri\c c\~ao do \TeX\ neste livro incorpora essa revis\~ao (veja 
\xref{newtex}).

{\tighten Voc\^e pode estar usando uma forma especializada de \TeX\, 
como ^{\LaTeX} ou ^{\AMSTeX} (veja \xref{resources}).  Apesar dessas 
formas especializadas serem auto-contidas, voc\^e ainda pode desejar 
usar algumas das facilidades do pr\'oprio \TeX\ de vez em quando, de 
forma a conseguir um controle mais apurado que somente \TeX\ pode 
fornecer.  Este livro pode ajud\'a-lo a aprender o que precisa saber 
acerca dessas facilidades sem ter que aprender sobre um monte de outras 
coisas nas quais voc\^e n\~ao est\'a interessado.  \par}

Dois de n\'os (K.A.H. e K.B.) fomos generosamente financiados pela 
Universidade de Massachusetts em Boston durante a prepara\c c\~ao deste 
livro.  Em particular, Rick Martin manteve as m\'aquinas em 
execu\c c\~ao, e Robert~A. Morris e Betty O'Neil tornaram as m\'aquinas 
dispon\'iveis.  Paul English da Interleaf nos ajudou a produzir provas 
para o desenho de capa.

Desejamos agradecer aos revisores do nosso livro: Richard Furuta da 
Universidade de Maryland; John Gourlay da Arbortext, Inc.; Jill Carter
Knuth e Richard Rubinstein da Digital Equipment Corporation.








We took to heart their perceptive and unsparing criticisms of the original manuscript, and the book has benefitted greatly from their insights.

We are particularly grateful to our editor, Peter Gordon of
Addison-Wesley.  This book was really his idea, and throughout its
development he has been a source of encouragement and valuable
advice.  We thank his assistant at Addison-Wesley, Helen Goldstein, for
her help in so many ways, and Loren Stevens of Addison-Wesley for her
skill and energy in shepherding this book through the production
process.  Were it not for our copyeditor, Janice Byer, a number of small
but irritating errors would have remained in this book.  We appreciate
her sensitivity and taste in correcting what needed to be corrected
while leaving what did not need to be corrected alone.  Finally, we wish
to thank Jim Byrnes of Prometheus Inc. for making this collaboration
possible by introducing us to each other.
\vskip1.5\baselineskip

\line{\it Deerfield, Massachusetts\hfil\rm P.\thinspace W.\thinspace A.}
\line{\it Manomet, Massachusetts\hfil\rm K.\thinspace A.\thinspace H.,
       K.\thinspace B.}

\vskip2\baselineskip

\noindent {\bf Preface to the free edition:} This book was originally
published in 1990 by Addison-Wesley.  In 2003, it was declared out of
print and Addison-Wesley generously reverted all rights to us, the
authors.  We decided to make the book available in source form, under
the GNU Free Documentation License, as our way of supporting the
community which supported the book in the first place.  See the
copyright page for more information on the licensing.

The illustrations which were part of the original book are not included
here.  Some of the fonts have also been changed; now, only
freely-available fonts are used.  We left the cropmarks and galley
information on the pages, to serve as identification.  An old version of
Eplain was used to produce it; see the {\tt eplain.tex} file for
details.

We don't plan to make any further changes or additions to the book
ourselves, except for correction of any outright errors reported to us,
and perhaps inclusion of the illustrations.

Our distribution of the book is at {\tt
ftp://tug.org/tex/impatient}.  You can reach us by email at {\tt
impatient@tug.org}.

\pagebreak
\byebye
